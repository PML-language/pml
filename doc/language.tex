\chapter{PML language and syntax}

\section{Sorts}

PML language uses \emph{sorts} to classify all the expressions of the
languages (excep sorts) like types, values, ordinals, \dots

\begin{tabular}{ll}
  $\iota$, \verb!<iota>! or \verb!<value>!
  & denote the sort for values i.e. results of computation. \\

  $\tau$, \verb!<tau>! or \verb!<term>!
  & denote the sort for terms i.e. program that evaluates to value. \\

  $\sigma$, \verb!<sigma>! of \verb!<stack>!
  & denote the sort of stacks which is used when programming with classical
  logic.\\

  $o$, \verb!<omicron>! or \verb!<prop>!
  & denote the sort of types for terms or propositions (following the
  Curry-Howard correspondance PML identifies types and proposition).
  $o$ is the unicode character omicron, not a lower case latin letter. \\

  $\kappa$, \verb!<kappa>! or \verb!<ordinal>!
  & denote the sort of ordinals used to index inductive and co-inductive
  types. \\

  \emph{sort} $\rightarrow$ \emph{sort}
  & the sort for higher-order function like type with parameters. These
  higher-order function should not be confused with functions as programs
  which are of sort $\iota$ or $\tau$. \\

  \verb!(! \emph{sort} \verb!)!
  & parenthesis can be used for grouping \\
\end{tabular}

The sort $\iota$ is a subsort of $\tau$. This means that any value can also
be considered as a term.

\section{Expressions}
