\documentclass[oneside,a4]{book}

\usepackage{amssymb,amsmath,amsthm}
\usepackage[mathletters]{ucs}
\usepackage[utf8x]{inputenc}
\usepackage{bussproofs}
\usepackage{enumitem}
\usepackage{caption}
\usepackage{xspace}
\usepackage{subcaption}
\usepackage{stmaryrd}
\usepackage{hyperref}
\usepackage{geometry}

\newcommand{\bnfeq}{\,\mathrel{::=}\;\,}
\newcommand{\bnfor}{\;\mathrel{\vert}\;}
\newcommand{\bnfst}{\\&\vert\;}
\renewcommand{\emptyset}{\varnothing}
\newcommand{\st}{\mid}
\newcommand{\pml}{$\text{PML}_2$\xspace}
\renewcommand\labelitemi{---}
\setlist[itemize]{noitemsep,nolistsep}

\newcommand*{\minw}[2]{%
  \makebox[#2]{#1}%
}

\newcommand{\vars}{\mathcal{V}}
\newcommand{\lvars}{\mathcal{V}_{ι}}
\newcommand{\tvars}{\mathcal{V}_{τ}}
\newcommand{\kvars}{\mathcal{V}_{κ}}
\newcommand{\svars}{\mathcal{V}_{σ}}
\newcommand{\ovars}{\mathcal{V}_{ο}}
\newcommand{\cnsts}{\mathcal{C}}
\newcommand{\props}{\mathcal{P}}
\newcommand{\sorts}{\mathcal{S}}
\newcommand{\terms}{\Lambda}
\newcommand{\types}{\mathcal{F}}
\newcommand{\N}{\mathbb{N}}
\newcommand{\arf}[1]{\lvert #1 \rvert_1}
\newcommand{\ars}[1]{\lvert #1 \rvert_2}
\newcommand{\sem}[1]{\ensuremath{\llbracket #1 \rrbracket}}

\newtheorem{definition}{Definition}
\newtheorem{example}{Example}
\newtheorem{lemma}{Lemma}
\newtheorem{theorem}{Theorem}
\newtheorem{corollary}{Corollary}
\newtheorem{remark}{Remark}

\begin{document}

Indices of arrows and $\vdash$ is a subset of $\{cl\}$.

Indices of semantics is a natural, $\bullet$, or $\infty$.

$B$ is parametrized by a functionnnal taking  a natural, $\bullet$
or $\infty$ as argument and retruning a set of terms.

The sort $o$ can be interpreted as set of values like the {\em default} interpretation
$\sem{\dots}_\bullet$. Others are only used for interpretation of arrows and
statement of the adequacy lemma.

Remark: epsilons will be interpreted in set $\sem{\dots}_\infty$. $\phi a.a$
can replace the old $\square$ term.

  \begin{align*}
    B_\bullet(X) &= X(\bullet)\\
    B_\infty(X)  &= \bigcap_{n \in \N} B_n(X) \\
    B_0(X) &= \Lambda     \\
    B_{n+1}(X) &= X(n+1) \cup \{ u = (\phi a.t) t_1 \dots t_n |
    t[x:=u] t_1 \dots t_n \in B_n(X)\} \\
    \sem{\Phi}_n &= B_n(\lambda p.\Phi) \text{ if } \Phi \in \sem{o}\\
    \sem{A \Rightarrow_{cl} B}_n &=
      B_n(\lambda p. \{\lambda x.t \;|\; \forall v \in \sem{A}_\bullet,
      t[x := v] \in \sem{B}_p^{\bot\bot}\})\\
    \sem{A \Rightarrow_{c} B}_n &=
      B_n(\lambda p. \{\lambda x.t \;|\; \forall v \in \sem{A}_\bullet,
      t[x := v] \in \sem{B}_\bullet^{\bot\bot}\})\\
    \sem{A \Rightarrow_{l} B}_n &=
      B_n(\lambda p. \{\lambda x.t \;|\; \forall v \in \sem{A}_\bullet,
      t[x := v] \in \sem{B}_p\})\\
    \sem{A \Rightarrow B}_n &=
      B_n(\lambda p. \{\lambda x.t \;|\; \forall v \in \sem{A}_\bullet,
       t[x := v] \in \sem{B}_\bullet)\\
    \sem{[(C_i : A_i)_{i \in I}]}_n
      &=   B_n(\lambda p. \bigcup_{i \in I} \{C_i[v] \;|\; v \in
    \sem{A_i}_\bullet \})\\
    \sem{\{(l_i : A_i)_{i \in I}; \dots\}}_n
      &= B_n(\lambda p. \{\{(l_i = v_i)_{i \in I \cup K}\} \;|\; \forall i \in I, v_i \in \sem{A_i}_\bullet\})\\
    \sem{\{(l_i : A_i)_{i \in I}\}}_n
      &= B_n(\lambda p. \{\{(l_i = v_i)_{i \in I}\} \;|\; \forall i \in I,
    v_i \in \sem{A_i}_\bullet \})\\
    \sem{\forall \chi:s.A}_n
      &= \bigcap_{\Phi \in \sem{s}}
           \sem{A[\chi := \Phi]}_n\\
    \sem{\exists \chi:s.A}_n
      &= \bigcup_{\Phi \in \sem{s}}
           \sem{A[\chi := \Phi]}_n\\
    \sem{\mu_\tau X.A}_n
      &=  \bigcup_{o < \sem{\tau}}
            \sem{A[X := \mu_o X.A]}_n\\
    \sem{\nu_\tau X.A}_n
      &= \lambda p. \bigcap_{o < \sem{\tau}}
            \sem{A[X := \nu_o X.A]}_p\\
    \sem{t \in A}_n
      &= B_n(\lambda p. \{v \in \sem{A}_p \;|\; \sem{t}
      \equiv v\})\\
    \sem{A \restriction t \equiv u}_n
       &= \begin{cases}
           B_n(\lambda p. \sem{A} \text{ if } \sem{t}
             \equiv \sem{u})\\
           B_n(\lambda p.\emptyset) \text{ otherwise}
    \end{cases}
  \end{align*}

\end{document}
