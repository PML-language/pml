\begin{definition}[atomic sorts]
  We denote $\sorts₀ = \{ι,τ,σ,κ,ο\}$ the set of our five atomic sorts. It
  contains the sort of values $ι$, the sort of terms $τ$, the sort of stacks
  $σ$, the sort of ordinals $κ$ and the sort of propositions $ο$.
\end{definition}

\begin{definition}[sorts]
  The set of all sorts $\sorts$ is generated from the set of atomic sorts
  $\sorts₀$ using the following grammar.
  \begin{align*}
    s,r &\bnfeq φ \bnfor s→r \bnfor s×r \tag{$φ ∈ \sorts$}
  \end{align*}
\end{definition}

\begin{definition}[variables]
  For every sort $s ∈ \sorts$ we require a (distinct) countable set of
  variables $\vars_s$. In particular, we will use the following sets.
  \begin{align*} %
    \lvars &= \{x,y,z\dots\} \tag{value variables} \\
    \svars &= \{α,β,γ\dots\} \tag{stack variables}
  \end{align*}
\end{definition}




One of the singularities of the \pml language is that it is built on
semantic intuitions. In particular, the interpretation of types precedes
the typing and subtyping rules. This means that we give ourselves a
semantics, and then only consider typing rules that are adequate with
respect to this semantics.

\begin{figure}
  \begin{align*}
    E,F \bnfeq&\\
    &\hspace{-1cm}\text{(Higher-order components)}\\
    \bnfor &\Phi                   \tag{element of a semantic domain}\\
    \bnfor &\chi                   \tag{variable}\\
    \bnfor &(\chi:s \mapsto E)     \tag{higher-order function}\\
    \bnfor &E\langle F \rangle     \tag{higher-order application}\\
    \bnfor &\varepsilon_{\chi:s}(E \in F) \bnfor
      \varepsilon_{\chi:s}(E \notin F)
      \tag{Choice operators for quantifiers}\\
    &\hspace{-1cm}\text{(Value components)}\\
    \bnfor &\lambda \chi.E
      \tag{$\lambda$-abstraction, $\chi \in \mathcal{V}_\iota$}\\
    \bnfor &C_k[E]                 \tag{constructor, $k \in \mathbb{N}$}\\
    \bnfor &\{(l_i=E_i)_{i \in I}\}\tag{record, $I \subset_{fin} \mathbb{N}$}\\
    \bnfor &\varepsilon_{\chi \in E_1}(F \notin E_2)
      \tag{value choice operator, $\chi \in \mathcal{V}_\iota$}\\
    &\hspace{-1cm}\text{(Term components)}\\
    \bnfor &E\;F                   \tag{application}\\
    \bnfor &\mu\chi.E
      \tag{$\mu$-abstraction, $\chi \in \mathcal{V}_\sigma$}\\
    \bnfor &[E]F                   \tag{named term}\\
    \bnfor &E.l_k                  \tag{record projection, $k \in \mathbb{N}$}\\
    \bnfor &[E|(C_i[\chi_i] \to E_i)_{i \in I}]
      \tag{pattern matching, $I \subset_{fin} \mathbb{N}$, $\chi_i \in
      \mathcal{V}_\iota$}\\
    \bnfor &\varphi \chi.E
      \tag{fixed-point for recursion, $\chi \in \mathcal{V}_\tau$}\\
    &\hspace{-1cm}\text{(Stack components)}\\
    \bnfor &\varepsilon            \tag{empty stack}\\
    \bnfor &[-\;E]F                \tag{pushed value argument}\\
    \bnfor &[E\;-]F                \tag{pushed function}\\
    \bnfor &\varepsilon_{\chi \in \lnot E}(F \notin E)
      \tag{value choice operator, $\chi \in \mathcal{V}_\sigma$}\\
    &\hspace{-1cm}\text{(Ordinal components)}\\
    \bnfor &\infty                 \tag{converging ordinal}\\
    \bnfor &E+1                    \tag{successor ordinal}\\
    \bnfor &\varepsilon_{\chi<E}(F_1 \in F_2) \bnfor
      \varepsilon_{\chi<E}(F_1 \notin F_2)
      \tag{Choice operators for ordinals}\\
    &\hspace{-1cm}\text{(Proposition components)}\\
    \bnfor &E \Rightarrow F        \tag{function type}\\
    \bnfor &[(C_i : A_i)_{i \in I}]
      \tag{sum type, $I \subset_{fin} \mathbb{N}$}\\
    \bnfor &\{(l_i : A_i)_{i \in I}; \dots\} \bnfor \{(l_i : A_i)_{i \in I}\}
      \tag{(extensible) product type, $I \subset_{fin} \mathbb{N}$}\\
    \bnfor &\forall \chi:s.E \bnfor \exists \chi:s.E
      \tag{Quantifiers}\\
    \bnfor &\mu_E \chi.F \bnfor \nu_E \chi.E
      \tag{sized (co-)inductive type}\\
    \bnfor &E \in F                \tag{membership type}\\
    \bnfor &E \restriction F_1 \equiv F_2 \tag{restriction type}
  \end{align*}
  \caption{Full syntax of expressions.}
  \label{expr}
\end{figure}
\begin{figure}
  \begin{prooftree}
    \AxiomC{}
    \UnaryInfC{$\Gamma, \chi : s \vdash \chi : s$}
    \DisplayProof\hfill
    \AxiomC{$\Gamma, \chi : s \vdash E : t$}
    \UnaryInfC{$\Gamma \vdash (\chi : s \mapsto t) : s \to t$}
    \DisplayProof\hfill
    \AxiomC{$\Gamma \vdash E : s \to t$}
    \AxiomC{$\Gamma \vdash F : s$}
    \BinaryInfC{$\Gamma \vdash E \langle F \rangle : t$}
  \end{prooftree}
  \begin{prooftree}
    \AxiomC{$\Phi \in \sem{s}$}
    \UnaryInfC{$\Gamma \vdash \Phi : s$}
    \DisplayProof\hfill
    \AxiomC{$\Gamma \vdash E : \tau$}
    \AxiomC{$\Gamma, \chi:s \vdash F : o$}
    \BinaryInfC{$\Gamma \vdash \varepsilon_{\chi:s}(E \in F) : s$}
    \DisplayProof\hfill
    \AxiomC{$\Gamma \vdash E : \tau$}
    \AxiomC{$\Gamma, \chi:s \vdash F : o$}
    \BinaryInfC{$\Gamma \vdash \varepsilon_{\chi:s}(E \notin F) : s$}
  \end{prooftree}
  \begin{prooftree}
    \AxiomC{$\Gamma, \chi : \iota \vdash E : \tau$}
    \UnaryInfC{$\Gamma \vdash \lambda \chi E : \iota$}
    \DisplayProof\hfill
    \AxiomC{$\Gamma \vdash E : \iota$}
    \UnaryInfC{$\Gamma \vdash C_k[E] : \iota$}
    \DisplayProof\hfill
    \AxiomC{$(\Gamma \vdash E_i : \iota)_{i \in I}$}
    \UnaryInfC{$\Gamma \vdash \{(l_i = E_i)_{i \in I}\} : \iota$}
  \end{prooftree}
  \begin{prooftree}
    \AxiomC{$\Gamma \vdash E_1 : o$}
    \AxiomC{$\Gamma \vdash E_2 : o$}
    \AxiomC{$\Gamma, \chi : \iota \vdash F : \tau$}
    \TrinaryInfC{$\Gamma\vdash\varepsilon_{\chi\in E_1}(F\notin E_2) : \iota$}
    \DisplayProof\hfill
    \AxiomC{$\Gamma \vdash E : \iota$}
    \UnaryInfC{$\Gamma \vdash E : \tau$}
  \end{prooftree}
  \begin{prooftree}
    \AxiomC{$\Gamma \vdash E : \tau$}
    \AxiomC{$\Gamma \vdash F : \tau$}
    \BinaryInfC{$\Gamma \vdash E\;F : \tau$}
    \DisplayProof\hfill
    \AxiomC{$\Gamma, \chi : \sigma \vdash E : \tau$}
    \UnaryInfC{$\Gamma \vdash \mu \chi.E : \tau$}
    \DisplayProof\hfill
    \AxiomC{$\Gamma \vdash E : \sigma$}
    \AxiomC{$\Gamma \vdash F : \tau$}
    \BinaryInfC{$\Gamma \vdash [E]F : \tau$}
  \end{prooftree}
  \begin{prooftree}
    \AxiomC{$\Gamma \vdash E : \iota$}
    \UnaryInfC{$\Gamma \vdash E.l_k : \tau$}
    \DisplayProof\hfill
    \AxiomC{$\Gamma \vdash E : \iota$}
    \AxiomC{$(\Gamma, \chi_i : \iota \vdash E_i : \tau)_{i \in I}$}
    \BinaryInfC{$\Gamma \vdash [E|(C_i[\chi_i] \to E_i)_{i \in I}] : \tau$}
    \DisplayProof\hfill
    \AxiomC{$\Gamma, \chi : \tau \vdash E : \iota$}
    \UnaryInfC{$\Gamma \vdash \varphi \chi.E : \tau$}
  \end{prooftree}
  \begin{prooftree}
    \AxiomC{}
    \UnaryInfC{$\Gamma \vdash \varepsilon : \sigma$}
    \DisplayProof\hfill
    \AxiomC{$\Gamma \vdash E : \iota$}
    \AxiomC{$\Gamma \vdash F : \sigma$}
    \BinaryInfC{$\Gamma \vdash [-\;E]F : \sigma$}
    \DisplayProof\hfill
    \AxiomC{$\Gamma \vdash E : \tau$}
    \AxiomC{$\Gamma \vdash F : \sigma$}
    \BinaryInfC{$\Gamma \vdash [E\;-]F : \sigma$}
  \end{prooftree}
  \begin{prooftree}
    \AxiomC{$\Gamma \vdash E : o$}
    \AxiomC{$\Gamma, \chi : \sigma \vdash F : \tau$}
    \BinaryInfC{$\Gamma \vdash \varepsilon_{\chi \in \lnot E}(F \notin E)
      : \sigma$}
    \DisplayProof\hfill
    \AxiomC{}
    \UnaryInfC{$\Gamma \vdash \infty : \kappa$}
    \DisplayProof\hfill
    \AxiomC{$\Gamma \vdash E : \kappa$}
    \UnaryInfC{$\Gamma \vdash E+1 : \kappa$}
  \end{prooftree}
  \begin{prooftree}
    \AxiomC{$\Gamma \vdash E : \kappa$}
    \AxiomC{$\Gamma \vdash F_1 : \tau$}
    \AxiomC{$\Gamma, \chi : \kappa \vdash F_2 : o$}
    \TrinaryInfC{$\Gamma \vdash \varepsilon_{\chi<E}(F_1 \in F_2) : \kappa$}
    \DisplayProof\hfill
    \AxiomC{$\Gamma, \chi : s \vdash E : o$}
    \UnaryInfC{$\Gamma \vdash \forall \chi:s.E : o$}
  \end{prooftree}
  \begin{prooftree}
    \AxiomC{$\Gamma \vdash E : \kappa$}
    \AxiomC{$\Gamma \vdash F_1 : \tau$}
    \AxiomC{$\Gamma, \chi : \kappa \vdash F_2 : o$}
    \TrinaryInfC{$\Gamma \vdash \varepsilon_{\chi<E}(F_1 \notin F_2):\kappa$}
    \DisplayProof\hfill
    \AxiomC{$\Gamma, \chi : s \vdash E : o$}
    \UnaryInfC{$\Gamma \vdash \exists \chi:s.E : o$}
  \end{prooftree}
  \begin{prooftree}
    \AxiomC{$\Gamma \vdash E : o$}
    \AxiomC{$\Gamma \vdash F : o$}
    \BinaryInfC{$\Gamma \vdash E \Rightarrow F : o$}
    \DisplayProof\hfill
    \AxiomC{$\Gamma \vdash E : o$}
    \AxiomC{$\Gamma \vdash F_1 : \tau$}
    \AxiomC{$\Gamma \vdash F_2 : \tau$}
    \TrinaryInfC{$\Gamma \vdash E \restriction F_1 \equiv F_2 : o$}
  \end{prooftree}
  \begin{prooftree}
    \AxiomC{$(\Gamma \vdash A_i : o)_{i \in I}$}
    \UnaryInfC{$\Gamma \vdash \{(l_i : A_i)_{i \in I}; \dots\} : o$}
    \DisplayProof\hfill
    \AxiomC{$(\Gamma \vdash A_i : o)_{i \in I}$}
    \UnaryInfC{$\Gamma \vdash \{(l_i : A_i)_{i \in I}\} : o$}
    \DisplayProof\hfill
    \AxiomC{$\Gamma \vdash E : \tau$}
    \AxiomC{$\Gamma \vdash F : o$}
    \BinaryInfC{$\Gamma \vdash E \in F : o$}
  \end{prooftree}
  \begin{prooftree}
    \AxiomC{$(\Gamma \vdash A_i : o)_{i \in I}$}
    \UnaryInfC{$\Gamma \vdash [(C_i : A_i)_{i \in I}] : o$}
    \DisplayProof\hfill
    \AxiomC{$\Gamma \vdash E : \kappa$}
    \AxiomC{$\Gamma, \chi : o \vdash F : o$}
    \BinaryInfC{$\Gamma \vdash \mu_E \chi.F : o$}
    \DisplayProof\hfill
    \AxiomC{$\Gamma \vdash E : \kappa$}
    \AxiomC{$\Gamma, \chi : o \vdash F : o$}
    \BinaryInfC{$\Gamma \vdash \nu_E \chi.F : o$}
  \end{prooftree}
  \caption{Sorting rules of the language.}
  \label{sorting}
\end{figure}
As the type system of \pml is higher-order, we first need to define the
syntax of expressions. It will gather the types, but also the terms and
several sort of first-order objects. This will include pre-values, pre-terms,
pre-stacks, syntactic ordinals and propositions.
\begin{definition}[sorts]
  Our language is built with expressions of several base sorts: $\iota$ for
  values, $\tau$ for terms, $\sigma$ for stacks, $\kappa$ for ordinals, and
  $o$ for propositions (or types). There is also a function sort to
  build parametric expressions. The set of all sorts $\mathcal{S}$ is
  generated by the following BNF grammar.
  \begin{align*}
    s,r \bnfeq &\iota \bnfor \tau \bnfor \sigma \bnfor \kappa
        \bnfor o \bnfor s \to t
  \end{align*}
\end{definition}
\begin{definition}[semantic domain]
  For every sort $s$ we define a set $\sem{s}$ called its
  semantic domain. It is defined inductively as follows.
  $$
    \sem{\iota } = \Lambda_\iota
    \quad\quad
    \sem{\tau  } = \Lambda_\tau
    \quad\quad
    \sem{\sigma} = \Pi
    \quad\quad
    \sem{\kappa} = \Omega + \omega
    \quad\quad
    \sem{s \to r} =
      \sem{r}^{\sem{s}}
  $$
  $$
    \sem{o     } =
      \{P \in \mathcal{P}(\Lambda_\iota / \equiv) \;|\; \square \in P\}
  $$
  Note that $\Omega$ denotes an ordinal that is large enough for semantics
  of our inductive and coinductive type to reach a stationary point. It must
  exist by a cardinality argument (see previous work \cite{subml}).
\end{definition}
\begin{definition}[higher-order expressions]
  The syntax of higher-order expressions is defined in Figure~\ref{expr},
  and valid expressions are those that can be assigned a sort using the
  rules of Figure~\ref{sorting}. An expression of sort $o$ will be called
  a type.
\end{definition}

\begin{definition}[semantics of closed parametric expressions]
  Let $E$ be a higher-order expression such that $\Gamma \vdash E : s$ is
  derivable using the rules of Figure~\ref{sorting}. Given a semantical
  map $\rho$ associating every $\chi \in FV(E)$ to an element of
  $\sem{\Gamma(\chi)}$, the expression $E\rho$ is called
  a closed parametric expression. Every such closed parametric expression is
  given a semantic interpretation $\sem{E\rho}$ according to
  Figures~\ref{semany}, \ref{semiota}, \ref{semtau}, \ref{semsigma},
  \ref{semkappa} or \ref{semomicron} depending on the sort $s$.
\end{definition}

\begin{lemma}[semantics of sorted expressions]
  If $\Gamma \vdash E : s$ is derivable using the rules of
  Figure~\ref{sorting} and if $\rho$ is a semantical map corresponding to
  $\Gamma$, then $\sem{E\rho} \in \sem{s}$.
\end{lemma}
\begin{proof}
  See the author's PhD thesis \cite[Chapter~4]{lepigrePhD}.
\end{proof}

\begin{figure}
  \begin{align*}
    \sem{\Phi} &= \Phi\\
    \sem{\chi} &= \chi\\
    \sem{(\chi:s \mapsto E)}
      &= (\chi:s \mapsto \sem{E})\\
    \sem{E\langle F \rangle}
      &= \sem{E} \langle \sem{F} \rangle\\
    \sem{\varepsilon_{\chi:s}(t \in A)}
      &= \begin{cases}
           \Phi \in \sem{s} \text{ such that}
             \sem{t} \in \sem{A[\chi := \Phi]
            }^{\bot\bot} \text{ if any}\\
           \text{ an arbitrary element of} \sem{s}
             \text { otherwise}
         \end{cases}\\
    \sem{\varepsilon_{\chi:s}(t \notin A)}
      &= \begin{cases}
           \Phi \in \sem{s} \text{ such that}
             \sem{t} \notin \sem{A[\chi := \Phi]
            }^{\bot\bot} \text{ if any}\\
           \text{ an arbitrary element of} \sem{s}
             \text { otherwise}
         \end{cases}
  \end{align*}
  \caption{Semantics for elements of arbitrary sort.}\label{semany}
\end{figure}

\begin{figure}
  \begin{align*}
    \sem{\lambda x.t}
      &= \lambda x.\sem{t}\\
    \sem{C_k[v]}
      &= C_k[\sem{v}]\\
    \sem{\{(l_i=v_i)_{i \in I}\}}
      &= \{(l_i = \sem{v_i})_{i \in I}\}\\
    \sem{\varepsilon_{x \in A}(t \notin B)}
      &= \begin{cases}
           v \in \sem{A} \setminus \{\square\}
             \text{ such that} \sem{t[x := v]} \notin
             \sem{B}^{\bot\bot} \text{ if any}\\
           \square \text { otherwise}
         \end{cases}
  \end{align*}
  \caption{Semantics for values (of sort $\iota$).}
  \label{semiota}
\end{figure}

\begin{figure}
  \begin{align*}
    \sem{t\;u}
      &= \sem{t}\; \sem{u}\\
    \sem{\mu\alpha.t}
      &= \mu \alpha.\sem{t}\\
    \sem{[\pi]t}
      &= [\sem{\pi}]\sem{t}\\
    \sem{v.l_k}
      &= \sem{v}.l_k\\
    \sem{[v|(C_i[x_i] \to t_i)_{i \in I}]}
      &= [\sem{v} | (C_i[x_i]
         \to \sem{t_i})_{i \in I}]\\
    \sem{\varphi a.v}
      &= \varphi a.\sem{v}
  \end{align*}
  \caption{Semantics for terms (of sort $\tau$).}
  \label{semtau}
\end{figure}

\begin{figure}
  \begin{align*}
    \sem{\varepsilon} &= \varepsilon\\
    \sem{[-\;v]\pi}
      &= [-\;\sem{v}]\sem{\pi}\\
    \sem{[t\;-]\pi}
      &= [\sem{t}\;-]\sem{\pi}\\
    \sem{\varepsilon_{\alpha \in \lnot A}(t \notin A)}
      &= \begin{cases}
           \pi \in \sem{A}^\bot \text{ such that}
             \sem{t[\alpha := \pi]} \notin
             \sem{B}^{\bot\bot} \text{ if any}\\
           [\square\;-]\varepsilon \text { otherwise}
         \end{cases}
  \end{align*}
  \caption{Semantics for stacks (of sort $\sigma$).}
  \label{semsigma}
\end{figure}

\begin{figure}
  \begin{align*}
    \sem{\infty} &= \Omega\\
    \sem{\tau+1} &= \sem{\tau} + 1\\
    \sem{\varepsilon_{\theta<\tau}(t \in A)}
      &= \begin{cases}
            o < \sem{\tau} \text{ such that}
              \sem{t} \in \sem{A[\theta := o]
             }^{\bot\bot}\\
            0 \text{ otherwise}
         \end{cases}\\
    \sem{\varepsilon_{\theta<\tau}(t \notin A)}
      &= \begin{cases}
            o < \sem{\tau} \text{ such that}
              \sem{t} \notin \sem{A[\theta := o]
             }^{\bot\bot}\\
            0 \text{ otherwise}
         \end{cases}
  \end{align*}
  \caption{Semantics for syntactic ordinals (of sort $\kappa$).}
  \label{semkappa}
\end{figure}

\begin{figure}
  \begin{align*}
    \sem{A \Rightarrow B} &=
      \{\lambda x.t \;|\; \forall v \in \sem{A} \setminus
      \{\square\},
      t[x := v] \in \sem{B}^{\bot\bot}\} \cup \{\square\}\\
    \sem{[(C_i : A_i)_{i \in I}]}
      &= \bigcup_{i \in I} \{C_i[v] \;|\; v \in \sem{A_i}
           \setminus \{\square\}\} \cup \{\square\}\\
    \sem{\{(l_i : A_i)_{i \in I}; \dots\}}
      &= \{\{(l_i = v_i)_{i \in I \cup K}\} \;|\; \forall i \in I, v_i \in \sem{A_i}
           \setminus \{\square\}\} \cup \{\square\}\\
    \sem{\{(l_i : A_i)_{i \in I}\}}
      &= \{\{(l_i = v_i)_{i \in I}\} \;|\; \forall i \in I, v_i \in
    \sem{A_i}
           \setminus \{\square\}\} \cup \{\square\}\\
    \sem{\forall \chi:s.A}
      &= \bigcap_{\Phi \in \sem{s}}
           \sem{A[\chi := \Phi]}\\
    \sem{\exists \chi:s.A}
      &= \bigcup_{\Phi \in \sem{s}}
           \sem{A[\chi := \Phi]}\\
    \sem{\mu_\tau X.A}
      &= \bigcup_{o < \sem{\tau}}
            \sem{A[X := \mu_o X.A]} \cup \{\square\}\\
    \sem{\nu_\tau X.A}
      &= \bigcap_{o < \sem{\tau}}
            \sem{A[X := \nu_o X.A]}\\
    \sem{t \in A}
      &= \{v \in \sem{A} \;|\; \sem{t}
      \equiv v\} \cup \{\square\}\\
    \sem{A \restriction t \equiv u}
      &= \begin{cases}
           \sem{A} \text{ if} \sem{t}
             \equiv \sem{u}\\
           \{\square\} \text{ otherwise}
         \end{cases}
  \end{align*}
  \caption{Semantics for proposition (of sort $o$).}\label{semomicron}
\end{figure}

\begin{definition}[judgements]
  The type system of \pml relies on four forms of judgements, where $\gamma$
  denotes a positivity context (a set of syntactic ordinals assumed to be
  non-zero), and $\Xi$ denotes an equational context (a set of program
  equivalences assumed to be correct).
  \begin{itemize}
    \item Typing judgments of the form $\gamma; \Xi \vdash t : A$, derived
          using the rules of Figure~\ref{typing}.
    \item Local subtyping judgements of the form $\gamma; \Xi \vdash t : A
          \subseteq B$, derived using the rules of Figure~\ref{subtyping}
          (and we use the notation $\gamma; \Xi \vdash A \subseteq B$ for
          $\gamma;\Xi\vdash\varepsilon_{x \in a}(x \notin B):A \subseteq B$).
    \item Equivalence decision judgements of the form $\Xi \vdash t \equiv u$
          (and we use the notation $\Xi \vdash v \not\equiv \square$ as a
          shorthand for $\Xi \vdash (\lambda x.\{\})\;v \equiv \{\}$). Such
          a judgement is not proved using derivation rules, but using an
          abstract decision procedure (that is necessarily incomplete).
    \item Syntactic ordinal ordering judgements of the form $\gamma \vdash
          \upsilon < \tau$ derived using similar rules as in previous work
          \cite{subml}.
  \end{itemize}
  Note that we implicitly consider that every syntactic element in our
  judgements is well-sorted, according to the rules of Figure~\ref{sorting}.
\end{definition}

\begin{figure}
  \begin{prooftree}
    \AxiomC{$\gamma; \Xi \vdash \varepsilon_{x \in A}(t \notin B) : A
      \subseteq C$}
    \AxiomC{$\Xi \vdash \varepsilon_{x \in A}(t \notin B) \not= \square$}
    \RightLabel{$\varepsilon$-axiom}
    \BinaryInfC{$\gamma; \Xi \vdash \varepsilon_{x \in A}(t \notin B) : C$}
  \end{prooftree}
  \begin{prooftree}
    \AxiomC{$\gamma; \Xi \vdash \lambda x.t : A \Rightarrow B \subseteq C$}
    \AxiomC{$\gamma; \Xi, \varepsilon_{x \in A}(t \notin B) \not=
      \square \vdash t[x := \varepsilon_{x \in A}(t \notin B)] : B$}
    \RightLabel{$\Rightarrow$-intro}
    \BinaryInfC{$\gamma; \Xi \vdash \lambda x.t : C$}
  \end{prooftree}
  \begin{prooftree}
    \AxiomC{$\gamma; \Xi \vdash t : A \Rightarrow B$}
    \AxiomC{$\gamma; \Xi \vdash u : A$}
    \RightLabel{$\Rightarrow$-elim}
    \BinaryInfC{$\gamma; \Xi \vdash t\;u : B$}
  \end{prooftree}
  \begin{prooftree}
    \AxiomC{$\gamma; \Xi \vdash t : (u \in A) \Rightarrow B$}
    \AxiomC{$\gamma; \Xi \vdash u : A$}
    \AxiomC{$\Xi \vdash u \equiv v$}
    \RightLabel{$\Rightarrow$-strong-elim}
    \TrinaryInfC{$\gamma; \Xi \vdash t\;u : B$}
  \end{prooftree}
  \begin{prooftree}
    \AxiomC{$\gamma; \Xi \vdash t[\alpha := \varepsilon_{\alpha \in
      \lnot A}(t \notin A)] : A$}
    \RightLabel{save}
    \UnaryInfC{$\gamma; \Xi \vdash \mu \alpha.t : A$}
    \DisplayProof\hfill
    \AxiomC{$\gamma; \Xi \vdash u : A$}
    \RightLabel{restore}
    \UnaryInfC{$\gamma; \Xi \vdash [\varepsilon_{\alpha\in\lnot A}(t \notin
      A)]u : B$}
  \end{prooftree}
  \begin{prooftree}
    \AxiomC{$\gamma; \Xi \vdash v[a := \varphi a.v] : A$}
    \RightLabel{unfold}
    \UnaryInfC{$\gamma; \Xi \vdash \varphi a.v : A$}
    \DisplayProof\hfill
    \AxiomC{$\gamma; \Xi \vdash v : \{l_k : A; \dots\}$}
    \RightLabel{$\times$-elim}
    \UnaryInfC{$\gamma; \Xi \vdash v.l_k : A$}
  \end{prooftree}
  \begin{prooftree}
    \AxiomC{$\gamma; \Xi \vdash \{(l_i = v_i)_{i \in I}\} :
      \{(l_i : A_i)_{i \in I}\} \subseteq C$}
    \AxiomC{$(\gamma; \Xi \vdash v_i : A_i)_{i \in I}$}
    \RightLabel{$\times$-intro}
    \BinaryInfC{$\gamma; \Xi \vdash \{(l_i = v_i)_{i \in I}\} : C$}
  \end{prooftree}
  \begin{prooftree}
    \AxiomC{$\gamma; \Xi \vdash v : A$}
    \AxiomC{$\gamma; \Xi \vdash C_k[v] : [C_k : A] \subseteq B$}
    \RightLabel{$+$-intro}
    \BinaryInfC{$\gamma; \Xi \vdash C_k[v] : B$}
  \end{prooftree}
  \begin{prooftree}
    \AxiomC{$\gamma; \Xi \vdash v : [(C_i : A_i)_{i \in I}]$}
    \AxiomC{$(\gamma; \Xi, w_i \not= \square
      \vdash t_i[x_i := w_i] : C)_{i \in I}$}
    \RightLabel{$+$-elim}
    \BinaryInfC{$\gamma; \Xi \vdash [v|(C_i[x_i] \to t_i)_{i \in I}] : C$}
  \end{prooftree}
  \begin{center}
    \vspace{-2mm}
    where $w_i = \varepsilon_{x_i \in A_i \restriction {C_i[x_i] \,\equiv\,
    v}}(t_i \notin C)$ for all $i \in I$
  \end{center}
  \begin{prooftree}
    \AxiomC{$\gamma; \Xi \vdash u : A$}
    \AxiomC{$\Xi \vdash t \equiv u$}
    \RightLabel{replace}
    \BinaryInfC{$\gamma; \Xi \vdash t : A$}
  \end{prooftree}
  \caption{Typing rules for terms (including values).}\label{typing}
\end{figure}

\begin{figure}
  \begin{prooftree}
    \AxiomC{$(\Xi \vdash \rho_1(\chi) \equiv \rho_2(\chi))_{\chi \in FV(A)}$}
    \RightLabel{$\subseteq$-refl}
    \UnaryInfC{$\gamma; \Xi \vdash t : A\rho_1 \subseteq A\rho_2$}
    \DisplayProof\hfill
    \AxiomC{$\gamma; \Xi \vdash A \subseteq B$}
    \RightLabel{gen}
    \UnaryInfC{$\gamma; \Xi \vdash t : A \subseteq B$}
  \end{prooftree}
  \begin{prooftree}
    \AxiomC{$\gamma; \Xi, w \not= \square \vdash w : A_2 \subseteq A_1$}
    \AxiomC{$\gamma; \Xi, w \not= \square \vdash t\;w : B_1 \subseteq B_2$}
    \AxiomC{$\Xi \vdash t \equiv v$}
    \RightLabel{$\Rightarrow$-sub}
    \TrinaryInfC{$\gamma; \Xi \vdash t : A_1 \Rightarrow B_1
      \subseteq A_2 \Rightarrow B_2$}
  \end{prooftree}
  \begin{center}
    \vspace{-2mm}
    where $w = \varepsilon_{x \in A_2}(t\;x \notin B_2)$
  \end{center}
  \begin{prooftree}
    \AxiomC{$I_1 \subseteq I_2$}
    \AxiomC{$(\gamma; \Xi \vdash (\lambda x.[x | C_i[x_i] \to x_i])\;t : A_i
      \subseteq B_i)_{i \in I_1}$}
    \AxiomC{$\Xi \vdash t \equiv v$}
    \RightLabel{$+$-sub}
    \TrinaryInfC{$\gamma; \Xi \vdash t : [(C_i : A_i)_{i \in I_1}]
      \subseteq [(C_i : B_i)_{i \in I_2}]$}
  \end{prooftree}
  \begin{prooftree}
    \AxiomC{$(\gamma; \Xi \vdash (\lambda x.x.l_i)\;t : A_i
      \subseteq B_i)_{i \in I}$}
    \AxiomC{$\Xi \vdash t \equiv v$}
    \RightLabel{$\times$-sub-strict}
    \BinaryInfC{$\gamma; \Xi \vdash t : \{(l_i : A_i)_{i \in I}\}
      \subseteq \{(l_i : B_i)_{i \in I}\}$}
  \end{prooftree}
  \begin{prooftree}
    \AxiomC{$I_2 \subseteq I_1$}
    \AxiomC{$(\gamma; \Xi \vdash (\lambda x.x.l_i)\;t:A_i
      \subseteq B_i)_{i \in I_2}$}
    \AxiomC{$\Xi \vdash t \equiv v$}
    \RightLabel{$\times$-sub-ext / $\times$-sub-mixed}
    \TrinaryInfC{$\gamma; \Xi \vdash t : \{(l_i : A_i)_{i \in I_1}; \dots\}
      \subseteq \{(l_i : B_i)_{i \in I_2} [; \dots]\}$}
  \end{prooftree}
  \begin{prooftree}
    \AxiomC{$\gamma; \Xi \vdash t : A[\chi := \psi] \subseteq B$}
    \RightLabel{$\forall$-left}
    \UnaryInfC{$\gamma; \Xi \vdash t : \forall \chi^s.A \subseteq B$}
    \DisplayProof\hfill
    \AxiomC{$\gamma; \Xi \vdash t : A \subseteq B[\chi := \psi]$}
    \RightLabel{$\exists$-right}
    \UnaryInfC{$\gamma; \Xi \vdash t : A \subseteq \exists \chi^s.B$}
  \end{prooftree}
  \begin{prooftree}
    \AxiomC{$\gamma; \Xi \vdash t : A \subseteq
      B[\chi := \varepsilon_{\chi:s}(t \notin B)]$}
    \AxiomC{$\Xi \vdash t \equiv v$}
    \RightLabel{$\forall$-right}
    \BinaryInfC{$\gamma; \Xi \vdash t : A \subseteq \forall \chi^s.B$}
  \end{prooftree}
  \begin{prooftree}
    \AxiomC{$\gamma; \Xi \vdash
      t : A[\chi := \varepsilon_{\chi:s}(t \notin A)] \subseteq B$}
    \AxiomC{$\Xi \vdash t \equiv v$}
    \RightLabel{$\exists$-left}
    \BinaryInfC{$\gamma; \Xi \vdash t : \exists \chi^s.A \subseteq B$}
  \end{prooftree}
  \begin{prooftree}
    \AxiomC{$\gamma; \Xi, u_1 \equiv u_2 \vdash t : A \subseteq B$}
    \AxiomC{$\Xi \vdash t \equiv v$}
    \RightLabel{$\restriction$-left}
    \BinaryInfC{$\gamma; \Xi \vdash
      t : A \restriction u_1 \equiv u_2 \subseteq B$}
    \DisplayProof\hfill
    \AxiomC{$\gamma; \Xi \vdash t : A \subseteq B$}
    \AxiomC{$\Xi \vdash u_1 \equiv u_2$}
    \RightLabel{$\restriction$-right}
    \BinaryInfC{$\gamma; \Xi \vdash t : A \subseteq
      B \restriction u_1 \equiv u_2$}
  \end{prooftree}
  \begin{prooftree}
    \AxiomC{$\gamma; \Xi, t \equiv u \vdash t : A \subseteq B$}
    \AxiomC{$\Xi \vdash t \equiv v$}
    \RightLabel{$\in$-left}
    \BinaryInfC{$\gamma; \Xi \vdash t : u \in A \subseteq B$}
    \DisplayProof\hfill
    \AxiomC{$\gamma; \Xi \vdash t : A \subseteq B$}
    \AxiomC{$\Xi \vdash t \equiv u$}
    \RightLabel{$\in$-right}
    \BinaryInfC{$\gamma; \Xi \vdash t : A \subseteq u \in B$}
  \end{prooftree}
  \begin{prooftree}
    \AxiomC{$\gamma; \Xi \vdash t : A \subseteq B[X := \mu_\infty X.B]$}
    \RightLabel{$\mu_\infty$-left}
    \UnaryInfC{$\gamma; \Xi \vdash t : A \subseteq \mu_\infty X.B$}
    \DisplayProof\hfill
    \AxiomC{$\gamma; \Xi \vdash t : A[X := \nu_\infty X.A] \subseteq B$}
    \RightLabel{$\nu_\infty$-right}
    \UnaryInfC{$\gamma; \Xi \vdash t : \nu_\infty X.A \subseteq B$}
  \end{prooftree}
  \begin{prooftree}
    \AxiomC{$\gamma; \Xi \vdash t : A \subseteq B[X := \mu_\upsilon X.B]$}
    \AxiomC{$\gamma \vdash \upsilon < \tau$}
    \RightLabel{$\mu$-right}
    \BinaryInfC{$\gamma; \Xi \vdash t : A \subseteq \mu_\tau X.B$}
  \end{prooftree}
  \begin{prooftree}
    \AxiomC{$\gamma; \Xi \vdash t : A[X := \nu_\upsilon X.A] \subseteq B$}
    \AxiomC{$\gamma \vdash \upsilon < \tau$}
    \RightLabel{$\nu$-left}
    \BinaryInfC{$\gamma; \Xi \vdash t : \nu_\tau X.A \subseteq B$}
  \end{prooftree}
  \begin{prooftree}
    \AxiomC{$\gamma, \tau; \Xi \vdash t : A[X := \mu_{\varepsilon_{\theta <
      \tau}(t \in A[X := \mu_\theta X.A])}X.A] \subseteq B$}
    \AxiomC{$\Xi \vdash t \equiv v$}
    \RightLabel{$\mu$-left}
    \BinaryInfC{$\gamma; \Xi \vdash t : \mu_\tau X.A \subseteq B$}
  \end{prooftree}
  \begin{prooftree}
    \AxiomC{$\gamma, \tau; \Xi \vdash t : A \subseteq B[X :=
      \nu_{\varepsilon_{\theta < \tau}(t \notin B[X := \nu_\theta X.B])}X.B]$}
    \AxiomC{$\Xi \vdash t \equiv v$}
    \RightLabel{$\nu$-right}
    \BinaryInfC{$\gamma; \Xi \vdash t : A \subseteq \nu_\tau X.B$}
  \end{prooftree}
  \caption{Local subtyping rules.}\label{subtyping}
\end{figure}

\begin{theorem}[adequacy]
  We have the following adequacy properties.
  \begin{itemize}
    \item If $\gamma; \Xi \vdash t : A$ is derivable and if the contexts
          $\gamma$ and $\Xi$ are realized, then $\sem{t}
          \in \sem{A}^{\bot\bot}$. Moreover, if $t$ is a
          value then $\sem{t} \not= \square$.
    \item If $\gamma; \Xi \vdash t : A \subseteq B$ is derivable, the contexts
          $\gamma$ and $\Xi$ are realized, and $\sem{t}
          \in \sem{A}^{\bot\bot}$, then $\sem{t
         } \in \sem{B}^{\bot\bot}$, In particular,
          if the judgment is of the form $\gamma; \Xi \vdash A \subseteq B$
          then $\sem{A} \subseteq \sem{B}$.
  \end{itemize}
\end{theorem}
\begin{proof}
  See the author's PhD thesis \cite[Chapter~4]{lepigrePhD}. An essential
  point of the semantics is that the bi-orthogonal completion operator
  $\Psi \mapsto \Psi^{\bot\bot}$ on the elements of $\sem{o}$
  is closed for values (i.e., any value in $\Psi^{\bot\bot}$ already appears
  in $\Psi$). The complex definitions of reduction and observational
  equivalence allow us to construct a model with this property (see previous
  work for more details \cite{lepigre2016,lepigrePhD}).
\end{proof}
\begin{corollary}[proof of equivalence]
  If $\vdash t : \{\} \restriction u_1 \equiv u_2$ is derivable for
  user-accessible terms $t$, $u_1$ and $u_2$, then the equivalence
  $u_1 \equiv u_2$ holds.
\end{corollary}
