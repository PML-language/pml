\documentclass[oneside,a4]{book}

\usepackage{amssymb,amsmath,amsthm}
\usepackage[mathletters]{ucs}
\usepackage[utf8x]{inputenc}
\usepackage{bussproofs}
\usepackage{enumitem}
\usepackage{caption}
\usepackage{xspace}
\usepackage{subcaption}
\usepackage{stmaryrd}
\usepackage{hyperref}
\usepackage{geometry}
\usepackage{mdframed}
\usepackage{thmbox}
\usepackage{longtable}

% Boxed figures
\usepackage{float}
\floatstyle{boxed}
\restylefloat{figure}

\newcommand{\bnfeq}{\,\mathrel{::=}\;\,}
\newcommand{\bnfor}{\;\mathrel{\vert}\;}
\newcommand{\bnfst}{\\&\vert\;}
\renewcommand{\emptyset}{\varnothing}
\newcommand{\st}{\mid}
\newcommand{\pml}{$\text{PML}_2$\xspace}
\renewcommand\labelitemi{---}
\setlist[itemize]{noitemsep,nolistsep}

\newcommand*{\minw}[2]{%
  \makebox[#2]{#1}%
}

\newcommand{\vars}{\mathcal{V}}
\newcommand{\lvars}{\mathcal{V}_{ι}}
\newcommand{\tvars}{\mathcal{V}_{τ}}
\newcommand{\kvars}{\mathcal{V}_{κ}}
\newcommand{\svars}{\mathcal{V}_{σ}}
\newcommand{\ovars}{\mathcal{V}_{ο}}
\newcommand{\cnsts}{\mathcal{C}}
\newcommand{\props}{\mathcal{P}}
\newcommand{\sorts}{\mathcal{S}}
\newcommand{\terms}{\Lambda}
\newcommand{\types}{\mathcal{F}}
\newcommand{\N}{\mathbb{N}}
\newcommand{\arf}[1]{\lvert #1 \rvert_1}
\newcommand{\ars}[1]{\lvert #1 \rvert_2}
\newcommand{\sem}[1]{\ensuremath{\llbracket #1 \rrbracket}}
\newcommand{\Mid}{$\mid$}

\newtheorem{definition}{Definition}
\newtheorem{example}{Example}
\newtheorem{lemma}{Lemma}
\newtheorem{theorem}{Theorem}
\newtheorem{corollary}{Corollary}
\newtheorem{remark}{Remark}

\begin{document}

\title{Syntax and Semantics of the \pml Language}
\author{Rodolphe Lepigre \& Christophe Raffalli}
\date{}

\maketitle

This document is intended as a reference for the current implementation of
the \pml language and its theory, but it is still work in progress. It is
based on the first author's PhD thesis~\cite{Lepigre2017PhD}, as well as a
joint work of the two authors~\cite{LepRaf2018a}. The present document is
nonetheless intended to be as self-contained as possible.

\tableofcontents

\mainmatter

\part{The language}

\chapter{PML Basics}

\section{functions and algebraic data types}

\subsection{A simple enumerated type}\label{basics-day}

We start with a very simple example defining a data type for the days of the
week. This example is inspired by the software foundation course of Benjamin
  Pierce et all\cite{Pierce:SF1}:

\begin{pmlcode}
  type day = [ Monday ; Tuesday ; Wednesday
    ; Thursday ; Friday ; Saturday ; Sunday ]
\end{pmlcode}

This defines a data type \dupml{day} with seven constant elements
representing each day of the week. This kind of type is
called \emph{enumerated type}, \emph{variant type} or \emph{sum type}.
The seven elements are distinct constants and must start with an upper case
letter to distinguish them from variables.

We define functions using this type:

\begin{pmlcode}
  val next_day : day ⇒ day =
    fun d {
      case d {
        Monday    → Tuesday
        Tuesday   → Wednesday
        Wednesday → Thursday
        Thursday  → Friday
        Friday    → Saturday
        Saturday  → Sunday
        Sunday    → Monday
      }
    }

  val previous_day : day ⇒ day =
    fun d {
      case d {
        Monday    → Sunday
        Tuesday   → Monday
        Wednesday → Tuesday
        Thursday  → Wednesday
        Friday    → Thursday
        Saturday  → Friday
        Sunday    → Saturday
      }
    }

  val second_next_day : day ⇒ day =
    fun d { next_day (next_day d) }
\end{pmlcode}

The \dupml{val} keyword indicates that we define a PML value and is followed
by a name for this value and its type. Here \dupml{day ⇒ day} means a total
pure function that takes a \dupml{day} as argument and returns an element of
the same type.

The \dupml{fun} keyword denotes a function definition, followed by
a name for the argument of the function and the body of the function.
Finally, \dupml{case} denotes a case analysis on the value of the variable
\dupml{d}.

We can now prove two simple properties of these two functions:

\begin{pmlcode}
  val trivial : second_next_day Monday ≡ Wednesday = qed
\end{pmlcode}

For properties, we define values as for programs using the \dupml{val} keyword!
PML follows the Curry-Howard correspondence and identify properties and types.
The type used for properties are richer, here it is an equality which is a
type inhabited if and only if the equality holds (we will see later what is
the element of this type). The body of the definition is just
\dupml{qed} meaning that the proof results only from a computation.
This first property is trivial for PML.

\begin{pmlcode}
  val next_previous_day : ∀d∈day, next_day (previous_day d) ≡ d =
    take d;
    case d {
      Monday    → qed
      Tuesday   → qed
      Wednesday → qed
      Thursday  → qed
      Friday    → qed
      Saturday  → qed
      Sunday    → qed
    }
\end{pmlcode}

Here, the type is more complex. It expresses that for any element \dupml{d}
of type \dupml{day}, applying the function \dupml{previous_day} to \dupml{d} and then applying
\dupml{next_day} to the result gives the initial value back.

The proof starts with the \dupml{take} keyword to introduce a fresh variable
\dupml{d} for an arbitrary day. In fact, we could use the \dupml{fun} keyword, both
are equivalent. Then, we perform a case analysis on \dupml{d} and in each case,
the proof results of a computation hence we can use \dupml{qed}. Indeed, if
\dupml{d ≡ Monday}, by definition \dupml{previous_day d ≡ Sunday}
and \dupml{next_day Sunday ≡ Monday ≡ d}. The same happens for all seven cases.

\begin{pmlcode}
  val previous_next_day : ∀d∈day, previous_day (next_day d) ≡ d =
    take d;
    set auto 1 0;
    qed
\end{pmlcode}

For the converse property, we give a shorter proof using the hability of PML
to perform some automatic theorem proving. In fact, it is very limited!
The directive \dupml{set auto 1 0} tells PML to perform at most one case
analysis (the 1) and no totality proof (the 0, we will see later what this
means). In fact, the is no magic: PML sees that the computation of
\dupml{next_day d} is blocked on a case analysis and try to proceed in the
proof using the same case analysis as the one in the program.

\begin{exercise}\label{exo-day-1}
  Prove that applying seven times the function \dupml{next_day} returns the initial
  day.
\end{exercise}

\subsection{Booleans}

The type of booleans is defined by

\begin{pmlcode}
  //type bool = [ true; false ]
\end{pmlcode}

Normally enumerated type should start with a capital letter. PML allows for an
exception with \dupml{true} and \dupml{false} as OCaml do. In fact booleans
are predefined and can not be redefined as they play a special role in a
few places in PML. This is why we give the definition as a comment.

Most definition in this section are in the standard library but we here redefine them
as they provide good examples.

We now define some standard function on booleans:

\begin{pmlcode}
  val not : bool ⇒ bool = fun b {
    case b { true → false | false → true }
  }

  val and : bool ⇒ bool ⇒ bool = fun b1 b2 {
    case b1 { true → b2 | false → false }
  }

  val or : bool ⇒ bool ⇒ bool = fun b1 b2 {
    if b1 { true } else { b2 }
  }
\end{pmlcode}

These examples are not different from the previous ones. The only novelty
are the function with two arguments. The type \dupml{bool ⇒ bool ⇒ bool}
should be read as \dupml{bool ⇒ (bool ⇒ bool)} (the function type is right
associative). This means that a function with two arguments is in fact
a function taking one argument returning a function.
The notation \dupml{fun b1 b2 { ⋯ }} is equivalent to \dupml{fun b1 { fun b2
    { ⋯ }}} making this more explicit.

The last definition show the \dupml{if ⋯ else ⋯} notation in PML which
is just a syntactic sugar for the case notation for booleans.

However, there is a problem with the definition of \dupml{and} and
\dupml{or}. Indeed, one usually expects that when evaluating a conjunction
(resp. a disjunction) of
two expressions, the second expression is only evaluated if the first
expression evaluates to \dupml{true} (resp. \dupml{false}).
As PML is in call by value, all arguments of a function are evaluated
before calling the function.

It is possible to fix this using the \dupml{def} keyword as follows

\begin{pmlcode}
  def land⟨a:τ,b:τ⟩ = if a { b } else { false }
  def lor ⟨a:τ,b:τ⟩ = if a { true } else { b }
\end{pmlcode}

In PML we define \emph{expressions} of various \emph{sorts}. For now we have seen
three sorts of expressions:
\begin{itemize}
\item propositions or types like \dupml{day}, \dupml{bool ⇒ bool}
  or \dupml{∀d∈day, previous_day (next_day d) ≡ d}. These are expressions of
  sort \dupml{ο} (omicron).
\item Values like \dupml{Monday}, \dupml{Tuesday} or \dupml{fun b { not b }}
  which are evaluated programs. These are expressiosn of sort \dupml{ι}
  (iota).
\item Terms or programs like \dupml{not b} or \dupml{if b { true } else {
    false }}. These are expressions of sort \dupml{τ} (tau). Moreover any
  value is also a term. We say that the sort \dupml{ι} is a \emph{subsort} of
  \dupml{τ}.
\end{itemize}

PML also support parametric expressions. Above,
\dupml{land} and \dupml{lor} are two expressions of sort \dupml{τ → τ →
  τ}. They can be seen as macros with two parameters. An important difference
between values defined with the \dupml{val} keyword and macros is that
\begin{itemize}
  \item Values are defined by giving a term and a type and PML performs
    type-checking to ensure that the term indeed belong to the given type.
    This ensures that the evaluation can not fail and will always returns
    a value of the given type.
  \item At the contrary, general expressions are not type-checked at their
    definition. PML checks that they have the intended sort, that is that
    \dupml{land} and \dupml{lor} returns terms when applied to two terms.
    But it does not check that these terms evaluates to boolean if given
    booleans. Macros are expended and type-checked only when used.
\end{itemize}

This means that \dupml{land⟨e1,e2⟩} is exactly the same
as \dupml{if e1 { e2 } else { false }}. This is what we want to
ensure that \dupml{e2} is only evaluated when \dupml{e1} evaluates to
\dupml{true}.

Then, we can give an infix syntax for conjunction and disjonction.
\begin{pmlcode}
  infix (&&) = land⟨⟩ priority 6 right associative
  infix (||) = lor ⟨⟩ priority 7 right associative
\end{pmlcode}

To do so we give the infix symbol between parenthesis (mostly all special
characteres are allowed except delimiters). We give the name of the associated macro (we
could also give an infix notation for a value if we remove \dupml{⟨⟩}).
The priority is a floating point number (the lower the number the
higher the priority) and finally the associativity of the symbol
(\dupml{left}, \dupml{right} or \dupml{none}).

We can now use our macros and their infix notation to define some functions:
\begin{pmlcode}
  val and : bool ⇒ bool ⇒ bool = fun b1 b2 { b1 && b2 }
  val or  : bool ⇒ bool ⇒ bool = fun b1 b2 { b1 || b2 }
  val xor : bool ⇒ bool ⇒ bool =
    fun b1 b2 { (b1 && not b2) || (b2 && not b1) }
\end{pmlcode}

We can prove a few properties:
\begin{pmlcode}
  val not_idempotent : ∀b∈bool, not (not b) ≡ b =
    take b;
    if b { qed } else { qed }

  val demorgan_and : ∀b1 b2∈bool, not (b1 && b2) ≡ not b1 || not b2 =
    take b1 b2;
    set auto 1 0;
    qed

  val demorgan_or : ∀b1 b2∈bool, not (b1 || b2) ≡ not b1 && not b2 =
    take b1 b2;
    set auto 1 0;
    qed
\end{pmlcode}

Nothing is new in these proofs as we are still manipulating finite data
types.

\begin{exercise}\label{exo-bool-1}
  Define implication with the infix notation \dupml{=>} of lower priority than
  \dupml{&&} and \dupml{||} and prove \dupml{b1 && b2 => b3 ≡ b1 => b2 => b3}.
\end{exercise}

\begin{exercise}\label{exo-bool-2}
  Prove that \dupml{xor} and \dupml{&&} satisfy the axiom of a commutative
  semi ring (the are both associative and commutative, \dupml{xor} distributes
  on \dupml{and}, \dupml{false} and \dupml{true} are neutral for \dupml{xor}
  and \dupml{and} respectively and an element is its own opposite for \dupml{xor}).
\end{exercise}

\subsection{Unary natural numbers}

A unary natural number is either zero or the successor of a natural
numbers. This is expressed by the following recursive type:

\begin{pmlcode}
  type rec nat = [ Zero; S of nat ]
\end{pmlcode}

The \dupml{rec} keyword allows to use \dupml{nat} in its own definition.
This definition should be read as ``an element of \dupml{nat} is either the
constant \dupml{Zero} or the \emph{constructor} \dupml{S} applied to an
element of type \dupml{nat}''. The application of a constructor is written
with
square bracket, hence this means that all elements of type \dupml{nat} are
\dupml{Zero}, \dupml{S[Zero]}, \dupml{S[S[Zero]]}, ⋯

Using this we define a few functions on natural numbers:

\begin{pmlcode}
  val zero : nat = Zero
  val succ : nat ⇒ nat = fun n { S[n] }
  val rec dble : nat ⇒ nat = fun n {
    case n {
      Zero → Zero
      S[p] → S[S[dble p]]
    }
  }
\end{pmlcode}

The last function is recursive (indicated by the \dupml{rec} keyword).
Its case analysis is also more complex, in the second case, when
\dupml{n} is a successor, we give a name to the predecessor of \dupml{n}.
This case analysis can be read as either \dupml{n} is equal to \dupml{Zero}
or it is equal to \dupml{S[p]} for some natural number \dupml{p}. This is the
natural case analysis resulting from the definition of the type \dupml{nat}.

Then, in the second case we say that the double of \dupml{n ≡ S[p]} is
\dupml{S[S[dble p]]} and it is here that the function \dupml{dble} calls
itself recursively.

It is important to remark that PML, contrary to most computer languages,
checks the termination of the function \dupml{dble}. For instance the
following is rejected with the error message \dupml{Cannot prove termination}.

\begin{badpmlcode}
  val rec dble : nat ⇒ nat = fun n {
    case n {
      Zero → Zero
      S[p] → S[S[dble n]]
    }
  }
\end{badpmlcode}

We defined the three functions above for a precise reason. PML allows to write
integer constants like \dupml{0}, \dupml{2}, \dupml{-2} or \dupml{42} and they are
replaced by terms using \dupml{zero}, \dupml{succ}, \dupml{dble} and \dupml{opp}.
For instance \dupml{42} is replaced by \dupml{dble (succ (dble (dble (succ (dble
  (dble (succ zero)))))))}.
This allows to use integer constants for any representation of numbers once the above
functions are provided.

We can define more interesting function and gives them an infix notation:

\begin{pmlcode}
  // Addition function.
  infix (+) = add priority 3 left associative

  val rec (+) : nat ⇒ nat ⇒ nat = fun n m {
    case n {
      0    → m
      S[k] → succ (k + m)
    }
  }

  // Multiplication function.
  infix (*) = mul priority 2 left associative

  val rec (*) : nat ⇒ nat ⇒ nat = fun n m {
    case n {
      0    → 0
      S[k] → m + k * m
    }
  }
\end{pmlcode}

We introduce the infix symbol before the definition to use it inside the
definition. We also remark that PML allows to use \dupml{0} for \dupml{Zero}
in case analysis. These are again recursive programs based on the
following equations:

\begin{eqnarray*}
  0 + m &= m \\
  {\rm succ}(n) + m &= {\rm succ}(n + m) \\
  0 × m &= 0 \\
  {\rm succ}(n) × m &= m + n × m \\
\end{eqnarray*}

Many other definitions are possible using other equations as long as they
lead to terminating functions. For instance an alternative definition of
addition could be:

\begin{pmlcode}
  val rec fast_add : nat ⇒ nat ⇒ nat = fun n m {
    case n {
      0    → m
      S[k] → succ (fast_add m k)
    }
  }
\end{pmlcode}

This definition is faster because the complexity of the first definition is
$O(n)$ while the latter is $O(\min(n,m))$. Let us prove that the two
definitions above are equivalent. Here is the beginning of the proof:

\begin{pmlcode}
  val rec add_eq_fast_add : ∀n m∈nat, n + m ≡ fast_add n m =
    take n m;
    case n {
      0    → qed
      S[k] → showing S[k + m] ≡ S[fast_add m k];
              showing k + m ≡ fast_add m k;
              {- unfinished -}
    }
\end{pmlcode}

The zero case is immediate because both \dupml{0 + m} and \dupml{fast_add 0
  m} evaluate to m. For the successor case, we use the \dupml{showing}
keyword to write the current goal, transformed as far as possible using
evaluation. The first \dupml{showing} correspond to evaluation,
The second corresponds to a simplification of the successor on both side.

Using \dupml{showing} corresponds to what is call backward reasonning, we
starts from the goal and make progress toward something which is true.
Forward reasonning is also possible as we will see later.

As for programs, proofs can be recursive, but here using
what we are proving on \dupml{k} and \dupml{m} would lead to
\dupml{k + m ≡ fast_add k m} which is not what we need. A solution is to
prove commutativity of addition which is anyway very usefull. Notice the
notion of \emph{goal} in PML written \dupml{{- ⋯ -}} which allows to
type-check unfinished programs or proofs.

For commutativity we first need two lemmas:
\begin{pmlcode}
  val rec add_n_zero : ∀n∈nat, n + 0 ≡ n =
    take n;
    case n {
      0    → qed
      S[k] → showing S[k] + 0 ≡ S[k];
              showing k + 0 ≡ k;
              use add_n_zero k
    }

  val rec add_n_sm : ∀n m∈nat, n + S[m] ≡ S[n + m] =
    take n m;
    case n {
      0    → qed
      S[k] → showing S[k] + S[m] ≡ S[S[k] + m];
              showing k + S[m] ≡ S[k + m];
              use add_n_sm k m
    }
\end{pmlcode}

In this two lemmas we immediately see that the current goal corresponds to an
induction hypothesis that we can use to finish the proof. It is important to
know that the \dupml{take}, \dupml{showing} and \dupml{use} keywords are
simple syntactic sugar that one can use to make proof readable and also to
gather some knowledge when we are struggling with a difficult proof.  For
instance, if a \dupml{showing} fails, either the current goal is not implied
by what we wrote of we must prove some lemmas to reach the wanted goal.

In fact, much shorter proofs are possible:
\begin{pmlcode}
  val rec add_n_zero : ∀n∈nat, n + 0 ≡ n = fun n {
    case n {
      0    → qed
      S[k] → add_n_zero k
    }
  }

  val rec add_n_sm : ∀n m∈nat, n + S[m] ≡ S[n + m] = fun n m {
    case n {
      0    → qed
      S[k] → add_n_sm k m
    }
  }
\end{pmlcode}

Using these lemmas we can finish the proof of commutativity and
our result on \dupml{fast_add}.

\begin{pmlcode}
  val rec add_commutative : ∀n m∈nat, n + m ≡ m + n =
    take n m;
    case n {
      0    → showing 0 + m ≡ m + 0;
              use add_n_zero m
      S[k] → showing S[k] + m ≡ m + S[k];
              showing S[k + m] ≡ S[m + k] by add_n_sm m k;
              showing k + m ≡ m + k;
              use add_commutative k m
    }

  val rec add_eq_fast_add : ∀n m∈nat, n + m ≡ fast_add n m =
    take n m;
    case n {
      0    → qed
      S[k] → showing S[k + m] ≡ S[fast_add m k];
              showing k + m ≡ fast_add m k;
              showing m + k ≡ fast_add m k by add_commutative m k;
              use add_eq_fast_add m k
    }
\end{pmlcode}

You should notice the use of \dupml{by} together with \dupml{showing} to
justify a reasoning with a proof. Again much shorter (but less readable
proofs) are possible.

\begin{pmlcode}
  val rec add_commutative : ∀n m∈nat, n + m ≡ m + n = fun n m {
    case n {
      0    → add_n_zero m
      S[k] → add_n_sm m k; add_commutative k m
    }
  }

  val rec add_eq_fast_add : ∀n m∈nat, n + m ≡ fast_add n m = fun n m {
    case n {
      0    → qed
      S[k] → add_commutative m k; add_eq_fast_add m k
    }
  }
\end{pmlcode}

\begin{exercise}\label{exo-nat-1}
  Prove the other semi ring axioms for addition and multiplication.
\end{exercise}

\begin{exercise}\label{exo-nat-2}
  Define the power function ($n^m$) and prove that $n^{m+p} = n^m × n^p$ and
  $n^{m × p} = (n^m)^p$.
\end{exercise}

\subsection{Lists}

Here is the definition of the type for lists:
\begin{pmlcode}
  type rec list⟨a⟩ = [Nil ; Cons of {hd : a ; tl : list⟨a⟩}]
\end{pmlcode}

Here we see two novelties. First, \dupml{list} is a parametric type
depending on another type. This means that \dupml{list} is
an expression of sort \dupml{ο → ο}.

Second, we are using records. A record is some kind of association list
which associates values to some labels. The labels are names.

In PML constructor of a sum type
are always applied to exactly one argument. Previously,
\dupml{Monday} and \dupml{Zero} where only shortcut for
\dupml{Monday[{}]} and \dupml{Zero[{}]} that is constructors applied to the
record with no field.

In the type of list, The constructor \dupml{Cons} must be applied to a record
with exactly two labels \dupml{hd} (the head of the list) and \dupml{tl} (the
tail of the list).
Therefore the above definition should be read as an element of type
\dupml{list⟨a⟩} is either \dupml{Nil} or \dupml{Cons[{ hd = h; tl = l}]}
where \dupml{h} is of type \dupml{a} and \dupml{l} is of type
\dupml{list⟨a⟩}.

Let us define a few usual functions on lists:

\begin{pmlcode}
  val rec length : ∀a, list⟨a⟩ ⇒ nat = fun l {
    case l {
      []       → zero
      hd :: tl → succ (length tl)
    }
  }

  val rec map : ∀a b, (a ⇒ b) ⇒ list⟨a⟩ ⇒ list⟨b⟩ = fun fn l {
    case l {
      []       → []
      hd :: tl → fn hd :: map fn tl
    }
  }

  infix (@) = app priority 3 left associative

  val rec (@) : ∀b, list⟨b⟩ ⇒ list⟨b⟩ ⇒ list⟨b⟩ = fun l1 l2 {
    case l1 {
      []       → l2
      hd :: tl → hd :: (tl @ l2)
    }
  }
\end{pmlcode}

First, as lists are very common, two syntactic sugars are provided:
\dupml{[]} for \dupml{Nil} and \dupml{h :: t} for
\dupml{Cons[{hd = h; tl = t}]}. They can be used both in terms and pattern.
  The symbol  \dupml{(::)} is right associative with a priority of 5.

The \dupml{length} and \dupml{map} function should not be very surprising.
One should note the \dupml{∀} in their type as they are \emph{polymorphic}:
they work for lists of elements of any type. For the append function, as we
previously did, we declare the infix symbol we want to use prior to the
definition.

Here are two classical functions on lists:
\begin{pmlcode}
  val rec rev_app : ∀a, list⟨a⟩ ⇒ list⟨a⟩ ⇒ list⟨a⟩ =
    fun l1 l2 {
      case l1 {
        [] → l2
        hd::tl → rev_app tl (hd::l2)
      }
    }

  val rec rev : ∀a, list⟨a⟩ ⇒ list⟨a⟩ =
    fun l { rev_app l [] }

  val rec fold_left : ∀a b, (a ⇒ b ⇒ a) ⇒ a ⇒ list⟨b⟩ ⇒ a =
    fun f a l {
      case l {
        []     → a
        hd::tl → fold_left f (f a hd) tl
      }
    }

  val rec fold_right : ∀a b, (b ⇒ a ⇒ a) ⇒ list⟨b⟩ ⇒ a ⇒ a =
    fun f l a {
      case l {
        []     → a
        hd::tl → f hd (fold_right f tl a)
      }
    }
\end{pmlcode}

The function \dupml{rev_app} is similar to \dupml{(@)}
except it reverse its first argument while concatenating it
on the second. However, it is more efficient because
\dupml{(@)} must remember in the stack the job to do after
the recursive call while for \dupml{rev_app}, the result of the recursive
call is the final result. This means that the second function is not using
a lot of stack space. We say that \dupml{rev_app} is \emph{recursive terminal}.

Then, from \dupml{rev_app} we can define \dupml{rev} to reverse a list.

the two last functions are very classical and defined by the following
equations:
\begin{eqnarray*}
  \dupml{fold_left f a [b1; b2; ⋯ ; bn]} &= \dupml{f (⋯ (f (f a b1) b2) ⋯)
    bn}\\
  \dupml{fold_right f [b1; b2; ⋯ ; bn] a} &= \dupml{f b1 (f b2 (⋯ (f bn a) ⋯))}
\end{eqnarray*}

Notice that the order of the variables \dupml{a}, \dupml{b1}, ⋯, \dupml{bn}
is the same on both side of each equation. The order of the parameters where
chosen to ensure this. Again, \dupml{fold_left} is recursive terminal.
We can now prove a property of these two functions:

\begin{pmlcode}
  val swap : ∀a b c, (a ⇒ b ⇒ c) ⇒ (b ⇒ a ⇒ c) =
    fun f { fun b a { f a b } }

  val rec fold_rev_app : ∀a b, ∀f∈(b ⇒ a ⇒ a), ∀x∈a, ∀l1 l2∈list⟨b⟩,
      fold_left (swap f) x (rev_app l1 l2)
      ≡ fold_left (swap f) (fold_right f l1 x) l2 =
    take f x l1 l2;
    case l1 {
      []     → show fold_left (swap f) x (rev_app l1 l2)
               ≡ fold_left (swap f) x l2
               ≡ fold_left (swap f) (fold_right f l1 x) l2
      hd::tl → show fold_left (swap f) x (rev_app l1 l2)
               ≡ fold_left (swap f) x (rev_app tl (hd::l2))
               ≡ fold_left (swap f) (fold_right f tl x) (hd::l2)
                 by fold_rev_app f x tl (hd::l2)
               ≡ fold_left (swap f) (f hd (fold_right f tl x)) l2
               ≡ fold_left (swap f) (fold_right f l1 x) l2
    }

  val rec fold_rev :  ∀a b, ∀f∈(b ⇒ a ⇒ a), ∀x∈a, ∀l∈list⟨b⟩,
      fold_right f l x ≡ fold_left (swap f) x (rev l) =
    take f x l;
    use fold_rev_app f x l []
\end{pmlcode}

This property is usefull in practice because it might be easier to
prove the correctness of some code with \dupml{fold_right} while
one might want to use \dupml{fold_left} for efficiency. This result  allows
to do both!

\begin{exercise}\label{exo-list-1}
  Prove that reversing twice a list gives the original list. You will need a
  lemma about \dupml{rev_app} which is not surprising ⋯ but also a lemma
  about \dupml{(@)}.
\end{exercise}

\begin{exercise}\label{exo-list-2}
  Prove some result about the composition of \dupml{map} with itself, \dupml{fold_right}
  and \dupml{fold_left}. This kind of lemmas is useful to avoid building
  intermediate lists.
\end{exercise}

\section{Totality issues}\label{sect-totality}

PML, as already mentionned, uses a call-by-value evaluation strategy and
supports some side effectq like LISP call-cc (see \ref{sect-classical}) or non
termination.  This leads to difficulties\dots For instance, by definition of
addition, we know that \dupml{S[x] + g y ≡ S[x + g y]}, but this can not be
proved if \dupml{g y} does not evaluate to a value. If one know that \dupml{g
  y} evaluates to a value, adding a line \dupml{let _ = g y;} will solve the
problem.

Moreover, PML will add such a let automatically when a computation is blocked
by an application which is not know yet to be total. PML will do this at most
3 times because the default level of automation is \dupml{set auto 0 3}. This
means no case analysis added and at most three \dupml{let} definition.

Here is an exemple that illustrates this. The following proof does not
type-check.

\begin{badpmlcode}
   val rec map_app : ∀a b, ∀f∈(a ⇒ b), ∀l1 l2∈list⟨a⟩,
                           map f l1 @ map f l2 ≡ map f (l1 @ l2) =
     take f l1 l2;
     case l1 {
       [] → qed
       h::l → use map_app f l l2
     }
\end{badpmlcode}

Let us write a more detailed proof:
\begin{pmlcode}
   val rec map_app : ∀a b, ∀f∈(a ⇒ b), ∀l1 l2∈list⟨a⟩,
                           map f l1 @ map f l2 ≡ map f (l1 @ l2) =
     take f l1 l2;
     case l1 {
       [] → qed
       h::l →  show map f l1 @ map f l2
                 ≡ (f h :: map f l) @ map f l2
                 ≡ f h :: (map f l @ map f l2)
                 ≡ f h :: (map f (l @ l2)) by map_app f l l2
                 ≡ map f (h :: (l @ l2))
                 ≡ map f (l1 @ l2)
     }
\end{pmlcode}

This works because there are more steps and each step needs less than three
\dupml{let} added.

Let us write a proof with all the needed \dupml{let} being explicit (these
are terms that appears in the previous proof and that would block some
evaluation if they where not values):

\begin{pmlcode}
   val rec map_app : ∀a b, ∀f∈(a ⇒ b), ∀l1 l2∈list⟨a⟩,
                           map f l1 @ map f l2 ≡ map f (l1 @ l2) =
     take f l1 l2;
     case l1 {
       [] → qed
       h::l → set auto 0 0; // make sure no let added
              let _ = map f l;
              let _ = map f l2;
              let _ = l @ l2;
              let _ = f h;
              use map_app f l l2
     }
\end{pmlcode}

And a much simpler proof just increasing the auto level:
\begin{pmlcode}
   val rec map_app : ∀a b, ∀f∈(a ⇒ b), ∀l1 l2∈list⟨a⟩,
                           map f l1 @ map f l2 ≡ map f (l1 @ l2) =
     take f l1 l2;
     case l1 {
       [] → qed
       h::l → set auto 0 4; use map_app f l l2
     }
\end{pmlcode}

A few explanation on why this is needed beside the fact that
by choice PML does not have only total functions (i.e. function yielding
values). In the above example, we need the totality of for instance
\dupml{map f l}. By typing, as we use only the arrow denoting total function,
PML knows this yield a value. But in two of the four proofs above, this
expression does not even appears. In the detailed proof, using the
\dupml{show} keyword, the expression appears, but equivalence in PML is
untyped! The use of an untyped equivalnce may seem strange, but is essential to allow proving and using
correct programs that are untyped, provided one can prove their correction.
Thus \dupml{let} (or something equivalent), are necessary to force PML
to typecheck these expressions. As this is very annoying, PML
 will add the needed \dupml{let} automatically.

%<!-- Local IspellDict: british -->
%<!-- Local IspellPersDict: ~/.ispell-british -->


\chapter{Advanced feature}

\section{Dealing with termination}\label{sect-nontermination}

Here is a classical example used to break termination
prover: the Mc Carthy 91 function. Here is the definition
of the function, using the arrow denoting possible non termination
as PML can not prove the termination of this function.

\begin{pmlcode}
  include lib.nat
  include lib.nat_proofs

  val rec mccarthy91 : nat ↛ nat = fun n {
    if n > 100 {
      n - 10
    } else {
      mccarthy91 (mccarthy91 (n + 11))
    }
  }
\end{pmlcode}

In fact this function is computing the same value (in a very complex way)
as the following.

\begin{pmlcode}
  include lib.nat

  val mccarthy91_easy : nat ⇒ nat = fun n {
    if n > 100 {
      n - 10
    } else {
      91
    }
  }
\end{pmlcode}

Now it is not very hard to prove the equivalence of both functions,
using 101 cases automatically generated by PML and computing
\dupml{mccarthy91 n} for all integers between 0 and 100!

\begin{pmlcode}
  val hard_lemma : ∀n∈nat, n ≤ 100 ⇒ mccarthy91 n ≡ 91 =
    take n; suppose n ≤ 100;
    set auto 101 1; {- qed -} // this takes ~3mn so we commented it

  val hard_is_easy : ∀n∈nat, mccarthy91_easy n ≡ mccarthy91 n =
    take n;
    if n > 100 {
      show mccarthy91_easy n ≡ n - 10 ≡ mccarthy91 n
    } else {
      deduce n ≤ 100 using { geq_gt 100 n; leq_geq n 100};
      show mccarthy91_easy n ≡ 91
        ≡ mccarthy91 n using hard_lemma n {}
    }
\end{pmlcode}

Notice the use of some lemmas from \dupml{lib.nat_proofs} to
prove that \dupml{n ≤ 100} is the second case.

Now, we can define a function that will be typed as \dupml{mccarthy_easy}
but evaluated as \dupml{mccarthy91}. This is useless here, but would be
useful if \dupml{mccarthy91} were an optimised version of
\dupml{mccarthy_easy}.

\begin{pmlcode}
  val mccarthy91 : nat ⇒ nat =
  fun n {
      check { mccarthy91_easy n }
      for { mccarthy91 n }
      because { hard_is_easy n }
  }
\end{pmlcode}

The \dupml{check ⋯ for ⋯} construct require a proof that the two
expressions are equivalent to accept to typecheck one expression and replace
it by the second expression. This new version of \dupml{mccathy91} is really
equivalent to the original one but with a stronger type (using the total arrow).

\section{Inductive and co-inductive data types}

\section{Classical proofs}\label{sect-classical}

%<!-- Local IspellDict: british -->
%<!-- Local IspellPersDict: ~/.ispell-british -->


\chapter{PML language and syntax}

\section{Sorts}\label{lang-sort}

PML language uses \emph{sorts} to classify all the expressions of the
languages (except sorts) like types, values, ordinals, \dots


\def\w{9.2cm}
\begin{longtable}{rll}
\multicolumn{2}{l}{\noindent \emph{sort} $::=$}\\
  & $\iota$ \Mid {\tt <iota>} \Mid {\tt <value>}
  & \parbox[t]{\w}{denote the sort for values i.e. results of computation.} \\

  \Mid
  & $\tau$ \Mid {\tt <tau>} \Mid {\tt <term>}
  & \parbox[t]{\w}{denote the sort for terms i.e. program that evaluates to value.} \\

  \Mid
  & $\sigma$ \Mid {\tt <sigma>} \Mid {\tt <stack>}
  & \parbox[t]{\w}{denote the sort of stacks which is used when programming with classical
  logic.}\\

  \Mid
  & $o$ \Mid {\tt <omicron>} \Mid {\tt <prop>}
  & \parbox[t]{\w}{denote the sort of types for terms or propositions (following the
  Curry-Howard correspondance PML identifies types and proposition).
  The letter $o$ is the unicode character omicron, not a lower case latin letter.} \\

  \Mid
  & $\kappa$ \Mid {\tt <kappa>} \Mid {\tt <ordinal>}
  & \parbox[t]{\w}{denote the sort of ordinals used to index inductive and co-inductive
  types.} \\

  \Mid
  & \emph{sort} $\rightarrow$ \emph{sort}
  & \parbox[t]{\w}{the sort for higher-order function like types with parameters. These
  higher-order functions should not be confused with functions as programs
  which are of sort $\iota$ or $\tau$. This symbol is right associative,
  $s_1 \rightarrow  s_2 \rightarrow s_3$ means
  $s_1 \rightarrow (s_2 \rightarrow s_3)$.} \\

  \Mid
  & {\tt (} \emph{sort} {\tt )}
  & \parbox[t]{\w}{parenthesis can be used for grouping.} \\
\end{longtable}

The sort $\iota$ is a subsort of $\tau$. This means that any value can also
be considered as a term.

\section{Expressions}\label{lang-expr}

In this section we describe the BNF of PML's expressions written \emph{expr}. We will denote
\begin{itemize}
\item \emph{value} the expression of sort $\iota$,
\item \emph{term} the expression of sort $\tau$,
\item \emph{stack} the expression of sort $\sigma$,
\item \emph{prop} the expression of sort $o$,
\item \emph{ordi} the expression of sort $\kappa$.
\end{itemize}

\subsection{Atoms}\label{lang-atoms}

We now give a few atomic tokens used in the grammar below.

\begin{longtable}{rll}
\emph{uid} &$::=$& {\tt [A-Z\_][a-zA-Z0-9\_']*} \Mid {\tt true}
\Mid {\tt false}\\
\emph{lid} &$::=$& {\tt [a-z\_][a-zA-Z0-9\_']*} \\
\emph{path} &$::=$& \emph{lid} \Mid \emph{lid}\dupml{.}\emph{path} \\
\emph{id}  &$::=$& \emph{lid} \Mid \emph{lid} : \emph{prop} \\
\emph{lbl} &$::=$& {\tt [a-zA-Z0-9\_']+} \\
\emph{int} &$::=$& {\tt [-]?[0-9]+} \\
\emph{float} &$::=$& {\tt [0-9]*('.'[0-9]*)?([eE][-+]?[0-9]*)?} \\
\emph{infix} &$::=$& \hbox{ must be declared see \ref{lang-top}}.\\
\emph{infix\_re} &$::=$& {\tt [\^~\!\!\!][(){}a-zA-Z0-9\_'"";,. $\backslash$n$\backslash$t$\backslash$r]+}\\
\emph{string} &$::=$& \dupml{"}.*\dupml{"}
  \hbox{ with the usual quoting like \dupml{"\n"}.}\\
\end{longtable}

\subsection{Values}

The priority level for terms and values are, from the lowest to the highest priority:
\begin{itemize}
\item F: full
\item S: sequence
\item R: prefix
\item I: infix
\item P: application
\item A: atom
\end{itemize}

When the priority is not F, we subscrit the non terminal
$\emph{value}$ and $\emph{term}$ with the priority.
We write the priority of each rule at the end of the rule.
The priority between infix symbols and their associativity are given
at the declaration see \ref{lang-top}.

\def\w{10cm}
\begin{longtable}{rlcl}
  \multicolumn{2}{l}{\emph{value} $::=$}\\
  & $\dupml{λ}\emph{id} \dots \emph{id}\dupml{.} \emph{term}_I$ & R & \\
  \Mid & $\dupml{fun} \; \emph{id} \dots \emph{id} \; \{ \emph{term} \}$ & A
    & \parbox[t]{\w}{denotes function with one or more arguments.} \\

  \Mid & \dupml{[]} & A &\\
  \Mid & $\emph{value}_I \dupml{::} \emph{value}_I$ & I &\\
  \Mid & $\emph{uid} \mid \emph{uid} \dupml{[} \emph{value} \dupml{]}$ & A
  & \parbox[t]{\w}{constructor applied to no argument or one argument. In
    fact, no argument means applied to unit, i.e. \dupml{{}}. \dupml{[]} is a short
    hand for $\dupml{Nil}$ and \dupml{t :: u} means \dupml{Cons[{hd = t; tl = u}]}.} \\

  \Mid & $\{ \emph{lbl} \dupml{=} \emph{value} \dupml{;} \dots \}$ & A
  & \parbox[t]{\w}{construction of a record. As in OCaml, a label alone means
    the label used both as label and value. For
    instance, \dupml{{ hd; tl }} means  \dupml{{ hd = hd; tl = tl }}.} \\

  \Mid & $\dupml{(} \emph{value}\dupml{,} \dots\dupml{)}$ & A
  & \parbox[t]{\w}{tuple construction. As in standard ML this is equivalent
    to a record with numerical labels $1,2,\dots$.} \\

  \Mid & \dupml{(} \emph{value} \dupml{:} \emph{prop} \dupml{)} & A
  & \parbox[t]{\w}{type annotation} \\
\end{longtable}


\subsection{Terms}

\def\w{9.2cm}
\begin{longtable}{rlcl}
    \multicolumn{2}{l}{\emph{term} $::=$}\\
  & $\emph{term}_P \; \emph{term}_A$ & P
  & \parbox[t]{\w}{function application. It is left associative: \dupml{t1 t2
    t3} means \dupml{(t1 t2) t3}.}\\

  \Mid & $\emph{term}_I \dupml{::} \emph{term}_I$ & I \\ \Mid & $\emph{uid}
  \dupml{[}\emph{term}\dupml{]}$ & A & \parbox[t]{\w}{syntactic sugar :
    \dupml{C[t]} means \dupml{(λx.C[x])t} when \dupml{t} is not a value. The
    same for the syntactic sugar for {\tt Cons} if one or both arguments are
    not valuesaf, one or two redexes are inserted.} \\

  \Mid & $\{ \emph{lbl} \dupml{=} \emph{term} ; \dots \}$ & A
  & \parbox[t]{\w}{As above, a redex is added for each field which is not a
    value in a record construction.} \\

  \Mid & $\dupml{(}\emph{term}, \dots\dupml{)}$ & A
    & \parbox[t]{\w}{As above, a redex is added for each field which is not a
    value in a tuple construction.} \\

  \Mid & \emph{int} & A & \parbox[t]{\w}{a syntactic sugar is created to
    represent the given integer using function {\tt zero},  \dupml{succ}, {\tt
      dble}, and  \dupml{opp}. For instance, \dupml{5} means \dupml{succ (dble
      (dble (succ zero)))}.} \\

  \Mid & $\!\!\!\begin{array}[t]{l}\dupml{if} \; \emph{term} \; \{ \emph{term} \}\\ \dupml{else}
  \; \{ \emph{term} \}\end{array}$ & A \\
  \Mid & $\dupml{case} \; \emph {term} \; \{ \emph{pat}  \dupml{→}\emph{term} \dots \}$ & A &
    \parbox[t]{\w}{case analysis over sum types. See below for the definition
      of the BNF \emph{pat}. The syntax $\dupml{if b {t} else {u}}$ means
      \dupml{case b { true → t false → u}}.} \\

  \Mid & U+2702 & A & \parbox[t]{\w}{scissors denote inaccessible part of
    a term, in general an impossible case}\\

  \Mid & $\dupml{save} \; \emph{lid} \{ \emph{term} \}$ & A & \parbox[t]{\w}{saves the
    current evaluation context (i.e. the stack) for future use with {\tt
      \dupml{restore}}.} \\

  \Mid & $\dupml{restore} \; \emph{lid} \; \emph{term}_A $ & I & \parbox[t]{\w}{restores
    a previously stored evaluation context and evaluate the given term in
    it. This statement corresponds to the $c$ effect.}\\

  \Mid & $\dupml{delim} \; \{ \emph{term} \} $ & A & \parbox[t]{\w}{If the given term
    does not use the $c$ effect in its type or the type of its free variables,
    we know that the \dupml{restore} statement can not escape the term and
    the $c$ effect can be ignored for that term.} \\

  \Mid & $\dupml{print} \; \emph{string}$ & A & \parbox[t]{\w}{Print the given
    string on the standard output. This expression is annotated with the $p$
    effect.}\\

  \Mid & $\dupml{fix} \; \emph{id} \{ \emph{term} \}$ & A & \parbox[t]{\w}{
      The term $\dupml{fix x {t}}$ is a fixpoint, meaning that
      it is equivalent to $\dupml{t}[\dupml{x}:=\dupml{fix x { t }}]$. PML must be able
      to prove the termination of this fixpoint unless the current context is
      annotated with the $l$ effect.} \\

  \Mid & $\emph{term}_R \dupml{;} \emph{term}_S$ & S & \parbox[t]{\w}{
    sequence of terms are redexes. The term \dupml{t; u} means \dupml{(fun x
      {u}) t}.}\\

  \Mid & $\dupml{let}\; \emph{id} \dupml{=} \emph{term}_R\dupml{;} \emph{term}_S$ & S \\
  \Mid & $\dupml{let rec}\; \emph{id} \dupml{=} \emph{term}_R\dupml{;} \emph{term}_S$ & S &
    \parbox[t]{\w}{The first form is a syntactic sugar for a redex.
      \dupml{let x : a = u; t} means \dupml{(fun x : a { t }) u}.
      The second form uses a fixpoint in the definition:
      \dupml{let rec x : a = u; t} means \dupml{(fun x : a { t }) (fix x : a {u})}.} \\

  \Mid & \dupml{(} \emph{term} \dupml{:} \emph{prop} \dupml{)} & A
  & \parbox[t]{\w}{type annotation} \\

  \Mid & $\dupml{{- ⋯ -}}$ & A & \parbox[t]{\w}{denotes a hole in a term, a term
      yet to be written. It must not contain a newline, but could use any
      other character.}\\

  \Mid & $\dupml{set} \; \emph{option} \dupml{;} \emph{term}_S$ & S &
      \parbox[t]{\w}{allow to set some option to tune pml
        behavior for the typechecking of the given term. See \ref{lang-top}
        for more detail.}\\
\end{longtable}

\subsection{Stacks}

There is no specific syntax for stacks except for identifier introduced by the
\dupml{save} keyword.

\def\w{9.2cm}
\begin{longtable}{rll}
\end{longtable}

\subsection{Ordinals}

Ordinals are used only to index least and greatest fixpoint used to build
inductive and co-inductive type. They are used to prove termination of
programs by \emph{ordinal induction} but also to prove subtyping.

\def\w{12cm}
\begin{longtable}{rll}
  \multicolumn{2}{l}{\emph{ordi} $::=$}\\
  & $\infty$ & \parbox[t]{\w}{denotes an ordinal large enough to imply the
    convergence of all PML $mu$ and $nu$ fixpoint. $\mu x, p$ means $\mu\_\infty x, t$. } \\
  \Mid & $\emph{ordi} +_o \emph{int}$ & \parbox[t]{\w}{denotes the addition
    of a positive natural to an ordinal.} \\
\end{longtable}

\subsection{Propositions}

Prorities for parsing proposition are the following from lowest to highest:
\begin{itemize}
  \item $F$ : full
  \item $P$ : product
  \item $R$ : restriction
  \item $M$ : member
  \item $A$ : atom
\end{itemize}

\def\w{9.5cm}
\begin{longtable}{rlcl}
  \multicolumn{2}{l}{\emph{prop} $::=$}\\
       & $\emph{prop}_P \dupml{⇒} \emph{prop}$ & F & \\
  \Mid & $\emph{prop}_P \dupml{→} \emph{prop}$ & F & \\
  \Mid & $\emph{prop}_P \dupml{↝} \emph{prop}$ & F &\\
  \Mid & $\emph{prop}_P \dupml{→_}(\emph{effects}) \emph{prop}$ & F &
  \parbox[t]{\w}{propositional
  implication and function type. The implication is indexed by the possible
  effects of the function. Currently $l$ if the function may loop, $p$
    if the function performs some printing on the standard output and $c$
    if the function may restore a previously saved stack. The symbol
  \dupml{⇒} is a short cut for pure function type
  i.e. \dupml{→_()}. The symbol \dupml{→} and \dupml{↝} mean respectively
  \dupml{→_(cp)} and \dupml{→_(cpl)}. }\\

  \Mid & $\emph{prop}_R \dupml{×} \emph{prop}_R \dupml{×} \dots$ & P & product type (i.e. type
  of tuple.). \\

  \Mid & $\{ \emph{lid} : \emph{prop} \dupml{;} \dots \}$ & A &\\
  \Mid & $\{ \emph{lid} : \emph{prop} \dupml{;} \dots \dupml{;} \dupml{...} \}$ & A &
  \parbox[t]{\w}{record type, indicating the type of each field. If there is
    three dots before the closing brace, the record type is open, more fields
    may be present in an element of this type. In fact, the strict version is
    a syntactic sugar.  \dupml{{ l : A; l' : B}} means \dupml{∃x y:ι, {
        l = x; l' = y } ∈ { l : A; l' : B ; ⋯ }}.  The final dots may
    be one unicode character or three ascii dots.} \\

  \Mid & $\emptyset$ & A &\\
  \Mid & \dupml{[.]} & A &\\
  \Mid & $\dupml{[} \emph{uid} \; \dupml{of} \; \emph{prop} ; \dots \dupml{]}$ & A &
  \parbox[t]{\w}{sum type indicating the type of the argument of each
    constructor. The annotation $\dupml{of}\; \emph{prop}$ may be ommited if the type is
    unit, i.e. \dupml{{}}. $\emptyset$ and \dupml{[.]} both denote the empty sum
    type. \dupml{[]} is the empty list and not a sum type.} \\

  \Mid & $\emph{term}_I \dupml{∈} \emph{prop}_M$ & M &
  \parbox[t]{\w}{denotes the singleton type containing only the given term of
    the given type.} \\

  \Mid & $\emph{prop}_R \dupml{|} \emph{term}_I \dupml{≡} \emph{term}_I$ & R &\\
  \Mid & $\emph{prop}_R \dupml{|} \emph{term} \dupml{↓}$ & R &\\
  \Mid & $\emph{term}_I \dupml{≡} \emph{term}_I$ & R &\\
  \Mid & $\emph{term}_I$ & A &
  \parbox[t]{\w}{corresponds to the restiction type. It is the empty type if
    the condition is false, otherwise it is the given type. The condition
    $t_1 \equiv t_2$ is true when $t_1$ and $t_2$ are equivalent (PML
    equivalence is more of less observational equivalence). $t_1\downarrow$
    is true if $t_1$ evaluates to a value. The two last forms are syntactic
    sugar: \dupml{t ≡ u} means \dupml{{} | t ≡ u} and \dupml{t} used as a
    proposition means \dupml{{} | t ≡ true}. The last form allows to use any
    term of boolean type as a proposition.} \\

  \Mid & $\emph{term}_I \dupml{≡} \emph{term}_I \dupml{↪} \emph{prop}_M$ & R & \\
  \Mid & $\emph{term}_I \dupml{↓}  ↪ \emph{prop}_M$ & R &
  \parbox[t]{\w}{denotes propositional implication. The type is the top type
    containing all terms if the condition is false. It is the given type
    otherwise. This type differs from restriction only when the condition is false.} \\

  \Mid & $\dupml{∀} \emph{id} \dots\dupml{,} \emph{prop}$ & F & \\
  \Mid & $\dupml{∀} \emph{id} \dots\dupml{:} \emph{sort}\dupml{,} \emph{prop}$ & F &\\
  \Mid & $\dupml{∀} \emph{id} \dots\dupml{∈} \emph{prop}\dupml{,} \emph{prop}$ & F &
  \parbox[t]{\w}{denotes universal quantification. In the first form, the
    sort of the variables is infered by PML while in the second form it is
    explicitely written. The last form is a short cut for dependant type.
    \dupml{∀ x y ∈  A,B} means \dupml{∀ x y : ι, x ∈ A ⇒ y ∈ A ⇒  B}.} \\

  \Mid & $\dupml{∃} \emph{id} \dots\dupml{,} \emph{prop}$ & F &\\
  \Mid & $\dupml{∃} \emph{id} \dots\dupml{:} \emph{sort}\dupml{,} \emph{prop}$ & F &\\
  \Mid & $\dupml{∃} \emph{id} \dots\dupml{∈} \emph{prop}\dupml{,} \emph{prop}$ & F &\\
  \Mid & $\{ \emph{id} \dupml{∈} \emph{prop}\}$ & F &
  \parbox[t]{\w}{denotes existential quantification. In the first form, the
    sort of the variables is infered by PML while in the second form it is
    explicitely written. The last form is a short cut for dependant product.
    \dupml{∃ x y ∈ A,B} means \dupml{∃ x y : ι, x
    ∈ A × y ∈ A × B}. The last form is also a
    short cut: \dupml{{x ∈  A}} means \dupml{∃ x, x ∈ A}.} \\

  \Mid & $\dupml{μ} \emph{id}\dupml{,} \emph{expr}$ & F &\\
  \Mid & $\dupml{μ_}\emph{ordi} \; \emph{id}\dupml{,} \emph{expr}$ & F &
  \parbox[t]{\w}{Least fixpoint of proposition or parametric
    proposition. This means that \dupml{μx, a} may be of sort $s_1 \rightarrow
    \dots s_n \rightarrow o$ when both \dupml{x} and \dupml{a} have this sort. The
    form with an ordinal denotes a partial construction of the fixpoint and
    is used to prove termination of programs.} \\

  \Mid & $\dupml{ν} \emph{id}\dupml{,} \emph{expr}$ & F &\\
  \Mid & $\dupml{ν_}\emph{ordi} \; \emph{id}\dupml{,} \emph{expr}$ & F &
  \parbox[t]{\w}{Greatest fixpoint of proposition or parametric
    proposition. This means that \dupml{νx, a} may be of sort $s_1 \rightarrow
    \dots s_n \rightarrow o$ when both \dupml{x} and \dupml{a} have this sort. The
    form with an ordinal denotes a partial construction of the fixpoint and
    is used to prove productivity of programs.} \\
\end{longtable}

\subsection{General expressions}


\def\w{11cm}
\begin{longtable}{rll}
  \multicolumn{2}{l}{\emph{expr} $::=$}\\
  & \emph{expr} \Mid \emph{value} \Mid \emph{ordi} \Mid \emph{prop} \\
  \Mid & \emph{lid} & variable. \\

  \Mid & $\dupml{(} \emph{lid} \dupml{:} \emph{sort} \dupml{↦} \emph{expr} \dupml{)}$ &
  \parbox[t]{\w}{corresponds to
  higher-order abstraction. If \dupml{e} is an expression of sort \dupml{s} using a
  variable \dupml{x} of sort \dupml{t} then \dupml{(x : t ↦ e)} is of sort
  \dupml{t → s}.} \\

  \Mid & $\emph{expr}\dupml{⟨}\emph{expr}, \dots\dupml{⟩}$ &
  \parbox[t]{\w}{denotes higher-order application. \dupml{e1⟨e2,e3⟩} is a short cut for \dupml{e1⟨e2⟩⟨e3⟩}. \dupml{e1⟨e2⟩} is of sort \dupml{s} if \dupml{e1} is of sort
    \dupml{t → s} and \dupml{e2} is of sort \dupml{s1}.}\\

  \Mid & $\emph{lid}\dupml{^}\emph{ordi}\dupml{⟨}\emph{expr}, \dots\dupml{⟩}$ &
  \parbox[t]{\w}{The identifier must have been declared with \dupml{type rec}
    or \dupml{type corec} and then, this corresponds to the same type with
    the \dupml{μ} or \dupml{ν} fixpoint indexed by the given ordinal (see
    section \ref{lang-top}).} \\

  \Mid & \dupml{(} \emph{expr} \dupml{)} & parenthesis for grouping.\\
\end{longtable}

\subsection{proofs}

PML does not have a specific language for proofs. However, we provide a few
syntactic sugar to make proofs more readable for non specialist.

We first give a BNF for a few equivalent keywords. Depending on the proofs,
some keyword may give a more natural proofs than other.

\begin{longtable}{lllclcl}
\emph{show} &::= & \dupml{show} &\Mid& \dupml{deduce} &\Mid& \dupml{prove}\\
\emph{know} &::= & \dupml{assume} &\Mid& \dupml{know}\\
\emph{from} &::= & \dupml{from} &\Mid& \dupml{showing} \\
\emph{by} &::= & \dupml{because} &\Mid& \dupml{using} &\Mid& \dupml{by} \\
\end{longtable}

First, we have the \dupml{qed} keyword and
two syntactic sugar to define function application that we use
when the type of the variable are hypothesis of a theorem:

\def\w{9.5cm}
\begin{longtable}{rlcl}
  \multicolumn{2}{l}{\emph{value} $::= ...$} & \\
  \Mid & $\dupml{qed}$ & A & a synonymous for the empty record $\{\}$.\\
  \Mid & $\dupml{take} \; \emph{id} \dots \emph{id }  \{ \emph{term} \}$ & A &\\
  \Mid & $\dupml{suppose}\; \emph{prop}, \dots ; \emph{term}_S$ & S &
   \parbox[t]{\w}{the syntax with the \dupml{take} keyword is fully equivalent
     the same with \dupml{fun} and is used to prove universal quantifications.
     We use \dupml{suppose} when we do not need
   a name for the variables, typically for types which are equivalence. For
   instance
   \dupml{suppose a ≡ b; t} means \dupml{fun _ : a ≡ b { t }}.} \\
\end{longtable}

Then, we have a few extension for expressions

\def\w{9.2cm}
\begin{longtable}{rlll}
  \multicolumn{2}{l}{\emph{expr} $::= ...$} \\
  \Mid & $\dupml{use} \; \emph{term}_R$ & R & \parbox[t]{\w}{the statement $\dupml{use}\;
    t$ is equivalent to $t$ but may be more natural when $t$ is the
    invocation of some lemma or theorem.}\\
  \Mid & $\emph{show} \; \emph{prop}_R $ & R \\
  \Mid & $\emph{show} \; \emph{prop} \;\emph{by} \; \emph{term}_R $ & R\\
  \Mid & $\emph{show} \; \emph{prop} \;\emph{by} \; \{\emph{term}\} $ & R \\
  \Mid & $\emph{show} \; \emph{term}_R
         \begin{array}[t]{l}
           \equiv \emph{term}_R \; \emph{by} \; \emph{term}_R \\
           \equiv \emph{term}_R \; \emph{by} \; \emph{term}_R \\
           \dots
         \end{array}$ & S &
         \parbox[t]{\w}{The first form \dupml{show p} is equivalent to
           \dupml{(qed : p)}. The second \dupml{show p by u} means \dupml{(u : p)}. The last form is equivalence to a sequence
           of \dupml{show}, the justification of each line being optional. For
           instance $\begin{array}[t]{ll}\dupml{show a} &\dupml{≡ b
             using l} \\ &\dupml{≡ c} \end{array}$ means $(l :
           a \equiv b); (\dupml{qed} : b \equiv c)$.}\\

  \Mid & $\emph{from} \; \emph{prop}\dupml{;} \emph{term}_S$ & S \\
  \Mid & $\emph{from} \; \emph{prop} \; \emph{by} \; \emph{term}_R \dupml{;} \emph{term}_S$ & S & \parbox[t]{\w}{is
    used for backward reasonning to mean that the current goal is implied by
    the given proposition. This can not be the last statement of a proof
    hence the term at the end. \dupml{from a by u; t} means
     \dupml{let _ : a = u; p}.}\\

  \Mid & $\emph{know} \; \emph{term}_R; \; \emph{term}_S$ &S& \\
  \Mid & $\emph{know} \; \emph{term}_R \; \emph{by} \; \emph{term}_R; \;
  \emph{term}_S$ & S  &
    \parbox[t]{\w}{\dupml{know t; p} perfoms
    a case analysis on \dupml{t} to prove that \dupml{t} is equal to \dupml{true}. It is
    expended to \dupml{case t { false → qed   true → p  }}. \dupml{know t by
    q; p} means \dupml{case t { false → q     true → p  }}. Again, this can not be the last
    statement of a proof hence the final term.} \\

\end{longtable}

\section{Top level statements}\label{lang-top}

PML supports some options to tune the behavior of its kernel using the \dupml{set} statement:
  More details are given on the command line options in section \ref{lang-cmd}.

\begin{longtable}{lrl}
  \emph{option} &::=& \dupml{log} \; \emph{string} \\
  &\Mid& \dupml{auto} \; \emph{int} \; \emph{int}\\
  &\Mid& \dupml{keep_intermediate}
\end{longtable}

\def\w{9.2cm}
\begin{longtable}{lrl}
  \multicolumn{2}{l}{\emph{top} ::=} \\
  & \dupml{sort} \emph{lid} \dupml{=} \emph{sort} & \parbox[t]{\w}{Give a
    definition for a sort.} \\
  \Mid & \dupml{def} \emph{lid} \dupml{⟨}\emph{id}, ...\dupml{⟩} = \emph{expr} \\
  \Mid & \dupml{def} \emph{lid} \dupml{⟨}\emph{id}, ...\dupml{⟩} : \emph{sort} =
  \emph{expr} & \parbox[t]{\w}{Give a definition for an expression of any
    sort. Arguments can be included. Most of the time the \dupml{let} or
    \dupml{type} toplevel command are preferred. More important, a
  value or an expression defined using \dupml{def} is typechecked when used, not
  at the definition. This can be seen as a \emph{macro} mecanism.} \\

  \Mid & \dupml{type} \emph{lid} \dupml{⟨}\emph{id}, ...\dupml{⟩} =
   \emph{prop} \\
  \Mid & \dupml{type rec} \emph{lid} \dupml{⟨}\emph{id}, ...\dupml{⟩} =
    \emph{prop} \\
  \Mid & \dupml{type corec} \emph{lid} \dupml{⟨}\emph{id}, ...\dupml{⟩} =
    \emph{prop} & \parbox[t]{\w}{The toplevel command \dupml{type t⟨a,b⟩ =
        ty} is equivalent to \dupml{def t⟨a,b⟩ : ο = ty}. \dupml{type rec t⟨a,b⟩ =
        ty} is equivalent to \dupml{def t⟨a,b⟩ : ο = μt , ty}. With
      \dupml{corec} \dupml{μ} is replaced by \dupml{ν}. When using \dupml{type rec} or
      \dupml{type corec}, the notation \dupml{t^o⟨a,b⟩} can be used to index
      the fixpoint by the given ordinal.}\\

  \Mid & \dupml{val} \emph{lid} : \emph{prop} = \emph{term} \\
  \Mid & \dupml{val rec} \emph{lid} : \emph{prop} = \emph{term} &
    \parbox[t]{\w}{This gives a name to the value resulting of the evaluation of the given
      term. Typechecking is performed before evaluation. With \dupml{val rec} a
      fixpoint is added in front of the definition.}\\

  \Mid & \dupml{assert} \emph{prop} ⊂ \emph{prop} \\
  \Mid & \dupml{assert} ¬ \emph{prop} ⊂ \emph{prop} & \parbox[t]{\w}{Allows
    to check that a subtyping relation can or can not be proved to hold by PML.}  \\

  \Mid & \dupml{include} \emph{path} & \parbox[t]{\w}{Load the definitions in
    a PML file. The last component of the path must be the name of a {\tt
      .pml} file without its extension. The previous elements of the path are
    folder names. PML searches file in the current directory and in the path
    directories (use {\tt --config} to know the current path).}\\

  \Mid & \dupml{set} \; \emph{option} & \parbox[t]{\w}{Set option controling
    PML kernel, see section \ref{lang-cmd}.}\\

  \Mid &
    \multicolumn{2}{l}{\dupml{infix (} \emph{infix\_re} \dupml{) =} \emph{lid} \dupml{priority}
         \emph{float} \dupml{left associative}} \\
  \Mid & \multicolumn{2}{l}{\dupml{infix (} \emph{infix\_re} \dupml{) =} \emph{lid} \dupml{priority}
         \emph{float} \dupml{right associative}} \\
  \Mid &\multicolumn{2}{l}{\dupml{infix (} \emph{infix\_re} \dupml{) =} \emph{lid} \dupml{priority}
         \emph{float} \dupml{non associative}} \\
  \Mid &
    \multicolumn{2}{l}{\dupml{infix (} \emph{infix\_re} \dupml{) =} \emph{lid}
      \dupml{⟨⟩ priority}
         \emph{float} \dupml{left associative}} \\
  \Mid & \multicolumn{2}{l}{\dupml{infix (} \emph{infix\_re} \dupml{) =} \emph{lid} \dupml{⟨⟩ priority}
         \emph{float} \dupml{right associative}} \\
  \Mid &\multicolumn{2}{l}{\dupml{infix (} \emph{infix\_re} \dupml{) =} \emph{lid} \dupml{⟨⟩ priority}
    \emph{float} \dupml{non associative}} \\
  & & \parbox[t]{\w}{Declaration of an infix symbol. The infix symbol is a
    sequence of authorized characters (see section \ref{lang-atoms}) except
    the reserved infix: \dupml{≡}, \dupml{∈} and \dupml{=}. The infix symbol
    can be associated to a value (the first syntax) or a higher-order
    definition (the latter syntax). The priority is a floating point
    number. The lower the number, the higher the priority.}
\end{longtable}

\section{Command line}\label{lang-cmd}

The command take a list of option described below and a list of pml file to
compile.  Here is the result of \verb!pml --help!:

\begin{verbatim}
Usage: pml [args] [f1.pml] ... [fn.pml]
  --log-file file      Write logs to the provided file.
  --log str            Enable the provided logs. Available options:
                    - a: automatic proving informations
                    - c: comparing informations
                    - e: equivalence decision procedure
                    - f: details of equivalence decision
                    - o: ordinal comparison
                    - p: syntax analysis
                    - s: subtyping informations
                    - t: typing informations
                    - u: unification informations
                    - y: size change principle
                    - z: effect computation.
  --full-compare       Show all the steps when comparing expressions.
  --always-colors      Always use colors.
  --timed              Display a timing report after the execution.
  --recompile          Force compilation of files given on command line.
  --quiet              Disables the printing definition data.
  --config             Prints local configuration.
  --lazy               Use lazy evaluation (default).
  --no-lazy            Do not use lazy evaluation.
  --auto               Set the default level for automatic theorem proving.
                       Two naturals: maximum number of nested case analysis
                       and number of let statement for totality.
  --keep-intermediate  Keep intermediate terms in normalisation in the pool
                       (more complete, yet to prove ? but slower).
  --help               Show this usage message.
  -help                Show this usage message.
  -h                   Show this usage message.
\end{verbatim} 

%<!-- Local IspellDict: british -->
%<!-- Local IspellPersDict: ~/.ispell-british -->



\chapter{Solutions}

\begin{description}
\item[\ref{exo-day-1}]

  \begin{pmlcode}
    include book.part1_doc.basics

    val rec iter : nat ⇒ ∀a, (a ⇒ a) ⇒ a ⇒ a = fun n f a {
      case n {
        0 → a
        S[n] → iter n f (f a)
      }
    }

    val next_day_seven : ∀d∈day, iter 7 next_day d ≡ d =
      take d;
      case d {
        Monday    → qed
        Tuesday   → qed
        Wednesday → qed
        Thursday  → qed
        Friday    → qed
        Saturday  → qed
        Sunday    → qed
      }
  \end{pmlcode}

\item[\ref{exo-bool-1}]

  \begin{pmlcode}
    infix (=>) = imply⟨⟩ priority 8 right associative

    def (=>)⟨b1:τ,b2:τ⟩ = if b1 { b2 } else { true }

    val tauto1 : ∀b1 b2 b3∈bool, (b1 && b2) => b3 ≡ b1 => (b2 => b3) =
      take b1 b2 b3;
      if b1 { if b2 { qed } else { qed }} else { qed }
  \end{pmlcode}

\item[\ref{exo-bool-2}]

  \begin{pmlcode}
    infix (^^) = xor priority 7 right associative

    val xor : bool ⇒ bool ⇒ bool = fun b1 b2 {
      if b1 { not b2 } else { b2 }
    }

    set auto 3 3

    val xor_com : ∀b1 b2∈bool, b1 ^^ b2 ≡ b2 ^^ b1 =
      take b1 b2; qed

    val xor_ass : ∀b1 b2 b3∈bool, b1 ^^ (b2 ^^ b3) ≡ (b1 ^^ b2) ^^ b3 =
      take b1 b2 b3; qed

    val xor_neutral : ∀b1∈bool, b1 ^^ false ≡ b1 =
      take b1; qed

    val xor_opp : ∀b1∈bool, b1 ^^ b1 ≡ false =
      take b1; qed

    val and_com : ∀b1 b2∈bool, b1 && b2 ≡ b2 && b1 =
      take b1 b2; qed

    val and_ass : ∀b1 b2 b3∈bool, b1 && (b2 && b3) ≡ (b1 && b2) && b3 =
      take b1 b2 b3; qed

    val and_abs :  ∀b1∈bool, b1 && false ≡ false =
      take b1; qed

    val and_neutral : ∀b1∈bool, b1 && true ≡ b1 =
      take b1; qed

    val xor_dist_and :
        ∀b1 b2 b3∈bool, b1 && (b2 ^^ b3) ≡ (b1 && b2) ^^ (b1 && b3) =
      take b1 b2 b3; qed
  \end{pmlcode}

\item[\ref{exo-nat-1}] See {\tt lib/nat\_proofs.pml}

\item[\ref{exo-nat-2}]

  \begin{pmlcode}
    include lib.nat
    include lib.nat_proofs

    val rec pow : nat ⇒ nat ⇒ nat = fun n e {
      case e {
        0    → 1
        S[e] → n * pow n e
      }
    }

    val rec pow_add
        : ∀n e1 e2∈nat, pow n (e1 + e2) ≡ pow n e1 * pow n e2 =
      take n e1 e2;
      case e1 {
        0    → show pow n e1 * pow n e2 ≡ 1 * pow n e2
                 ≡ pow n e2 by mul_one_n (pow n e2)
        S[e] → show pow n (e1 + e2) ≡ n * pow n (e + e2)
                 ≡ n * (pow n e * pow n e2) by pow_add n e e2
                 ≡ (n * pow n e) * pow n e2
                   by mul_assoc n (pow n e) (pow n e2)
                 ≡ pow n e1 * pow n e2
      }

    val rec pow_one_n : ∀n∈nat, pow 1 n ≡ 1 =
      take n;
      case n {
        0    → qed
        S[m] → show 1 * pow 1 m ≡ pow 1 m by mul_one_n (pow 1 m)
                                ≡ 1 by pow_one_n m
      }

    val rec pow_mul_rev
        : ∀n e1 e2∈nat, pow n (e1 * e2) ≡ pow (pow n e2) e1 =
      take n e1 e2;
      case e1 {
        0    → qed
        S[e] → show pow n (e2 + e * e2) ≡ pow n e2 * pow n (e * e2)
                 by pow_add n e2 (e * e2)
               ≡ pow n e2 * pow (pow n e2) e by pow_mul_rev n e e2
               ≡ pow (pow n e2) e1
        }

    val rec pow_mul
        : ∀n e1 e2∈nat, pow n (e1 * e2) ≡ pow (pow n e1) e2 =
      take n e1 e2;
      use mul_comm e1 e2;
      use pow_mul_rev n e2 e1;
      qed
  \end{pmlcode}

\item[\ref{exo-list-1}]
  \begin{pmlcode}
    val rec rev_app_rev_app : ∀a, ∀l1 l2 l3∈list⟨a⟩,
        rev_app (rev_app l1 l2) l3 ≡ rev_app l2 (l1 @ l3) =
      take l1 l2 l3;
      case l1 {
        []   → qed
        h::l → show rev_app (rev_app l1 l2) l3
               ≡ rev_app (rev_app l (h::l2)) l3
               ≡ rev_app (h::l2) (l @ l3)
                 by rev_app_rev_app l (h::l2) l3
               ≡ rev_app l2 (l1 @ l3)
      }

    val rec app_l_nil : ∀a, ∀l∈list⟨a⟩, l @ [] ≡ l =
      take l;
      case l {
        []   → qed
        h::l → app_l_nil l
      }

    val rev_rev : ∀a, ∀l∈list⟨a⟩, rev (rev l) ≡ l =
      take l;
      deduce rev (rev l) ≡ l @ [] by rev_app_rev_app l [] [];
      use app_l_nil l
   \end{pmlcode}

\item[\ref{exo-list-2}]
  \begin{pmlcode}
    infix (°) = compose priority 5 right associative
    val (°) : ∀a b c, (b ⇒ c) ⇒ (a ⇒ b) ⇒ (a ⇒ c) =
      fun f g { fun x { f (g x) } }

    val rec map_map : ∀a b c, ∀f∈(b ⇒ c), ∀g∈(a ⇒ b), ∀l∈list⟨a⟩,
                        map f (map g l) ≡ map (f ° g) l =
      take f g l;
      case l {
        []   → qed
        h::l → map_map f g l
      }

    val rec fold_right_map :
      ∀a b c, ∀f ∈(b ⇒ c ⇒ c), ∀g∈(a ⇒ b), ∀l∈list⟨a⟩, ∀x∈c,
       fold_right f (map g l) x ≡ fold_right (fun x y {f (g x) y}) l x =
      take f g l x;
      case l {
        []    → qed
        h::l' → let k = fun x y {f (g x) y};
                show fold_right f (map g l) x
                  ≡ f (g h) (fold_right f (map g l') x)
                  ≡ k h (fold_right k l' x)
                    by fold_right_map f g l' x
                  ≡ fold_right k l x
      }

    val rec fold_left_map :
      ∀a b c, ∀f ∈(c ⇒ b ⇒ c), ∀g∈(a ⇒ b), ∀l∈list⟨a⟩, ∀x∈c,
       fold_left f x (map g l) ≡ fold_left (fun x y {f x (g y)}) x l =
      take f g l x;
      case l {
        []    → qed
        h::l' → let k = fun x y {f x (g y)};
                show fold_left f x (map g l)
                  ≡ fold_left f (f x (g h)) (map g l')
                  ≡ fold_left k (k x h) l'
                    by fold_left_map f g l' (k x h)
                  ≡ fold_left k x l
      }
  \end{pmlcode}
\end{description}
%<!-- Local IspellDict: british -->
%<!-- Local IspellPersDict: ~/.ispell-british -->



\part{The theory}

\chapter{Abstract machine and equivalence}

The notion of computation, or evaluation, is at the heart of the (classical)
realizability techniques that are used for the formal definition of the \pml
language. In this chapter, we give the syntax and the operational semantics
of the language as a pure, untyped, calculus. It will be expressed using a
\emph{Krivine Abstract Machine}, which will allow us to account for control
operators, as well as observational equivalence of programs.\footnote{This
notion of program equivalence will play a very important role as it will be
partially reflected in the type system of the language through ``equivalence
types''.}

\section{Syntax of the abstract machine}

The abstract machine we will consider has the peculiarity of having a
call-by-value reduction strategy, which requires a syntax formed with four
entities: values, terms, stacks and processes. Note that the distinction
between terms and values is specific to our call-by-value presentation, they
would be collapsed in call-by-name.
\begin{definition}[variables]
  We require three disjoint, countable sets of variables $\mathcal{V}_{ι}$,
  $\mathcal{V}_{σ}$ and $\mathcal{V}_{τ}$ for $λ$-variables, $μ$-variables
  and terms variables respectively.
  $$
    \mathcal{V}_{ι} = \{x, y, z ...\}
    \quad\quad\quad
    \mathcal{V}_{σ} = \{α, β, γ ...\}
    \quad\quad\quad
    \mathcal{V}_{τ} = \{a, b, c ...\}
  $$
\end{definition}
As usual, $λ$-variables and $μ$-variables will be bound in terms to
respectively form functions and capture continuations. Term variables
are intended to be substituted by (unevaluated) terms, and not only
values. They will be bound by our fixpoint operator in terms, and they
will also be bound by quantifiers in formulas to express properties
ranging over the set of all terms.

\begin{figure}
  \begin{align*}
    (Λ_{ι})\quad\quad \makebox[1.8em]{v,w}
      \bnfeq &x                     \tag{$λ$-variable}\\
      \bnfor &λx.t                  \tag{$λ$-abstraction}\\
      \bnfor &C_k[v]                \tag{constructor, $k∈\mathbb{N}$}\\
      \bnfor &\{(l_i = v_i)_{i∈I}\}
                            \tag{record, $I ⊆_\text{fin} \mathbb{N}$}\\
      \bnfor &\square               \tag{box, invalid value}\\[10pt]
    (Λ)\quad\quad \makebox[1.8em]{t,u}
      \bnfeq &a                     \tag{term variable}\\
      \bnfor &v                     \tag{value as a term}\\
      \bnfor &t\,u                  \tag{application}\\
      \bnfor &μα.t                  \tag{$μ$-abstraction}\\
      \bnfor &[π]t                  \tag{named term}\\
      \bnfor &v.l_k                 \tag{record projection, $k∈\mathbb{N}$}\\
      \bnfor &[v\,| (C_i[x_i] → t_i)_{i∈I}]
                            \tag{case analysis, $I ⊆_\text{fin} \mathbb{N}$}\\
      \bnfor &φa.v                  \tag{fixpoint}\\
      \bnfor &R(v,t)                \tag{special instruction}\\
      \bnfor &δ(v,w,t)              \tag{special instruction}\\[10pt]
    (Π)\quad\quad \makebox[1.8em]{π,ξ}
      \bnfeq &ε                     \tag{empty stack}\\
      \bnfor &α                     \tag{$μ$-variable}\\
      \bnfor &[{-}\;v]π             \tag{pushed argument}\\
      \bnfor &[t\;\,{-}]π           \tag{pushed function}\\[10pt]
    (Λ \times Π)\quad\quad \makebox[1.8em]{p,q}
      \bnfeq &t∗π                   \tag{processe}
  \end{align*}
  \caption{Syntax of the untyped calculus.}
  \label{fig:untyped_syntax}
\end{figure}

\begin{definition}[untyped calculus]
  Values, terms, stacks and processes are mutually inductively defined by the
  \emph{BNF grammar} of Figure~\ref{fig:untyped_syntax}. The names of the
  corresponding sets are displayed on the left: $(Λ_{ι})$ for values, $(Λ)$
  for terms, $(Π)$ for stacks, and $(Λ \times Π)$ for processes.
\end{definition}

Terms and values form a variation of the $λμ$-calculus \cite{Parigot1992},
enriched with records, variants and a fixpoint operator. Values of the form
$C_k[v]$ (where $k ∈ \mathbb{N}$) correspond to variants, or constructors.
Note that they always have exactly one argument in our language. Case analysis
on variants is performed using the syntax $[v\,| (C_i[x_i] → t_i)_{i∈I}]$,
where the pattern $C_i[x_i]$ is mapped to the term $t_i$ for all index $i$ in
the finite set $I$. Similarly, values of the form $\{(l_i = v_i)_{i∈I}\}$
correspond to records, which are tuples with named fields. The projection
operation $v.l_k$ can be used to access the value labelled $l_k$ in a record
$v$.
\begin{remark}
  The syntax $[v\,| (C_i[x_i] → t_i)_{i∈I}]$ for case analyses and the syntax
  $\{(l_i = v_i)_{i∈I}\}$ for records are part of our meta-language. We only
  use them as a short notation for arbitrary lists of patterns or record
  fields. In the language, the full list of patterns or fields always needs
  to be specified. We would thus, for example, write $\{l_1=v_1; l_2=v_2\}$ or
  $[v\,| C_1[x_1] → t_1 | C_2[x_2] → t_2]$ in the case where $I = \{1, 2\}$.
\end{remark}
Terms of the form $φa.v$ denote a fixpoint, which can be used for general
recursion. They roughly corresponds to the \verb#let rec# construct of the
OCaml language. However, binding a term in a value will allow us to encode
mutually recursive functions with records.
%
The value $\square$ and terms of form $R(v,t)$ or $δ(v,w,t)$ are only
included for a technical purpose. In particular, they are not intended
to be used for programming. The value $\square$ will be used in the
definition of our semantics, terms of the form $R(v,t)$ will allow us
to distinguish records from other sorts of values in our definition of
observational equivalence, and terms of the form $δ(v,w,t)$ will be
used to ensure that our model has an essential property (again related
to equivalence).

\begin{remark}
  We enforce values in constructors, record fields, projections and case
  analyses. This makes the calculus simpler because only $β$-reduction
  will need to manipulate the stack. Syntactic sugar such as the following
  can be defined to hide these restrictions to the user.
  $$
    t.l_k := (λx.x.l_k)\,t
    \quad\quad\quad\quad
    C_k[t] := (λx.C_k[x])\,t
  $$
  The elimination of such syntactic sugar corresponds to a form of partial
  \verb#let#-normalization \cite{Moggi1989} or \verb#A#-normalization
  \cite{Flanagan1993}. The translation can hence be seen as a natural
  compilation step \cite{Tarditi1996, Chlipala2005}.
\end{remark}

\section{Substitution and base evaluation}

We will now define a first evaluation relation of our calculus, that will
then be extended in the next section. Before going into reduction, we first
need to define the notions of free variables and substitutions. Although
they are fairly usual, these definitions introduce our notations.

\begin{definition}[free variables and closed expression]
  Given a value, term, stack or process $ψ$ we denote $FV_ι(ψ)$ (resp.
  $FV_σ(ψ)$, resp. $FV_τ(ψ)$) the set of free $λ$-variables (resp. free
  $μ$-variables, resp. free term variables) of $ψ$. We also denote
  $FV(ψ) = FV_ι(ψ) ∪ FV_σ(ψ) ∪ FV_τ(ψ)$ the set of all the free variables
  of $ψ$. We say that $ψ$ is closed if $FV(ψ) = ∅$. We denote $Λ_{ι}^{*}$
  the set of all the closed values, $Λ^{*}$ the set of all the closed terms,
  and $Π^{*}$ the set of all the closed stacks.
\end{definition}

\begin{definition}[substitutions]
  A substitution is a map $ρ$ such that for all $x ∈ \mathcal{V}_{ι}$ we have
  $ρ(x) ∈ Λ_{ι}$, for all $α ∈ \mathcal{V}_{σ}$ we have $ρ(α)∈Π$, and for all
  $a ∈ \mathcal{V}_{τ}$ we have $ρ(a) ∈ Λ$. Importantly, we also require that
  $ρ(χ) ≠ χ$ for only finitely many $χ ∈ \mathcal{V}_ι ∪ \mathcal{V}_{σ} ∪
  \mathcal{V}_{τ}$. We denote $\mathcal{S}$ the set of all the substitutions,
  and $dom(ρ) = \{χ \st ρ(χ) ≠ χ\}$ the (finite) domain of the substitution
  $ρ$. In particular, $ρ_{id} ∈ \mathcal{S}$ is called the identity
  substitution and is defined as $ρ_{id}(χ) = χ$ for all
  $χ ∈ \mathcal{V}_{ι} ∪ \mathcal{V}_{σ} ∪ \mathcal{V}_{τ}$.
  %
  For every $ρ ∈ \mathcal{S}$ we denote $ρ[x := v]$ (resp. $ρ[α := π]$, resp.
  $ρ[a := t]$) the substitution remapping the variable $x ∈ \mathcal{V}_ι$
  (resp. $α ∈ \mathcal{V}_{σ}$, resp. $a ∈ \mathcal{V}_{τ}$) to the value
  $v ∈ Λ_{ι}$ (resp. stack $π ∈ Π$, resp. term $t ∈ Λ$) in $ρ$. In the case
  where $ρ = ρ_{id}$ we will simply write $[x := v]$, $[α := π]$ and
  $[a := t]$.
  %
  Let $ρ ∈ \mathcal{S}$ be a substitution and $ψ$ be a value, term, stack
  or process. We denote $ψρ$ the value, term, stack or process formed by
  simultaneously substituting (without capture) every variable $χ ∈ FV(ψ)$
  with $ρ(χ)$ in $ψ$.
\end{definition}
\begin{definition}[composition of substitutions]
  Given $ρ_1$, $ρ_2 ∈ \mathcal{S}$ we denote $ρ_1 ∘ ρ_2$ the substitution
  formed by composing $ρ_1$ and $ρ_2$. It is defined by taking
  $(ρ_1 ∘ ρ_2)(χ) = (ρ_1(χ))ρ_2$ for all $χ ∈ dom(ρ_1)$ and it coincides
  with $ρ_2$ on every other variables. In particular, if $ψ$ is a value,
  term, stack or process we will have $ψ(ρ_1 ∘ ρ_2) = (ψρ_1)ρ_2$.
\end{definition}

\begin{figure}
  \begin{align*}
    \hspace{4.5cm}%hack
    t\,u ∗ π          &\quad\;≻\quad  u ∗ [t\;\,{-}]π   \tag{Push}\\
    v ∗ [t\;\,{-}]π   &\quad\;≻\quad  \,t ∗ [{-}\;v]π
                         \tag{if $v ∉ \mathcal{V}_{ι} ∪ \{\square\}$, Swap}\\
    λx.t ∗ [{-}\;v]π  &\quad\;≻\quad  t[x := v] ∗ π     \tag{Pop} \\
    μα.t ∗ π          &\quad\;≻\quad  t[α := π] ∗ π     \tag{Save}\\
    [ξ]t ∗ π          &\quad\;≻\quad  t ∗ ξ             \tag{Restore}\\
    \{(l_i = v_i)_{i∈I}\}.l_k ∗ π
                      &\quad\;≻\quad  v_k ∗ π       \tag{if $k ∈ I$, Find}\\
    [C_k[v]\,| (C_i[x_i] → t_i)_{i∈I}] ∗ π
                      &\quad\;≻\quad  t_k[x_k := v] ∗ π
                                                    \tag{if $k ∈ I$, Match}\\
    φa.v ∗ π          &\quad\;≻\quad  v[a := φa.v] ∗ π  \tag{Unfold}\\
    R(\{(l_i = v_i)_{i∈I}\},t) * π
                      &\quad\;≻\quad  t ∗ π             \tag{R-Rule}\\
    \square ∗ [t\;\,{-}]π
                      &\quad\;≻\quad  \square ∗ π       \tag{Erase 1}\\
    \square ∗ [{-}\;v]π
                      &\quad\;≻\quad  \square ∗ π       \tag{Erase 2}\\
    [\square\,| (C_i[x_i] → t_i)_{i∈I}]
                      &\quad\;≻\quad  \square ∗ π       \tag{Erase 3}\\
    \square.l_k ∗ π   &\quad\;≻\quad  \square ∗ π       \tag{Erase 4}
  \end{align*}
  \caption{Base reduction relation for the abstract machine.}
  \label{fig:base_red}
\end{figure}

Processes form the internal state of our abstract machine. They are to be
thought of as a term put in some evaluation context represented using a
stack. Intuitively, the stack $π$ in the process $t∗π$ contains the
arguments to be fed to $t$. Since we are in call-by-value the stack also
handles the storing of functions while their arguments are being evaluated.
The operational semantics of our language is given by a relation $(≻)$
over processes.
\begin{definition}[base reduction relation]
  The relation $(≻) ⊆ (Λ×Π) × (Λ×Π)$ is defined as the smallest relation
  satisfying the reduction rules of Figure~\ref{fig:base_red}. We denote
  $(≻^{+})$ its transitive closure, $(≻^{*})$ its reflexive-transitive
  closure and $(≻^k)$ its $k$-fold application.
\end{definition}

The rules (Push), (Swap) and (Pop) are those that handle $β$-reduction. When
the abstract machine encounters an application, the term that is in function
position is stored on the stack to evaluate its argument first. Once the
argument has been completely computed, a value faces the stack containing the
function. The function can then be evaluated with the computed (value)
argument stored on the stack, ready to be consumed by the function as soon as
it evaluates to a $λ$-abstraction. A capture-avoiding substitution can then be
performed to effectively apply the argument to the function.

The (Save) and (Restore) rules handle the classical part of computation. When
a $μ$-abstraction is reached, the current stack is captured and substituted
for the corresponding $μ$-variable. Conversely, when a term of the form $[ξ]t$
is reached, the current stack is discarded and evaluation resumes with the
process $t ∗ ξ$. The rules (Find), (Match) and (Unfold) are provided to handle
record projection, case analysis and recursion respectively.

A rule is then provided for reducing processes of the form $R(v,t) ∗ π$ to
$t ∗ π$ when the value $v$ is a record. Note that if $v$ is not a record then
no reduction rule apply on processes of the form $R(v,t) ∗ π$. These facts
will be used to show that records, $λ$-abstractions and other forms of
values have a different computational behaviour in our abstract machine.
The last four rules are used to handle the special value $□$, which consumes
the surrounding part of its computational environment. This will be discussed
further when defining the semantical interpretation of our type system.
%
Finally, note that processes of the form $δ(v,w,t) ∗ π$ are left untouched
by the relation $(≻)$. They will however be given a reduction rule in the
following section.

\begin{theorem}[compatibility of reduction and substitution]\label{thm:redcompatall}%
  Let $ρ ∈ \mathcal{S}$ be a substitution and $p$, $q ∈ Λ×Π$ be two processes.
  If $p ≻ q$ (resp. $p ≻^{*} q$, resp. $p ≻^{+} q$, resp. $p ≻^k q$) then
  $pρ ≻ qρ$ (resp. $pρ ≻^{*} qρ$, resp $pρ ≻^{+} qρ$, resp. $pρ ≻^k qρ$).
\end{theorem}
\begin{proof}
  Immediate case analysis (and induction), all rules being local.
\end{proof}

\section{Classification of processes}

We are now going to give the vocabulary that will be used to describe specific
classes of processes. In particular we need to identify processes that are to
be considered as the evidence of a successful computation, and those that are
to be recognised as the expression of a failure (e.g., a crash).
\begin{definition}[classification of processes]
  A process $p ∈ Λ×Π$ is said to be:
  \begin{itemize}
    \item \emph{final} if $p = v∗ε$ for some $v ∈ Λ_{ι}$,
    \item \emph{$δ$-like} if $p = δ(v,w,t) ∗ π$ for some $v,w ∈ Λ_{ι}$,
      $t ∈ Λ$ and $π ∈ Π$,
    \item \emph{blocked} if there is no $q ∈ Λ×Π$ such that $p ≻ q$,
    \item \emph{stuck} if it is not final nor $δ$-like and if $pρ$ is
      blocked for all $ρ ∈ \mathcal{S}$,
    \item \emph{non-terminating} if there is an infinite sequence
      $(p_i)_{i∈\mathbb{N}}$ with $p_0 = p$ and $p_i ≻ p_{i+1}$.
  \end{itemize}
\end{definition}

When a process becomes stuck, non-terminating or $δ$-like during its
reduction, it will remain so forever. In particular, no substitution
will ever be able to turn it into a process that might lead to a
successful end of computation (i.e., reduce to a final process).
\begin{lemma}[stability under substitution]\label{lem:redstable}%
  Let $p ∈ Λ×Π$ be a process and $ρ ∈ \mathcal{S}$ be a substitution. If $p$
  is final (resp. $δ$-like, stuck, non-terminating), then so is $pρ$.
\end{lemma}
\begin{proof}
  If $p$ is final then $p = v∗ε$ for some $v ∈ Λ_{ι}$. Since $(v∗ε)ρ =
  vρ∗ερ = vρ∗ε$ the process $pρ$ is also final. If $p$ is $δ$-like then
  $p = δ(v,w,t)∗π$ for some $v$, $w ∈ Λ_{ι}$, $t ∈ Λ$ and $π ∈ Π$. Since
  $(δ(v,w,t)∗π)ρ = (δ(v,w,t))ρ ∗ πρ = δ(vρ,wρ,tρ) ∗ πρ$ the process $pρ$
  is also $δ$-like. If $p$ is stuck, then we suppose that there is a
  substitution $ρ_0 ∈ \mathcal{S}$ such that $(pρ)ρ_0$ is not blocked.
  This contradicts the fact that $p$ is stuck since $p(ρ_0 ∘ ρ) = (pρ)ρ_0$
  is not blocked, and hence $pρ$ must be stuck. Finally, if $p$ is
  non-terminating then we have a sequence $(p_i)_{i∈\mathbb{N}}$ such that
  $p_0 = p$ and $p_i ≻ p_{i+1}$ for all $i ∈ \mathbb{N}$. To show that
  $pρ$ is non-terminating we need to construct a sequence
  $(q_i)_{i∈\mathbb{N}}$ such that $q_0 = pρ$ and $q_i ≻ q_{i+1}$ for
  all $i ∈ \mathbb{N}$. We can take $q_i = p_iρ$ for all $i ∈ \mathbb{N}$.
  Indeed, we have $q_0 = pρ$ since $p₀ = p$ and for all $i ∈ \mathbb{N}$ we
  have $p_iρ ≻ p_{i+1}ρ$ by Lemma~\ref{thm:redcompatall} as $p_i ≻ p_{i+1}$.
\end{proof}

\begin{lemma}[characterization of stuck processes]\label{remark}%
  A process is stuck if and only if it is of one of the following forms,
  where $n$, $m$, $k ∈ \mathbb{N}$ and $I$, $J$, $K ⊆_{fin} \mathbb{N}$ such
  that $k ∉ K$.
  \begin{center}
  $
    \hspace{1.5cm}
    C_n[v].l_m ∗ π
    \hfill
    (λx.t).l_m ∗ π
    \hfill
    C_n[v] ∗ [{-}\;w]π
    \hfill
    \{(l_i = v_i)_{i∈I}\} ∗ [{-}\;v]π
    \hspace{1.5cm}
  $
  \\[2mm]
  $
    \hspace{1.5cm}
    [λx.t\,| (C_i[x_i] → t_i)_{i∈I}] ∗ π
    \hfill
    [\{(l_i = v_i)_{i∈I}\}\,| (C_j[x_j] → t_j)_{j∈J}] ∗ π
    \hspace{1.5cm}
  $
  \\[2mm]
  $
    \hspace{1.5cm}
    [C_k[v]\,|(C_i[x_i] → t_i)_{i∈K}] ∗ π
    \hfill
    \{(l_i = v_i)_{i∈K}\}.l_k ∗ π
    \hspace{1.5cm}
  $
  \\[2mm]
  $
    \hspace{1.5cm}
    R(λx.t,u) ∗ π
    \hfill
    R(C_n[v],u) ∗ π
    \hfill
    R(□,u) ∗ π
    \hspace{1.5cm}
  $
  \end{center}
  \vspace{-3mm}%hack
\end{lemma}
\begin{proof}
  Using a simple case analysis we first rule out the thirteen forms of
  processes that immediately reduce using $(≻)$. As stuck processes are
  neither final nor $δ$-like, we can again rule out two forms of processes.
  We are now left with eighteen forms of processes, among which seven
  are not stuck (see the proof of Lemma~\ref{lem:possibilities}). It is easy
  to see that the eleven remaining forms of processes are stuck. Indeed, given
  their structure no reduction rule will ever apply to them, even after a
  substitution.
\end{proof}
The proof of Lemma~\ref{remark} has been (partially) checked using OCaml's
exhaustivity checker for patterns. Indeed, the abstract syntax tree
corresponding to our language can be encoded into OCaml data types easily.
It is then possible to use pattern matching to enumerate possible forms of
processes in such a way that it is neither redundant nor incomplete (i.e.,
that the OCaml compiler does not complain with a warning). The OCaml source
file used for this purpose is available online.
\begin{center}
  \url{https://lepigre.fr/these/classification.ml}
\end{center}

\begin{lemma}[characterization of blocked processes]\label{lem:possibilities}%
  A blocked process $p ∈ Λ×Π$ is either stuck, final, $δ$-like, or of one of
  the following seven forms.
  \begin{center}
  $
    \hspace{1.5cm}
    x.l_k ∗ π
    \hfill
    x ∗ [{-}\;v]π
    \hfill
    [x\,| (C_i[x_i] → t_i)_{i∈I}] ∗ π
    \hspace{1.5cm}
  $
  \\[2mm]
  $
    \hspace{1.5cm}
    x ∗ [t\;\,{-}]π
    \hfill
    R(x,u) ∗ π
    \hfill
    a ∗ π
    \hfill
    v ∗ α
    \hspace{1.5cm}
  $
  \end{center}
\end{lemma}
\begin{proof}
  As for Lemma~\ref{remark}, we can rule out the thirteen forms of
  processes that immediately reduce using $(≻)$, final processes and
  $δ$-like processes. After ruling out the eleven forms of stuck processes
  of Lemma~\ref{remark} we are left with seven forms of processes. It
  remains to show that they are not stuck by finding a substitution
  $ρ ∈ \mathcal{S}$ unlocking their reduction.

  For processes of the first four forms we can take $ρ = [x := □]$
  since we have $□.l_k ∗ πρ ≻ □ ∗ πρ$, $□ ∗ [{-}\;vρ]πρ ≻ □ ∗ πρ$,
  $[□\,| (C_i[x_i] → t_iρ)_{i∈I}] ∗ π ≻ □ ∗ πρ$ and
  $□ ∗ [tρ\;\,{-}]πρ ≻ □ ∗ πρ$ respectively.
  %
  For a process of the form $R(x,u) ∗ π$ we can take $ρ = [x := \{\}]$ as
  $R(\{\},uρ) ∗ πρ ≻ uρ ∗ πρ$.
  %
  For a process of the form $a ∗ π$ we can take $ρ=[a := \{l_k=\{\}\}.l_k]$
  as $\{l_k = \{\}\}.l_k ∗ πρ ≻ \{\} ∗ πρ$.
  %
  Finally, for a process of the form $v ∗ α$ we can take
  $ρ = [α := [\{\}\;{-}]ε]$ as we will have
  $vρ ∗ [\{\}\;{-}]ε ≻ \{\} ∗ [{-}\;vρ]ε$ if $v ≠ □$ and
  $vρ ∗ [\{\}\;{-}]ε ≻ □ ∗ ε$ otherwise.
\end{proof}

\section{Stratified equivalence and reduction relations}

Let us first consider generic definitions related to reduction relations. In
particular, we will give a generic way to derive the observational equivalence
relation induced by a reduction relation.
\begin{definition}[convergence/divergence]
  Let $R ⊆ (Λ×Π) × (Λ×Π)$ be a reduction relation such that for every final
  process $p ∈ Λ×Π$, there is no $q ∈ Λ×Π$ such that $p \mathrel{R} q$. We
  say that a process $p ∈ Λ×Π$ converges for the relation $R$,
  and we write $p {⇓}_R$, if there is a final process $q ∈ Λ×Π$
  such that $p \mathrel{R^{∗}} q$. If $p$ does not converge we say that it
  diverges (for the relation $R$) and we write $p {⇑}_R$.
\end{definition}
\begin{definition}[observational equivalence]
  Let $R ⊆ (Λ×Π) × (Λ×Π)$ be a reduction relation. The observational
  equivalence relation induced by $R$ is denoted $(≡_R)$ and defined as
  follows.
  \begin{center}
  $
    (≡_R) = \{(t,u) \st ∀π∈Π, ∀ρ∈\mathcal{S}, tρ∗π {⇓}_R ⇔ uρ∗π {⇓}_R\}
  $
  \end{center}
  \vspace{-3mm}%hack
\end{definition}
\begin{lemma}\label{obseqRequiv}%
  For all $R ⊆ (Λ×Π) × (Λ×Π)$, $(≡_R)$ is an equivalence relation.
\end{lemma}
\begin{proof}
  Immediate by definition.
\end{proof}

The idea now is to extend our reduction relation $(≻)$ with a new,
surprising reduction rule. It will reduce processes of the form
$δ(v,w,t)∗π$ to $t∗π$ in the case where $v \not\equiv w$ for some
equivalence relation $(≡)$, and remain stuck otherwise. However, by
adding such a reduction rule, it is not possible to take $(≡) = (≡_{≻})$.
Indeed, this would make the definitions of reduction and equivalence
circular. Consequently, we need to be very careful so that everything
remains well-defined. We will rely on a stratified construction of both
the reduction relation and the equivalence relation.
\begin{definition}[stratification]
  For all $i ∈ \mathbb{N}$ we define two relations $(↠_i)$ and $(≡_i)$ as
  follows.
  \begin{center}
  $
    \hspace{1.5cm}
    (↠_i) = (≻) ∪ \{(δ(v,w,t)∗π, t∗π) \st ∃j<i, v \not\equiv_j w\}
    \hfill
    (≡_i) = \cap_{j ≤ i} (≡_{↠_j})
    %\{(t,u) \st ∀j≤i, ∀π∈Π, ∀ρ∈\mathcal{S}, tρ∗π {⇓}_j ⇔ uρ∗π {⇓}_j\}
    \hspace{1.5cm}
  $
  \end{center}
  The relations are well-defined: although $(≡_i)$ depends on $(↠_j)$ for all
  $j ≤ i$, $(↠_i)$ only depends on $(≡_j)$ for all $j < i$. Note that
  $(↠_0) = ({≻})$ as there is no $j<0$, and hence $(≡_0) = (≡_{≻})$.
\end{definition}

\begin{lemma}\label{isequiv}%
  For all $i ∈ \mathbb{N}$, $(≡_i)$ is an equivalence relation.
\end{lemma}
\begin{proof}
  Immediate by Lemma~\ref{obseqRequiv}, an intersection of
  equivalence relations being one itself.
\end{proof}

We can then define our actual reduction relation and equivalence relation
as a union and an intersection over the previously defined relations.
\begin{definition}[reduction and equivalence]
  The relations $(↠)$ and $(≡)$ are defined as follows.
  \begin{center}
  $
    \hfill
    (↠) = \cup_{i∈\mathbb{N}} (↠_i)
    \hfill
    (≡) = \cap_{i∈\mathbb{N}} (≡_i)
    \hfill
  $
  \end{center}
  For convenience, we will generally use the following, equivalent
  formulations.

  \vspace{2mm}
  \noindent$\hspace{1.5cm} (≡) = \cap_{i∈\mathbb{N}} (≡_{↠_i})
    = \{(t,u) \st ∀i∈\mathbb{N}, ∀π∈Π,
    ∀ρ∈\mathcal{S}, tρ∗π {⇓}_i ⇔ uρ∗π {⇓}_i\}$

  \vspace{2mm}
  \noindent$\hspace{1.5cm} (\not\equiv) = \cup_{i∈\mathbb{N}} (\not\equiv_{↠_i})
    = \{(t,u), (u,t) \st ∃i∈\mathbb{N}, ∃π∈Π, ∃ρ∈\mathcal{S},
    tρ∗π {⇓}_i ∧ uρ∗π {⇑}_i\}$

  \vspace{2mm}
  \noindent$\hspace{1.5cm} (↠) = (≻) ∪ \{(δ(v,w,t)∗π, t∗π) \st v \not\equiv w\} $
\end{definition}
Note that the definition of $(↠)$ corresponds exactly to what we aimed
for: an extension of $(≻)$ with a reduction rule for $δ$-like terms
carrying two non-equivalent values.

\begin{remark}
  We have $(↠_i) ⊆ (↠_{i+1})$ and $(≡_{i+1}) ⊆ (≡_i)$. Consequently, the
  construction of $(↠_i)_{i∈\mathbb{N}}$ and $(≡_i)_{i∈\mathbb{N}}$ forms
  a fixpoint at the ordinal $ω$. Surprisingly, this property will not be
  explicitly required in the following.
\end{remark}
\begin{lemma}
  $(≡)$ is an equivalence relation.
\end{lemma}
\begin{proof}
  Immediate by Lemma~\ref{isequiv}, an intersection of equivalence relations
  being one itself.
\end{proof}

\section{Congruence properties for equivalence}

We are now going to establish important properties of our equivalence
relation $(≡)$. In particular, we will show that it behaves in the expected
way with respect to substitution. For instance, we will show that arbitrary
substitutions preserve equivalence, and that the substitution of equivalent
values in a term yields equivalent terms.

\begin{theorem}[substitutivity]\label{fullsubstequiv}%
  Let $t$, $u ∈ Λ$ be two terms and $ρ ∈ \mathcal{S}$ be a substitution. If
  $t ≡ u$ then $tρ ≡ uρ$.
\end{theorem}
\begin{proof}
  Let us take $i_0 ∈ \mathbb{N}$, $ρ_0 ∈ \mathcal{S}$ and $π_0 ∈ Π$, and prove
  ${{(tρ)ρ_0 ∗ π_0} {⇓}_{i_0}} ⇔ {{(uρ)ρ_0 ∗ π_0} {⇓}_{i_0}}$, which rewrites
  as ${{t(ρ_0 ∘ ρ) ∗ π_0} {⇓}_{i_0}} ⇔ {{u(ρ_0 ∘ ρ) ∗ π_0} {⇓}_{i_0}}$. We can
  thus conclude by definition of $t ≡ u$ using $i_0$, the substitution
  $ρ_0 ∘ ρ$ and the stack $π_0$.
\end{proof}

\begin{theorem}[extensionality for values]\label{thm:extval}%
  Let $v_1$, $v_1 ∈ Λ_ι$ be values, $t ∈ Λ$ be a term and $x ∈ \cal{V}_ι$
  be a $λ$-variable. If we have $v_1 ≡ v_2$ then we also have
  ${t[x := v_1]} ≡ {t[x := v_2]}$.
\end{theorem}
\begin{proof}
  We are going to prove the contrapositive so we suppose ${t[x := v_1]}
  \not\equiv {t[x := v_2]}$ and we show $v_1 \not\equiv v_2$. Let us first
  assume that neither $v_1$ nor $v_2$ is equal to $□$ or to a $λ$-variables.
  By definition, we know that there $i∈\mathbb{N}$, $π∈Π$ and $ρ∈\mathcal{S}$
  such that ${(t[x := v_1])ρ ∗ π} {⇓}_i$ and ${(t[x := v_2])ρ ∗ π} {⇑}_i$ (up
  to symmetry). As $x$ is bound we can rename it in such a way that
  ${(t[x := v_1])ρ} = {tρ[x := v_1ρ]}$ and that ${(t[x := v_2])ρ} =
  {tρ[x := v_2ρ]}$. To finish the proof, we need to find $i_0 ∈ \mathbb{N}$,
  $π_0 ∈ Π$ and $ρ_0 ∈ \mathcal{S}$ such that ${v_1ρ_0 ∗ π_0} {⇓}_{i_0}$
  and ${v_2ρ_0 ∗ π_0} {⇑}_{i_0}$ (up to symmetry). We can take $i_0 = i$,
  $π_0 = {[(λx.tρ)\;\,{-}]π}$ and $ρ_0 = ρ$ since ${v_1ρ ∗ [(λx.tρ)\;\,{-}]π}
  ↠_i {tρ[x := v_1ρ] ∗ π} {⇓}_i$ and ${v_2ρ ∗ [(λx.tρ)\;\,{-}]π} ↠_i
  {tρ[x := v_2ρ] ∗ π} {⇑}_i$. Note that here, it is essential that $v_1ρ$ and
  $v_2ρ$ are not equal to $□$ or to some $λ$-variable, as otherwise the first
  reduction steps could not be taken.

  It remains to show that $v_1 \not\equiv v_2$ when either $v_1$ or $v_2$
  (or both) is equal to $□$ or to a $λ$-variable. First, we can assume
  $v_1 ≠ v_2$ as otherwise we would immediately get a contradiction with
  ${t[x := v_1]} \not\equiv {t[x := v_2]}$ by reflexivity of $({≡})$. As a
  consequence, it is not possible that $v_1 = v_2 = □$ or that $v_1 = v_2 =
  x ∈ \mathcal{V}_{ι}$.
  %
  Let us consider the case where $v_1 = x ∈ \mathcal{V}_{ι}$ and $v_2 = □$,
  or symmetrically, $v_1 = □$ and $v_2 = x ∈ \mathcal{V}_{ι}$. To tell these
  values apart we can show $v_1 \not\equiv_0 v_2$ using the substitution
  $ρ = [x := \{\}]$ and the stack $π = [{-}\;\,\{\}]ε$. Indeed, we have
  ${□ρ ∗ π} = {□ ∗ [{-}\;\,\{\}]ε} ↠_0 {□ ∗ ε}  {⇓}_0$ and ${xρ ∗ π} =
  {\{\} ∗ [{-}\;\,\{\}]ε} {⇑}_0$.
  %
  We now consider the case where $v_1 = x ∈ \mathcal{V}_{ι}$ and
  $v_2 = y ∈ \mathcal{V}_{ι}$, with $x ≠ y$. To tell these values apart we
  show $v_1 \not\equiv_0 v_2$ using the substitution $ρ = [x := \{\}][y := □]$
  and (again) the stack $π = [{-}\;\,\{\}]ε$. Indeed, ${v_1ρ ∗ π} =
  {\{\} ∗ [{-}\;\,\{\}]ε} {⇑}_0$ and
  ${v_2ρ ∗ π} = {□ ∗ [{-}\;\,\{\}]ε} {⇓}_0$.

  Remain four cases, $v_1 = x  ∈ \mathcal{V}_{ι}$ or $v_1 = □$ and $v_2$ is
  neither a $λ$-variable nor box and the two symmetric cases. When $v_1 =  □$,
  we can take $ρ$ the identity substitution and $π = [{-}\;\,\{\}]ε$ as we have
  ${v_1ρ ∗ π} = {□ ∗ [{-}\;\,\{\}]ε} {⇓}_0$ and
    ${v_2ρ ∗ π} = {v_2 ∗ [{-}\;\,\{\}]ε} {⇑}_0$. When $v_1 = x$, we take
    $ρ = [x := □]$ and the same stack to reach the same conclusion.
\end{proof}

A result analogous to Theorem~\ref{thm:extval} also holds for substituting
equivalent terms to a term variable. However, it is harder to prove, and it
requires a proof by induction on the reduction levels. The core of the proof
happens in that of Lemma~\ref{afullextlem}, which itself makes use of the
following lemma.
\begin{lemma}\label{lem:aposs}%
  Let $p ∈ Λ×Π$ be a process, $t ∈ Λ$ be a term, and $a ∈ \mathcal{V}_τ$ be a
  term variable. If ${p[a := t]} {⇓}_k$ for some $k ∈ \mathbb{N}$ then there
  is a blocked process $q ∈ Λ×Π$ such that $p ≻^{∗} q$ and either:
  \begin{itemize}
    \item $q = {v ∗ ε}$ for some value $v ∈ Λ_{ι}$,
    \item $q = {a ∗ π}$ for some stack $π ∈ Π$,
    \item $k ≠ 0$ and $q = {δ(v,w,u) ∗ π}$ for some values $v$, $w ∈ Λ_{ι}$,
      term $t ∈ Λ$ and stack $π ∈ Π$. Moreover, in this case, we also know
      that ${v[a := t]} \not\equiv_j {w[a := t]}$ for some $j < k$.
  \end{itemize}
\end{lemma}
\begin{proof}
  If $p$ is non-terminating then so is $p[a := t]$ according to
  Lemma~\ref{lem:redstable}. Since $({≻}) ⊆ ({↠}_k)$ this contradicts
  ${{p[a := t]}} {⇓}_k$, and thus there must be a blocked process $q ∈ Λ×Π$
  such that $p ≻^{∗} q$. Using Theorem~\ref{thm:redcompatall} we obtain
  ${p[a := t]} ≻^{∗} {q[a := t]}$, which tells us that ${q[a := t]} {⇓}_k$.
  This means that $q$ cannot be stuck, as otherwise $q[a := t]$ would also be
  stuck by Lemma~\ref{lem:redstable}, and that would contradict
  ${q[a := t]} {⇓}_k$.
  %
  Let us now suppose that $q = {δ(v,w,u) ∗ π}$ for some $v$, $w ∈ Λ_ι$,
  $t ∈ Λ$, and $π ∈ Π$. Since ${δ(vρ,wρ,uρ) ∗ π} {⇓}_k$, there must be $j < k$
  (and thus $k ≠ 0$) such that ${vρ} \not\equiv_j {wρ}$, otherwise we would
  obtain a contradiction.
  %
  By Lemma~\ref{lem:possibilities}, it remains to rule out the following forms
  for $q$, where $b ≠ a$.
  \begin{center}
  $
    \hspace{1.5cm}
    x.l_k ∗ π
    \hfill
    x ∗ [{-}\;v]π
    \hfill
    [x\,| (C_i[x_i] → t_i)_{i∈I}] ∗ π
    \hspace{1.5cm}
  $
  \\[2mm]
  $
    \hspace{1.5cm}
    x ∗ [t\;\,{-}]π
    \hfill
    R(x,u) ∗ π
    \hfill
    b ∗ π
    \hfill
    v ∗ α
    \hspace{1.5cm}
  $
  \end{center}
  It is however easy to see that if $q$ was of one of these forms, then
  $q[a := t]$ would still be blocked, and that would again contradict
  ${q[a := t]} {⇓}_k$.
\end{proof}
\begin{lemma}\label{afullextlem}%
  Let $u_1$, $u_2$, $t ∈ Λ$ be three terms, $a ∈ \mathcal{V}_τ$ be a term
  variable, and $k ∈ \mathbb{N}$ be a natural number. If $u_1 ≡_k u_2$ then
  ${t[a := u_1]} ≡_k {t[a := u_2]}$.
\end{lemma}
\begin{proof}
  We do a proof by induction on the equivalence level, so we take $k_0 ∈
  \mathbb{N}$ and we assume that the property holds for all $k < k_0$. We
  then suppose that $u_1 ≡_{k_0} u_2$ and show ${t[a := u_1]} ≡_{k_0}
  {t[a := u_2]}$. By definition, we need to take $π ∈ Π$, $ρ ∈ \mathcal{S}$
  and show ${{(t[a := u_1])ρ∗π} {⇓}_{k_0}} ⇔ {{(t[a := u_2])ρ∗π} {⇓}_{k_0}}$.
  Since $a$ is bound we are free to rename it, hence we may assume
  ${(t[a := u_1])ρ} = {tρ[a := u_1ρ]}$, ${(t[a := u_2])ρ} = {tρ[a := u_2ρ]}$
  and $a ∉ FV_τ(π) ∪ FV_τ(t_1) ∪ FV_τ(t_2)$. By symmetry, we can thus
  suppose ${tρ[a := u_1ρ] ∗ π} {⇓}_{k_0}$ and show
  ${tρ[a := u_2ρ] ∗ π} {⇓}_{k_0}$.

  We will now build a sequence $(t_i,π_i,l_i)_{i ∈ I}$ defined in such
  a way that for every natural number $i ∈ I$ we have ${tρ[a := u_1ρ] ∗ π}
  ↠_{k_0}^{∗} {t_i[a := u_1ρ] ∗ π_i[a := u_1ρ]}$ in $l_i$ steps, and
  $t_i ∗ p_i$ is blocked. We also require $(l_i)_{i ∈ I}$ to be increasing,
  and to have a strictly increasing subsequence. Under this condition the
  sequence must be finite, as if it was infinite then $tρ[a := u_1ρ]∗π$
  would be non-terminating, and this would contradict ${tρ[a := u_1ρ]∗π}
  {⇓}_{k_0}$. As a consequence, our sequence has a finite number $n+1$ of
  elements (for some $n ∈ \mathbb{N}$), and we can denote it
  $(t_i,π_i,l_i)_{i ≤ n}$. To show that $(l_i)_{i ≤ n}$ has a strictly
  increasing subsequence, we will ensure that it does not have three equal
  consecutive values.

  To define $(t_0,π_0,l_0)$ we look at the reduction of $tρ ∗ π$. As
  ${(tρ∗π)[a := u_1ρ]} = {tρ[a := u_1]∗π} {⇓}_{k_0}$, we can apply
  Lemma~\ref{lem:aposs} to obtain a blocked process $p$ such that
  ${tρ ∗ π} ≻^j p$. We then take ${t_0 ∗ π_0} = p$ and $l_0 = j$.
  According to Theorem~\ref{thm:redcompatall} we have
  ${(tρ ∗ π)[a := u_1ρ]} ≻^j {t_0[a := u_1ρ] ∗ π_0[a := u_1ρ]}$. As a
  consequence, we know that ${(tρ ∗ π)[a := u_1ρ]} ↠_{k_0}^{∗}
  {t_0[a := u_1ρ] ∗ π_0[a := u_1ρ]}$ in $l_0 = j$ steps.
  %
  To define $(t_{i+1},π_{i+1},l_{i+1})$, we consider the (already constructed)
  blocked  process ${t_i ∗ π_i}$. By construction, we know that
  ${tρ[a := u_1ρ] ∗ π} ↠_{k_0}^{∗} {t_i[a := u_1ρ] ∗ π_i[a := u_1ρ]}$ in $l_i$
  steps. Hence, by Lemma~\ref{lem:aposs}, we know that $t_i ∗ π_i$ can only be
  of three different shapes.
  \begin{itemize}
    \item If ${t_i ∗ π_i} = {v ∗ ε}$ for some $v ∈ Λ_ι$ then the sequence ends
      with $n = i$.
    \item If $t_i = a$ then we consider the process ${t_i[a := u_1ρ] ∗ π_i}$.
      By construction, we know that we have
      ${(t_i[a := u_1ρ] ∗ π_i)[a := u_1ρ]} {⇓}_{k_0}$,
      and Lemma~\ref{lem:aposs} gives us a blocked process $p$ such that
      ${t_i[a := u_1ρ] ∗ π_i} ≻^j p$. By Theorem~\ref{thm:redcompatall}
      ${(t_i[a := u_1ρ] ∗ π_i)[a := u_1ρ]} ≻^j {p[a := u_1ρ]}$, and hence
      ${t_i[a := u_1ρ] ∗ π_i[a := u_1ρ]} ↠_{k_0}^{∗} {p[a := u_1ρ]}$ in $j$
      steps. We then take as definition ${t_{i+1} ∗ π_{i+1}} = p$ and
      $l_{i+1} = l_i + j$.
      %
      Now, is it possible to have $j = 0$? This can only happen when
      ${t_i[a := u_1ρ] ∗ π_i}$ is of one of the three forms of
      Lemma~\ref{lem:aposs}. It cannot be of the form ${a ∗ π}$ as we assumed
      that $a$ does not appear in $u_1ρ$. If it is of the form $v ∗ ε$, then
      we reached the end of the sequence with $i = n$ so there is no problem.
      We only have to be careful when ${t_i[a := u_1ρ]} = {δ(v,w,u)}$. In
      this case, we will make sure that we always have $l_{i+2} > l_{i+1}$
      (see the following case).
    \item If $t_i = {δ(v,w,u)}$ for some $v$, $w ∈ Λ_ι$ and $u ∈ Λ$, then we
      know ${v[a := u_1ρ]} \not\equiv_m {w[a := u_1ρ]}$ for some $m < k_0$
      (here $k_0 ≠ 0$, so this case is trivial in the base case of our
      induction). Hence, we have ${t_i[a := u_1ρ] ∗ π_i} = {δ(v[a := u_1ρ],
      w[a := u_1ρ], u[a := u_1ρ]) ∗ π_i} ↠_{k_0} {u[a := u_1ρ] ∗ π_i}$.
      Moreover, ${t_i[a := u_1ρ] ∗ π_i[a := u_1ρ]} ↠_{k_0}
      {u[a := u_1ρ] ∗ π_i[a := u_1ρ]}$ by definition of $({↠}_{k_0})$. Since
      we know that ${t[a := u_1ρ] ∗ π} ↠_{k_0}^{∗} {t_i[a := u_1ρ] ∗ π_i[a :=
      u_1ρ]}$ in $l_i$ steps, we get that ${t[a := u_1ρ] ∗ π} ↠_{k_0}^{∗}
      {u[a := u_1ρ] ∗ π_i[a := u_1ρ]}$ in $l_i+1$ steps. And moreover, we also
      know that ${(u[a := u_1ρ] ∗ π_i)[a := u_1ρ]} = {u[a := u_1ρ] ∗ π_i[a :=
      u_1ρ]} {⇓}_{k_0}$.
      %
      We now consider the reduction of the process ${u[a := u_1ρ]∗π_i}$.
      According to Lemma~\ref{lem:aposs} there is a blocked process $p$ such
      that ${u[a := u_1ρ]∗π_i} ≻^j p$. Using Theorem~\ref{thm:redcompatall} we
      obtain ${u[a := u_1ρ] ∗ π_i[a := u_1ρ]} ≻^j {p[a := u_1ρ]}$ from which
      we deduce that ${u[a := u_1ρ] ∗ π_i[a := u_1ρ]} ↠_{k_0}^{∗} {p[a :=
      u_1ρ]}$ in $j$ steps. We then define ${t_{i+1} ∗ π_{i+1}} = p$ and
      $l_{i+1} = l_i + j + 1$. Note the that we indeed have $l_{i+1} > l_i$.
  \end{itemize}
  Intuitively, the sequence $(t_i,π_i,l_i)_{i ≤ n}$ mimics the reduction of
  the process $t[a := u_1ρ] ∗ π$, while making explicit every substitution of
  $a$ and every reduction of a $δ$-like state.

  To end the proof, we will show that for all $i ≤ n$ we have
  ${t_i[a := u_2ρ] ∗ π_i[a := u_2ρ]} {⇓}_{k_0}$. For $i = 0$, this will give
  us ${t[a := u_2ρ]∗π} {⇓}_{k_0}$, which is the expected result. As by
  construction ${t_n ∗ π_n} = {v ∗ ε}$, we have ${t_n[a := u_2ρ] ∗ π_n[a :=
  u_2ρ]} = {v[a := u_2ρ]∗ε}$, and hence ${t_n[a := u_2ρ] ∗ π_n[a := u_2ρ]}
  {⇓}_{k_0}$.
  %
  We assume ${t_{i+1}[a := u_2ρ] ∗ π_{i+1}[a := u_2ρ]} {⇓}_{k_0}$ for
  some $0 ≤ i < n$, and show ${t_i[a := u_1ρ] ∗ π_i[a := u_2ρ]} {⇓}_{k_0}$.
  By construction, $t_i ∗ π_i$ can be of two shapes, since only $t_n ∗ π_n$
  can be of the form $v ∗ ε$.
  \begin{itemize}
    \item If $t_i = a$ then we know that ${u_1ρ ∗ π_i} ↠_{k_0}^{∗}
      {t_{i+1} ∗ π_{i+1}}$. As a consequence, Theorem~\ref{thm:redcompatall}
      gives us ${u_1ρ ∗ π_i[a := u_2ρ]} ↠_{k_0}^{∗} {t_{i+1}[a := u_2ρ] ∗
      π_i[a := u_2ρ]}$, and we then get ${u_1ρ ∗ π_i[a := u_2ρ]} {⇓}_{k_0}$
      by induction hypothesis. Since $u_1 ≡_{k_0} u_2$, this implies
      ${u_2ρ ∗ π_i[a := u_2ρ]} = {(t_i ∗ π_i)[a := u_2ρ]} {⇓}_{k_0}$.

    \item If $t_i = δ(v,w,u)$ then we have ${u ∗ π_i} ↠_{k_0}^{∗} {t_{i+1} ∗
      π_{i+1}}$. As a consequence, Theorem~\ref{thm:redcompatall} gives us
      ${u[a := u_2ρ] ∗ π_i[a := u_2ρ]} ↠_{k_0}^{∗} {t_{i+1}[a := u_2ρ] ∗
      π_{i+1}[a := u_2ρ]}$. Using the induction hypothesis we obtain
      ${u[a := u_2ρ] ∗ π_i[a := u_2ρ]} {⇓}_{k_0}$. To conclude the proof, it
      remains to show that ${δ(v[a := u_2ρ],w[a := u_2ρ], u[a := u_2ρ]) ∗
      π_i[a := u_2ρ]} ↠_{k_0}^{∗} {u[a := u_2ρ] ∗ π_i[a := u_2ρ]}$.
      %
      We need to find $j < k_0$ such that ${v[a := u_2ρ]} \not\equiv_j
      {w[a := u_2ρ]}$. By construction, there is $m < k_0$ such that
      ${v[a := u_1ρ]} \not\equiv_m {w[a := u_1ρ]}$, and we will show
      ${v[a := u_2ρ]} \not\equiv_m {w[a := u_2ρ]}$. Using the global
      induction hypothesis twice, we obtain that ${v[a := u_1ρ]} ≡_m
      {v[a := u_2ρ]}$ and that ${w[a := u_1ρ]} ≡_m {w[a := u_2ρ]}$.
      Now if we suppose ${v[a := u_2ρ]} ≡_m {w[a := u_2ρ]}$ then we
      have ${v[a := u_1ρ]} ≡_m {v[a := u_2ρ]} ≡_m {w[a := u_2ρ]} ≡_m
      {w[a := u_1ρ]}$. As this contradicts the fact that
      ${v[a := u_1ρ]} \not\equiv_m {w[a := u_1ρ]}$, it must be that
      ${v[a := u_2ρ]} \not\equiv_m {w[a := u_2ρ]}$.
  \end{itemize}
\end{proof}
\begin{theorem}[extensionality for terms]\label{fullextterm}%
  Let $u_1$, $u_2$, $t ∈ Λ$ be three terms and $a ∈ \mathcal{V}_{τ}$ be a term
  variable. If we have $u_1 ≡ u_2$ then we also have
  ${t[a := u_1]} ≡ {t[a := u_2]}$.
\end{theorem}
\begin{proof}
  We suppose that $u_1 ≡ u_2$, which means that $u_1 ≡_i u_2$ for all
  $i ∈ \mathbb{N}$. We need to show that ${t[a := u_1]} ≡ {t[a := u_2]}$
  so we take $i_0 ∈ \mathbb{N}$ and we show ${t[a := u_1]} ≡_{i_0}
  {t[a := u_2]}$. By hypothesis we have $u_1 ≡_{i_0} u_2$ and hence we can
  conclude using Lemma~\ref{afullextlem}.
\end{proof}

\section{Partial axiomatization of equivalence}

The equivalence relation $(≡)$ plays an important rôle in the interpretation
of the types of the language. As a consequence, we will very often need to
perform equational reasoning, but the definition of the relation cannot be
directly implemented as a decision procedure. To solve this problem, we will
now extract a partial axiomatization of $(≡)$ from its definition. This
formulation will then allow us to easily implement a partial decision
procedure. A summary of the results of this section is displayed in
Figure~\ref{fig:axiomatization_equiv}.

\begin{figure}
  \centering
  Reduction steps as equivalences
  %
  \begin{align*}
    % CBV β-reduction
    (λx.t)\;v &\;≡\; t[x := v]
    \tag{if $x ∈ \mathcal{V}_{ι}$, $t ∈ Λ$, and $v ∈ Λ_{ι} \setminus
    (\mathcal{V}_{ι} ∪ \{□\})$}\\
    % Record projection
    \{(l_i = v_i)_{i∈I}\}.l_k &\;≡\; v_k
    \tag{if $k ∈ I ⊆_\text{fin} \mathbb{N}$ and $∀i∈I, v_i ∈ Λ_{ι}$}\\
    % Pattern-matching
    [C_k[v]\,| (C_i[x_i] → t_i)_{i∈I}] &\;≡\; t_k[x_k := v]
    \tag{if $k ∈ I ⊆_\text{fin} \mathbb{N}$, $v ∈ Λ_{ι}$,
    and $∀i∈I, (x_i, t_i) ∈ \mathcal{V}_{ι} × Λ$}\\
    % Fixpoint step
    φa.v &\;≡\; v[a := φa.v]
    \tag{if $a ∈ \mathcal{V}_{τ}$, and $v ∈ Λ_{ι}$}\\
    % Application with box
    t\;□ &\;≡\; □\;v \;≡\;□
    \tag{if $t ∈ Λ$, resp. $v ∈ Λ_{ι} \setminus \mathcal{V}_{ι}$}\\
    % Box projection
    □.l_k &\;≡\; □
    \tag{if $k ∈ \mathbb{N}$}\\
    % Box matching
    [□\,| (C_i[x_i] → t_i)_{i∈I}] &\;≡\; □
    \tag{if $I ⊆_\text{fin} \mathbb{N}$,
    and $∀i∈I, (x_i, t_i) ∈ \mathcal{V}_{ι} × Λ$}\\
  \end{align*}
  %
  Derived equivalences
  \begin{align*}
    % Elimination of μ-abstraction
    μα.t &\;≡\; t
    \tag{if $α ∈ \mathcal{F}_{σ}$, $t ∈ Λ$, and $α ∉ FV(t)$}\\
    % Restore after capture.
    μα.[α]t &\;≡\; μα.t
    \tag{if $α ∈ \mathcal{F}_σ$ and $t ∈ Λ$}\\
    % Contraction of μ-abstractions
    μα.μβ.t &\;≡\; μα.t[β := α]
    \tag{if $α, β ∈ \mathcal{F}_{σ}$, and $t ∈ Λ$}\\
    % Extraction of μ-abstractions on the right.
    t\;(μα.u) &\;≡\; μβ.t\;u[α := [t\;\,{-}]β]
    \tag{if $t, u ∈ Λ$, $α, β ∈ \mathcal{F}_{σ}$, and
    $β \notin FV_σ(t) ∪ FV_σ(u)$}\\
    % Extraction of μ-abstractions on the left.
    (μα.t)\;v &\;≡\; μβ.t[α := [{-}\;\,v]β]\;v
    \tag{\small if $t ∈ Λ$, $α, β ∈ \mathcal{F}_{σ}$, $v ∈ Λ_{ι} \setminus
    (\mathcal{V}_{ι} ∪ \{□\})$ and $β \notin FV_σ(t) ∪ FV_σ(v)$}\\
    % Erasing an argument.
    ([π]t)\; v &\;≡\; [π]t
    \tag{if $t ∈ Λ$, $π ∈ Π$ and $v ∈ Λ_{ι} \setminus
    (\mathcal{V}_{ι} ∪ \{□\})$}\\
    % Erasing a function.
    t\;([π]u) &\;≡\; [π]u
    \tag{if $t, u ∈ Λ$ and $π ∈ Π$}\\
    % Erasing a stack.
    [π₁][π₂]t &\;≡\; [π₂]t
    \tag{if $t ∈ Λ$ and $π₁, π₂ ∈ Π$}
  \end{align*}
  %
  Distinguishable terms (inequivalences)
  \begin{align*}
    % Different constructors.
    C_n[v] &\;\not\equiv\; C_m[w]
    \tag{if $v, w ∈ Λ_{ι}$ and $m, n ∈ \mathbb{N}$ with $n ≠ m$}\\
    % Different fields.
    \{(l_i = v_i)_{i∈I_1}\} &\;\not\equiv\; \{(l_i = w_i)_{i∈I_2}\}
    \tag{\small if $I₁, I₂ ⊆_{\text{fin}} \mathbb{N}$, $∀i∈I₁, v_i ∈ Λ_{ι}$,
    and  $∀i∈I₂, w_i ∈ Λ_{ι}$ with $I₁ ≠ I₂$}\\
    % Constructor not record.
    C_n[v] &\;\not\equiv\; \{(l_i = v_i)_{i∈I}\}
    \tag{if $n ∈ \mathbb{N}$, $v ∈ Λ_{ι}$, $I ⊆_{\text{fin}} \mathbb{N}$ and
    $∀i∈I, v_i ∈ Λ_{ι}$}\\
    % Constructor not λ-abstraction.
    C_n[v] &\;\not\equiv\; λx.t
    \tag{if $n ∈ \mathbb{N}$, $v ∈ Λ_{ι}$, $x ∈ \mathcal{V}_{ι}$, and $t∈Λ$}\\
    % Record not λ-abstraction.
    \{(l_i = v_i)_{i∈I}\} &\;\not\equiv\; λx.t
    \tag{if $I ⊆_{\text{fin}} \mathbb{N}$, $∀i∈I, v_i ∈ Λ_{ι}$,
    $x ∈ \mathcal{V}_{ι}$, and $t∈Λ$}\\
    % Different constructor contents.
    C_n[v] &\;\not\equiv\; C_n[w]
    \tag{if $v, w ∈ Λ_{ι}$ and $n ∈ \mathbb{N}$ with $v \not\equiv w$}\\
    % Different field contents.
    \{(l_i = v_i)_{i∈I}\} &\;\not\equiv\; \{(l_i = w_i)_{i∈I}\}
    \tag{\small if $I ⊆_{\text{fin}} \mathbb{N}$, $∀i∈I₁, v_i, w_i ∈ Λ_{ι}$,
    and  $∃k∈I, v_i \not\equiv w_i$}\\
  \end{align*}
  %
  \caption{Axiomatization of observational equivalence}
  \label{fig:axiomatization_equiv}.
\end{figure}

To prove that our equivalence axioms are actually satisfied by the definition
of $(≡)$, we will rely on the following lemmas. They rely on very simple
observations relating $(≡)$ to the base reduction relation $(≻)$ and to its
induced equivalence relation $(≡_{≻})$.\footnote{Recall that the equivalence
relation $(≡_{≻})$ coincides with $(≡_{↠_0})$ since $(↠_0) = (≻)$.}
\begin{lemma}[equivalence compatible with base reduction]\label{lem:eqred}
  Let $t$, $u ∈ Λ$ be terms. If for all stack $π ∈ Π$ there is a process
  $p ∈ Λ×Π$ such that ${t ∗ π} ≻^{∗} p$ and ${u ∗ π} ≻^{∗} p$ then $t ≡ u$.
  As an immediate consequence, if for every stack $π ∈ Π$ we have
  ${t ∗ π} ≻^{∗} {u ∗ π}$ then $t ≡ u$.
\end{lemma}
\begin{proof}
  Let us take $i₀ ∈ \mathbb{N}$, $π₀ ∈ Π$ and $ρ₀ ∈ \mathcal{S}$, and show
  that ${tρ₀ ∗ π₀} {⇓}_{i₀}$ if and only if ${uρ₀ ∗ π₀ {⇓}_{i₀}}$. Since we
  have $(≻) ⊆ (↠_{i₀})$ then it is enough to exhibit a common reduct of
  $tρ₀ ∗ π₀$ and $uρ₀ ∗ π₀$.
  %
  Let us consider a renaming substitution $σ ∈ \mathcal{S}$ mapping the
  variables of $FV(π₀)$ to fresh variables. Note that $σ$ has an inverse
  $σ^{-1}$ defined as $σ(χ) = χ$ for all $χ∈\text{dom}(σ)$. As a consequence,
  we have ${tρ₀ ∗ π₀} = {(tρ₀ ∗ π₀σ)σ^{-1}} = {(t ∗ π₀σ)(ρ₀ \circ σ^{-1})}$,
  and similarly ${uρ₀ ∗ π₀} = {(u ∗ π₀σ)(ρ₀ \circ σ^{-1})}$.
  %
  By hypothesis, there is a process $p$ such that ${t ∗ π₀σ} ≻^{∗} p$ and
  ${u ∗ π₀σ} ≻^{∗} p$. We can thus use Theorem~\ref{thm:redcompatall} to
  obtain ${tρ₀∗π₀} = {(t ∗ π₀σ)(ρ₀ \circ σ^{-1})} ≻^{∗} p₀(ρ₀ \circ σ^{-1})$,
  and similarly ${uρ₀ ∗ π₀} ≻^{∗} p₀(ρ₀ \circ σ^{-1})$.
\end{proof}
\begin{lemma}[base inequivalence implies inequivalence]\label{lem:stcntrex}%
  We have $(≡) ⊆ (≡_{≻})$. In particular, if $t \not\equiv_{≻} u$ for some
  $t, u ∈ Λ$ then $t \not\equiv u$. As a consequence, to show $t \not\equiv u$
  it is enough to find $π∈Π$ and $ρ∈\mathcal{S}$ such that ${tρ ∗ π} {⇓}_{≻}$
  and ${uρ ∗ π} {⇑}_{≻}$. And in particular, it is also enough to find $π∈Π$
  such that ${t ∗ π} {⇓}_{≻}$ and ${u ∗ π} {⇑}_{≻}$.
\end{lemma}
\begin{proof}
  The inclusion immediately follows by definition of $(≡)$ since $(≡_{≻})$
  coincides with $(≡_{↠_0})$. The remaining of the lemma follows trivially
  from the definition of $(≡_{≻})$.
\end{proof}

% CBV β-reduction
\begin{theorem}[equivalence for $β$-reduction]
  For all $x ∈ \mathcal{V}_{ι}$, $t ∈ Λ$, and $v ∈ Λ_{ι} \setminus
  (\mathcal{V}_{ι} ∪ \{□\})$ we have $(λx.t)\;v ≡ t[x := v]$.
\end{theorem}
\begin{proof}
  We have ${(λx.t)\;v ∗ π} ≻ {v ∗ [(λx.t)\;\,{-}]π} ≻ {λx.t ∗ [{-}\;\,v]π} ≻
  t[x := v] ∗ π$ for all $π ∈ Π$, so we can conclude using Lemma~\ref{lem:eqred}.
  Note that it is required that $v ≠ □$, since otherwise the second reduction
  step could not be taken.
\end{proof}

% Record projection
\begin{theorem}[equivalence for record projection]
  For all $I ⊆_{\text{fin}} \mathbb{N}$, $k ∈ I$ and $(v_i)_{i∈I} ∈ Λ_{ι}^I$
  we have $\{(l_i = v_i)_{i∈I}\}.l_k ≡ v_k$.
\end{theorem}
\begin{proof}
  We have ${\{(l_i = v_i)_{i∈I}\}.l_k ∗ π} ≻ {v_k ∗ π}$ for all $π ∈ Π$ since
  $k ∈ I$. As a consequence, we can conclude using Lemma~\ref{lem:eqred}.
\end{proof}

% Pattern-matching
\begin{theorem}[equivalence for case analysis]
  For all $I ⊆_{\text{fin}} \mathbb{N}$, $k ∈ I$ and $(x_i, t_i)_{i ∈ I} ∈
  (\mathcal{V}_{ι} × Λ)^I$ we have  $[C_k[v]\,| (C_i[x_i] → t_i)_{i∈I}] ≡
  t_k[x_k := v]$.
\end{theorem}
\begin{proof}
  We have ${[C_k[v]\,| (C_i[x_i] → t_i)_{i∈I}] ∗ π} ≻ {t_k[x_k := v] ∗ π}$
  for all stack $π ∈ Π$ since $k ∈ I$, and we can hence conclude using
  Lemma~\ref{lem:eqred}.
\end{proof}

% Fixpoint step
\begin{theorem}[equivalence for fixpoint unrolling]
  For all $a ∈ \mathcal{V}_{τ}$ and $v ∈ Λ_{ι}$ we have
  ${φa.v} ≡ {v[a := φa.v]}$.
\end{theorem}
\begin{proof}
  We have ${φa.v ∗ π} ≻ {v[a := φa.v] ∗ π}$ for all $π ∈ Π$, hence we can
  conclude using Lemma~\ref{lem:eqred}.
\end{proof}

% Reduction steps involving the box.
\begin{theorem}[equivalences involving $□$]
  The following four equivalences are satisfied:
  \begin{enumerate}[topsep=0pt,itemsep=0ex]
    \item for all $t ∈ Λ$ we have $t\;□ ≡ □$,
    \item for all $v ∈ Λ_{ι} \setminus \mathcal{V}_{ι}$ we have $□\;v ≡ □$,
    \item for all $k ∈ \mathbb{N}$ we have $□.l_k ≡ □$, and
    \item for all $I ⊆_\text{fin} \mathbb{N}$ and
          $(x_i, t_i)_{i∈I} ∈ (\mathcal{V}_{ι} × Λ)^I$
          we have $[□\,| (C_i[x_i] → t_i)_{i∈I}] ≡ □$.
  \end{enumerate}
\end{theorem}
\begin{proof}
  In all cases, we can conclude by Lemma~\ref{lem:eqred} thanks to the fact
  that the following sequences of reductions hold for all $π ∈ Π$.
  %
  For (1) we have ${t\;□ ∗ π} ≻ {□ ∗ [t\;\,{-}]π} ≻ {□ ∗ π}$.
  %
  For (2) we have ${□\;v ∗ π} ≻ {v ∗ [□\;\,{-}]π} ≻ {□ ∗ π}$ if $v = □$, and
  ${□\;v ∗ π} ≻ {v ∗ [□\;\,{-}]π} ≻ {□ ∗ [{-}\;\,v]π} ≻ {□ ∗ π}$ otherwise.
  Note that if $v = x$ for some $x ∈ \mathcal{V}_{ι}$ then the equivalence is
  false due to the fact that the process ${x ∗ [□\;\,{-}]π}$ is blocked,
  while ${□ ∗ π}$ may not be.
  %
  For (3) we have have ${□.l_k ∗ π} ≻ {□ ∗ π}$,
  %
  and for (4) we have ${[□\,| (C_i[x_i] → t_i)_{i∈I}] ∗ π} ≻ {□ ∗ π}$.
\end{proof}

\begin{remark}
  Note that $R(\{(l_i = v_i)_{i∈I}\},t) ≡ t$ for all $I ⊆_\text{fin}
  \mathbb{N}$, $t ∈ Λ$, and $∀i∈I, v_i ∈ Λ_{ι}$. Indeed, we have
  ${R(\{(l_i = v_i)_{i∈I}\},t) ∗ π} ≻ {t ∗ π}$ for all $π ∈ Π$, so we
  can also use Lemma~\ref{lem:eqred}. However, this result is not
  useful since terms of the form $R({-},{-})$ are only included for a
  technical purpose. In particular, they will never be manipulated by
  the user of the system, and they are not even included in the
  implementation. Of course, such terms will still arise in the
  model when we quantify over all stacks. However, the syntactic procedure
  that is implemented will obviously not rely on an enumeration of all
  stacks, but rather on local reasoning using the axioms derived in this
  section.
\end{remark}
\begin{remark}
  It is not the case in general that $δ(v₁,v₂,t) ≡ t$ in the case where
  $v₁ \not\equiv v₂$. For example, we have ${δ(v₁,v₂,w) ∗ ε} {⇑}_0$ and
  ${w ∗ ε} {⇓}_0$, which imply $δ(v₁,v₂,w) \not\equiv w$. Like with terms
  of the form $R({-},{-})$, $δ$-like terms will never be manipulated by
  the user explicitly, so the fact that they behave in a weird way with
  respect to equivalence will not be a problem at all in practice.
\end{remark}

% Elimination of μ-abstraction
\begin{theorem}[equivalence with non-occuring $μ$-variable]
  For all $α ∈ \mathcal{F}_{σ}$ and $t ∈ Λ$, if $α ∉ FV(t)$ then $μα.t ≡ t$.
\end{theorem}
\begin{proof}
  We can conclude using Lemma~\ref{lem:eqred} since ${μα.t ∗ π} ≻ {t ∗ π}$ for
  all stack $π ∈ Π$.
\end{proof}

% Restore after capture.
\begin{theorem}[restore after capture]
  For all $α ∈ \mathcal{F}_σ$ and $t ∈ Λ$ we have $μα.[α]t ≡ μα.t$.
\end{theorem}
\begin{proof}
  Since ${μα.[α]t ∗ π} ≻ {([α]t)[α:=π] * π} = {[π]t[α := π] ∗ π} ≻
  {t[α := π] ∗ π}$ and ${μα.t ∗ π} ≻ {t[α := π] ∗ π}$ for all $π ∈ Π$, we
  can conclude using Lemma~\ref{lem:eqred}.
\end{proof}

\begin{remark}
  If we combine the previous two results, we can obtain the equivalence
  $μα.[α]t ≡ t$ for all $t ∈ Λ$ and $α ∈ \mathcal{V}_{σ}$ such that $α ∉
  FV_{σ}(t)$. This can be seen as a form of $η$-equivalence for
  for $μ$-abstractions, as remarked by Parigot~\cite{Parigot1992}.
\end{remark}

% Contraction of μ-abstractions
\begin{theorem}[contraction of $μ$-abstractions]
  For all $α, β ∈ \mathcal{F}_{σ}$ and $t ∈ Λ$ we have $μα.μβ.t ≡ μα.t[β := α]$.
\end{theorem}
\begin{proof}
  By definition, we have ${μα.μβ.t ∗ π} ≻ {μβ.t[α := π] ∗ π} ≻
  {(t[α := π])[β := π] ∗ π}$ and, similarly, ${μβ.t[α := β] ∗ π} ≻
  {(t[α := β])[β := π] ∗ π}$. To be able to conclude using
  Lemma~\ref{lem:eqred} we need to show that
  ${(t[α := π])[β := π]} = {(t[α := β])[β := π]}$. This is immediate since
  we may assume $β ∉ FV_{σ}(π)$ up to renaming.
\end{proof}

% Extraction of μ-abstractions on the right.
\begin{theorem}[extraction of a $μ$-abstraction from an argument]
  For all $t, u ∈ Λ$, $α, β ∈ \mathcal{F}_{σ}$, and $β \notin FV_σ(t) ∪
  FV_σ(u)$ we have $t\;(μα.u) ≡ μβ.t\;u[α := [t\;\,{-}]β]$.
\end{theorem}
\begin{proof}
  We have ${t\;(μα.u) ∗ π} ≻ {μα.u ∗ [t \;\,{-}]π} ≻ {u[α := [t \;\,{-}]\pi] ∗
  [t \;\,{-}]\pi}$ and ${μβ.t\;u[α := [t\;\,{-}]β] ∗ π} ≻  {t\;u[α :=
  [t\;\,{-}]\pi] ∗ π} ≻ {u[α := [t\;\,{-}]\pi]∗ [t\;\,{-}] π}$. Hence, we can
  conclude by lemma~\ref{lem:eqred}.
\end{proof}

% Extraction of μ-abstractions on the left.
\begin{theorem}[extraction of a $μ$-abstraction from a function]
  For all $t ∈ Λ$, $α, β ∈ \mathcal{F}_{σ}$, $v ∈ Λ_{ι} \setminus
  (\mathcal{V}_{ι} ∪ \{□\})$, if we have $β \notin FV_σ(t) ∪ FV_σ(v)$ then
  the equivalence $(μα.t)\;v ≡ μβ.t[α := [{-}\;\,v]β]\;v$ is satisfied.
\end{theorem}
\begin{proof}
  We have ${(μα.t)\;v ∗ π} ≻ {v ∗ [(μα.t) \;\,{-}]\pi} ≻ {μα.t∗[{-}\;\,v]\pi}
  ≻ {t[α:= [{-}\;\,v]\pi]∗[{-}\;\,v]\pi}$ and
  ${μβ.t[α := [{-}\;\,v]β]\;v∗ π}  ≻ {t[α := [{-}\;\,v]π]\;v∗ π}
  ≻ {v ∗ [t[α := [{-}\;\,v]π]\;,{-}] π} ≻
  {t[α := [{-}\;\,v]π]∗[{-}\;\,v]\pi}$. Hence, we can
  conclude by lemma~\ref{lem:eqred}.
\end{proof}

% Erasing an argument.
\begin{theorem}[erasure of an argument]
  For all $t ∈ Λ$, $π ∈ Π$ and $v ∈ Λ_{ι} \setminus (\mathcal{V}_{ι} ∪ \{□\})$
  we have $([π]t)\;v ≡ [π]t$.
\end{theorem}
\begin{proof}
  We have ${([π_0]t)\;v ∗ π} ≻ {v ∗ [[π_0]t\;,{-}] π} ≻ {[π_0]t ∗ [{-}\;\,v]\pi}
  ≻ {t ∗ π_0}$ and ${[π_0]t ∗  π} ≻ {t ∗ π_0}$. Hence, we can
  conclude by lemma~\ref{lem:eqred}.
\end{proof}

% Erasing a function.
\begin{theorem}[erasure of a function]
  For all $t, u ∈ Λ$ and $π ∈ Π$ we have $t\;([π]u) ≡ [π]u$.
\end{theorem}
\begin{proof}
  We have ${t\;([π_0]u)∗ π} ≻ {([π_0]u)∗ [t\;,{-}]π} ≻ {u ∗  π_0}$ and
  ${[π_0]u∗ π} ≻ {u ∗  π_0}$. Hence, we can
  conclude by lemma~\ref{lem:eqred}.
\end{proof}

% Erasing a stack.
\begin{theorem}[erasure of a stack]
  For all $t ∈ Λ$ and $π₁, π₂ ∈ Π$ we have $[π₁][π₂]t ≡ [π₂]t$.
\end{theorem}
\begin{proof}
  Since we have ${[π₁][π₂]t ∗ π} ≻ {[π₂]t ∗ π₁} ≻ {t ∗ π₂}$ and ${[π₂]t ∗ π} ≻
  {t ∗ π₂}$ for all $π ∈ Π$, we can conclude using Lemma~\ref{lem:eqred}.
\end{proof}

TODO inequivalence

\section{Canonical values}

The idea now is to characterise the equivalence classes of the different forms
of values. The results presented here will be required to show that the
semantics of our types is closed under equivalence. We will first start by
showing that $□$ is only equivalent to itself among all values. Similarly, it
is possible to show that $λ$-variables are only equivalent to themselves.
\begin{theorem}[canonicity for $□$]\label{thm:canonbox}%
  Let $v ∈ Λ_{ι}$ be a value. We have $□ ≡ v$ if and only if $v = □$.
\end{theorem}
\begin{proof}
  If $v = □$ then we immediately have $□ ≡ v$ by reflexivity. It remains to
  show that $□ \not\equiv v$ for every value $v ≠ □$. In the case where
  $v \notin \cal{V}_ι$ we can use Lemma~\ref{lem:stcntrex} with the stack
  $π = [\{\}\;\,{-}]ε$ as we have ${□ ∗ [\{\}\;\,{-}]ε} ≻ {□ ∗ ε} {⇓}_{≻}$
  and ${v ∗ [\{\}\;\,{-}]ε} ≻ {\{\} ∗ [{-}\;\,v]ε} {⇑}_{≻}$.
  %
  If $v = x ∈ \mathcal{V}_{ι}$ then we can use Lemma~\ref{lem:stcntrex}
  with $ρ = [x := \{\}]$ and $π = {[\{\}\;\,{-}]ε}$ since
  ${vρ ∗ π} = {\{\} ∗ [\{\}\;\,{-}]ε} ≻ {\{\} ∗ [{-}\;\,\{\}]ε} {⇑}_{≻}$ and
  ${□ρ ∗ [\{\}\;\,{-}]ε} = {□ ∗ [\{\}\;\,{-}]ε} ≻ {□ ∗ ε} {⇓}_{≻}$ as above.
\end{proof}
\begin{theorem}[canonicity for $λ$-variables]\label{thm:varequiv}%
  Let $x ∈ \mathcal{V}_{ι}$ be a $λ$-variable and $v ∈ Λ_{ι}$ be a value. We
  have $x ≡ v$ if and only if $v = x$.
\end{theorem}
\begin{proof}
  If $v = x$ then we have $v ≡ x$ by reflexivity. It remains to show that
  $x \not\equiv v$ for every value $v ≠ x$. In the case where $v = □$ we
  can conclude using Theorem~\ref{thm:canonbox}. If $v = y ∈ \mathcal{V}_{ι}$
  we can use Lemma~\ref{lem:stcntrex} with $ρ = {[x := \{\}][y := □]}$ and
  $π = {[\{\}\;\,{-}]ε}$. Indeed,
  ${xρ ∗ π} = {\{\} ∗ [\{\}\;\,{-}]ε} ≻ {\{\} ∗ [{-}\;\,\{\}]ε} {⇑}_{≻}$ and
  ${vρ ∗ π} = {□ ∗ [\{\}\;\,{-}]ε} ≻ {□ ∗ ε} {⇓}_{≻}$. Finally, if $v ≠ □$
  and $v \notin \mathcal{V}_{ι}$ then we can use Lemma~\ref{lem:stcntrex}
  with $ρ = [x := □]$ and $π = {[\{\}\;\,{-}]ε}$. Indeed,
  ${xρ ∗ π} = {□ ∗ [\{\}\;\,{-}]ε} ≻ {□ ∗ ε} {⇓}_{≻}$, and since $vρ$ cannot
  be $□$ or a $λ$-variable we have
  ${vρ ∗ π} = {vρ ∗ [\{\}\;\,{-}]ε} ≻ {\{\} ∗ [{-}\;\,vρ]ε} {⇑}_{≻}$.
\end{proof}

We will now characterise the values that are equivalent to a given variant,
and those that are equivalent to a given record. In both cases, the equivalent
values have the same structure and equivalent subvalues.
\begin{theorem}[canonicity for variants]\label{thm:canoncons}%
  Let $k ∈ \mathbb{N}$ be a natural number and $v$, $w₀ ∈ Λ_{ι}$ be values.
  The equivalence $C_k[v] ≡ w₀$ holds if and only if $w₀ = C_k[w]$ for some
  $w ∈ Λ_{ι}$ such that $w ≡ v$.
\end{theorem}
\begin{proof}
  Let us first assume $w₀ = C_k[w]$ for some $w ∈ Λ_{ι}$ such that $w ≡ v$.
  To show ${C_k[v]} ≡ {C_k[w]}$, we can simply use Theorem~\ref{thm:extval}
  with the term $t = {C_k[x]}$.
  %
  Let us now suppose that ${C_k[v]} ≡ w₀$, and show that $w₀$ is of the form
  $C_k[w]$ for some $w ∈ Λ_{ι}$ such that $w ≡ v$. We reason by case on the
  possible forms of the value $w₀$. Using Theorems~\ref{thm:varequiv},
  \ref{thm:canonbox}, \ref{thm:nequivlamcons}, \ref{thm:consneqrecord} and
  \ref{thm:consnequiv} we obtain $w₀ = C_k[w]$ for some $w ∈ Λ_{ι}$. Now, if
  $w \not\equiv v$ then we immediately obtain that $C_k[v] \not\equiv C_k[w]$
  using Theorem~\ref{thm:aux_cons}. As a consequence, it must be that $v ≡ w$.
\end{proof}
\begin{theorem}[canonicity for records]\label{thm:canonrecord}%
  Let $I ⊆_{\text{fin}} \mathbb{N}$ be a set of indices such that $v_i ∈
  Λ_{ι}$ for all $i∈I$, and let $w ∈ Λ_{ι}$ be a value. The equivalence
  $\{(l_i = v_i)_{i∈I}\} ≡ w$ holds if and only if we have $w = \{(l_i =
  w_i)_{i∈I}\}$ for some values $(w_i)_{i∈I} ∈ Λ_{ι}^I$ such that
  $w_i ≡ v_i$ for all $i∈I$.
\end{theorem}
\begin{proof}
  Let us first assume that we have $w = \{(l_i = w_i)_{i∈I}\}$ with $v_i ≡
  w_i$ for all $i∈I$ and show $\{(l_i = v_i)_{i∈I}\} ≡ \{(l_i = w_i)_{i∈I}\}$.
  Up to renaming, we may assume $I = \{i ∈ \mathbb{N} \st 1≤i≤n\}$ with $n =
  |I|$. For all $0 ≤ k ≤ n$ we define $R_k = \{(l_i = r_i)_{i∈I}\}$ where
  $r_i = w_i$ if $i < k$ and $r_i = v_i$ otherwise. We have $R₀ = \{(l_i =
  v_i)_{i∈I}\}$ and $R_n = \{(l_i = w_i)_{i∈I}\}$, so we need to show that
  $R₀ ≡ R_n$. We are going to prove that $R₀ ≡ R_k$ for all $0 ≤ k ≤ n$ by
  induction on $k$. When $k = 0$ this is immediate by reflexivity. Let us now
  suppose that $R₀ ≡ R_k$ for some $0 ≤ k < n$ and show that $R₀ ≡ R_{k+1}$.
  By transitivity, it is enough to show $R_k ≡ R_{k+1}$ which follows easily
  from Theorem~\ref{thm:extval} since we assumed $v_k ≡ w_k$.
  %
  Let us now suppose $\{(l_i = v_i)_{i∈I}\} ≡ w$ and show that $w$ is of the
  form $\{(l_i = w_i)_{i∈I}\}$ with $w_i ≡ v_i$ for all $i ∈ I$. Using
  Theorems~\ref{thm:varequiv}, \ref{thm:canonbox}, \ref{thm:nequivlamreco},
  \ref{thm:consneqrec} and \ref{thm:reconequiv} we obtain $w = \{(l_i =
  w_i)_{i∈I}\}$ with $(w_i)_{i∈I} ∈ Λ_ι^I$. Now, if $v_k \not\equiv w_k$ for
  some $k ∈ I$ then we immediately get $\{(l_i = v_i)_{i∈I}\} \not\equiv
  \{(l_i = w_i)_{i∈I}\}$ using Theorem~\ref{thm:aux_reco}. As a consequence,
  it must be that $v_k ≡ w_k$ for all $k ∈ I$.
\end{proof}

To conclude this section, we provide a last theorem establishing that
$λ$-abstractions can only be equivalent to $λ$-abstractions.
\begin{theorem}[canonicity for $λ$-abstractions]\label{thm:canonlambda}%
  Let $x ∈ \mathcal{V}_{ι}$ be a $λ$-variable, $t ∈ Λ$ be a term and
  $v ∈ Λ_{ι}$ be a value. If the equivalence $λx.t ≡ v$ holds, then there
  must be $y ∈ \mathcal{V}_{ι}$ and $u ∈ Λ$ such that $v = λy.u$.
\end{theorem}
\begin{proof}
  We reason by case on the possible forms of $v$. Using
  Theorems~\ref{thm:canonbox}, \ref{thm:varequiv}, \ref{thm:canoncons}
  and~\ref{thm:canonrecord} we obtain that $v$ cannot be $□$,
  a $λ$-variable, a record nor a variant. The only remaining possibility is
  that $v = λy.u$ for some $y ∈ \mathcal{V}_{ι}$ and $u ∈ Λ$.
\end{proof}


\chapter{Higher-order formulas and typing rules}

\begin{definition}[atomic sorts]
  We denote $\sorts₀ = \{ι,τ,σ,κ,ο\}$ the set of our five atomic sorts. It
  contains the sort of values $ι$, the sort of terms $τ$, the sort of stacks
  $σ$, the sort of ordinals $κ$ and the sort of propositions $ο$.
\end{definition}

\begin{definition}[sorts]
  The set of all sorts $\sorts$ is generated from the set of atomic sorts
  $\sorts₀$ using the following grammar.
  \begin{align*}
    s,r &\bnfeq φ \bnfor s→r \bnfor s×r \tag{$φ ∈ \sorts$}
  \end{align*}
\end{definition}

\begin{definition}[variables]
  For every sort $s ∈ \sorts$ we require a (distinct) countable set of
  variables $\vars_s$. In particular, we will use the following sets.
  \begin{align*} %
    \lvars &= \{x,y,z\dots\} \tag{value variables} \\
    \svars &= \{α,β,γ\dots\} \tag{stack variables}
  \end{align*}
\end{definition}




One of the singularities of the \pml language is that it is built on
semantic intuitions. In particular, the interpretation of types precedes
the typing and subtyping rules. This means that we give ourselves a
semantics, and then only consider typing rules that are adequate with
respect to this semantics.

\begin{figure}
  \begin{align*}
    E,F \bnfeq&\\
    &\hspace{-1cm}\text{(Higher-order components)}\\
    \bnfor &\Phi                   \tag{element of a semantic domain}\\
    \bnfor &\chi                   \tag{variable}\\
    \bnfor &(\chi:s \mapsto E)     \tag{higher-order function}\\
    \bnfor &E\langle F \rangle     \tag{higher-order application}\\
    \bnfor &\varepsilon_{\chi:s}(E \in F) \bnfor
      \varepsilon_{\chi:s}(E \notin F)
      \tag{Choice operators for quantifiers}\\
    &\hspace{-1cm}\text{(Value components)}\\
    \bnfor &\lambda \chi.E
      \tag{$\lambda$-abstraction, $\chi \in \mathcal{V}_\iota$}\\
    \bnfor &C_k[E]                 \tag{constructor, $k \in \mathbb{N}$}\\
    \bnfor &\{(l_i=E_i)_{i \in I}\}\tag{record, $I \subset_{fin} \mathbb{N}$}\\
    \bnfor &\varepsilon_{\chi \in E_1}(F \notin E_2)
      \tag{value choice operator, $\chi \in \mathcal{V}_\iota$}\\
    &\hspace{-1cm}\text{(Term components)}\\
    \bnfor &E\;F                   \tag{application}\\
    \bnfor &\mu\chi.E
      \tag{$\mu$-abstraction, $\chi \in \mathcal{V}_\sigma$}\\
    \bnfor &[E]F                   \tag{named term}\\
    \bnfor &E.l_k                  \tag{record projection, $k \in \mathbb{N}$}\\
    \bnfor &[E|(C_i[\chi_i] \to E_i)_{i \in I}]
      \tag{pattern matching, $I \subset_{fin} \mathbb{N}$, $\chi_i \in
      \mathcal{V}_\iota$}\\
    \bnfor &\varphi \chi.E
      \tag{fixed-point for recursion, $\chi \in \mathcal{V}_\tau$}\\
    &\hspace{-1cm}\text{(Stack components)}\\
    \bnfor &\varepsilon            \tag{empty stack}\\
    \bnfor &[-\;E]F                \tag{pushed value argument}\\
    \bnfor &[E\;-]F                \tag{pushed function}\\
    \bnfor &\varepsilon_{\chi \in \lnot E}(F \notin E)
      \tag{value choice operator, $\chi \in \mathcal{V}_\sigma$}\\
    &\hspace{-1cm}\text{(Ordinal components)}\\
    \bnfor &\infty                 \tag{converging ordinal}\\
    \bnfor &E+1                    \tag{successor ordinal}\\
    \bnfor &\varepsilon_{\chi<E}(F_1 \in F_2) \bnfor
      \varepsilon_{\chi<E}(F_1 \notin F_2)
      \tag{Choice operators for ordinals}\\
    &\hspace{-1cm}\text{(Proposition components)}\\
    \bnfor &E \Rightarrow F        \tag{function type}\\
    \bnfor &[(C_i : A_i)_{i \in I}]
      \tag{sum type, $I \subset_{fin} \mathbb{N}$}\\
    \bnfor &\{(l_i : A_i)_{i \in I}; \dots\} \bnfor \{(l_i : A_i)_{i \in I}\}
      \tag{(extensible) product type, $I \subset_{fin} \mathbb{N}$}\\
    \bnfor &\forall \chi:s.E \bnfor \exists \chi:s.E
      \tag{Quantifiers}\\
    \bnfor &\mu_E \chi.F \bnfor \nu_E \chi.E
      \tag{sized (co-)inductive type}\\
    \bnfor &E \in F                \tag{membership type}\\
    \bnfor &E \restriction F_1 \equiv F_2 \tag{restriction type}
  \end{align*}
  \caption{Full syntax of expressions.}
  \label{expr}
\end{figure}
\begin{figure}
  \begin{prooftree}
    \AxiomC{}
    \UnaryInfC{$\Gamma, \chi : s \vdash \chi : s$}
    \DisplayProof\hfill
    \AxiomC{$\Gamma, \chi : s \vdash E : t$}
    \UnaryInfC{$\Gamma \vdash (\chi : s \mapsto t) : s \to t$}
    \DisplayProof\hfill
    \AxiomC{$\Gamma \vdash E : s \to t$}
    \AxiomC{$\Gamma \vdash F : s$}
    \BinaryInfC{$\Gamma \vdash E \langle F \rangle : t$}
  \end{prooftree}
  \begin{prooftree}
    \AxiomC{$\Phi \in \sem{s}$}
    \UnaryInfC{$\Gamma \vdash \Phi : s$}
    \DisplayProof\hfill
    \AxiomC{$\Gamma \vdash E : \tau$}
    \AxiomC{$\Gamma, \chi:s \vdash F : o$}
    \BinaryInfC{$\Gamma \vdash \varepsilon_{\chi:s}(E \in F) : s$}
    \DisplayProof\hfill
    \AxiomC{$\Gamma \vdash E : \tau$}
    \AxiomC{$\Gamma, \chi:s \vdash F : o$}
    \BinaryInfC{$\Gamma \vdash \varepsilon_{\chi:s}(E \notin F) : s$}
  \end{prooftree}
  \begin{prooftree}
    \AxiomC{$\Gamma, \chi : \iota \vdash E : \tau$}
    \UnaryInfC{$\Gamma \vdash \lambda \chi E : \iota$}
    \DisplayProof\hfill
    \AxiomC{$\Gamma \vdash E : \iota$}
    \UnaryInfC{$\Gamma \vdash C_k[E] : \iota$}
    \DisplayProof\hfill
    \AxiomC{$(\Gamma \vdash E_i : \iota)_{i \in I}$}
    \UnaryInfC{$\Gamma \vdash \{(l_i = E_i)_{i \in I}\} : \iota$}
  \end{prooftree}
  \begin{prooftree}
    \AxiomC{$\Gamma \vdash E_1 : o$}
    \AxiomC{$\Gamma \vdash E_2 : o$}
    \AxiomC{$\Gamma, \chi : \iota \vdash F : \tau$}
    \TrinaryInfC{$\Gamma\vdash\varepsilon_{\chi\in E_1}(F\notin E_2) : \iota$}
    \DisplayProof\hfill
    \AxiomC{$\Gamma \vdash E : \iota$}
    \UnaryInfC{$\Gamma \vdash E : \tau$}
  \end{prooftree}
  \begin{prooftree}
    \AxiomC{$\Gamma \vdash E : \tau$}
    \AxiomC{$\Gamma \vdash F : \tau$}
    \BinaryInfC{$\Gamma \vdash E\;F : \tau$}
    \DisplayProof\hfill
    \AxiomC{$\Gamma, \chi : \sigma \vdash E : \tau$}
    \UnaryInfC{$\Gamma \vdash \mu \chi.E : \tau$}
    \DisplayProof\hfill
    \AxiomC{$\Gamma \vdash E : \sigma$}
    \AxiomC{$\Gamma \vdash F : \tau$}
    \BinaryInfC{$\Gamma \vdash [E]F : \tau$}
  \end{prooftree}
  \begin{prooftree}
    \AxiomC{$\Gamma \vdash E : \iota$}
    \UnaryInfC{$\Gamma \vdash E.l_k : \tau$}
    \DisplayProof\hfill
    \AxiomC{$\Gamma \vdash E : \iota$}
    \AxiomC{$(\Gamma, \chi_i : \iota \vdash E_i : \tau)_{i \in I}$}
    \BinaryInfC{$\Gamma \vdash [E|(C_i[\chi_i] \to E_i)_{i \in I}] : \tau$}
    \DisplayProof\hfill
    \AxiomC{$\Gamma, \chi : \tau \vdash E : \iota$}
    \UnaryInfC{$\Gamma \vdash \varphi \chi.E : \tau$}
  \end{prooftree}
  \begin{prooftree}
    \AxiomC{}
    \UnaryInfC{$\Gamma \vdash \varepsilon : \sigma$}
    \DisplayProof\hfill
    \AxiomC{$\Gamma \vdash E : \iota$}
    \AxiomC{$\Gamma \vdash F : \sigma$}
    \BinaryInfC{$\Gamma \vdash [-\;E]F : \sigma$}
    \DisplayProof\hfill
    \AxiomC{$\Gamma \vdash E : \tau$}
    \AxiomC{$\Gamma \vdash F : \sigma$}
    \BinaryInfC{$\Gamma \vdash [E\;-]F : \sigma$}
  \end{prooftree}
  \begin{prooftree}
    \AxiomC{$\Gamma \vdash E : o$}
    \AxiomC{$\Gamma, \chi : \sigma \vdash F : \tau$}
    \BinaryInfC{$\Gamma \vdash \varepsilon_{\chi \in \lnot E}(F \notin E)
      : \sigma$}
    \DisplayProof\hfill
    \AxiomC{}
    \UnaryInfC{$\Gamma \vdash \infty : \kappa$}
    \DisplayProof\hfill
    \AxiomC{$\Gamma \vdash E : \kappa$}
    \UnaryInfC{$\Gamma \vdash E+1 : \kappa$}
  \end{prooftree}
  \begin{prooftree}
    \AxiomC{$\Gamma \vdash E : \kappa$}
    \AxiomC{$\Gamma \vdash F_1 : \tau$}
    \AxiomC{$\Gamma, \chi : \kappa \vdash F_2 : o$}
    \TrinaryInfC{$\Gamma \vdash \varepsilon_{\chi<E}(F_1 \in F_2) : \kappa$}
    \DisplayProof\hfill
    \AxiomC{$\Gamma, \chi : s \vdash E : o$}
    \UnaryInfC{$\Gamma \vdash \forall \chi:s.E : o$}
  \end{prooftree}
  \begin{prooftree}
    \AxiomC{$\Gamma \vdash E : \kappa$}
    \AxiomC{$\Gamma \vdash F_1 : \tau$}
    \AxiomC{$\Gamma, \chi : \kappa \vdash F_2 : o$}
    \TrinaryInfC{$\Gamma \vdash \varepsilon_{\chi<E}(F_1 \notin F_2):\kappa$}
    \DisplayProof\hfill
    \AxiomC{$\Gamma, \chi : s \vdash E : o$}
    \UnaryInfC{$\Gamma \vdash \exists \chi:s.E : o$}
  \end{prooftree}
  \begin{prooftree}
    \AxiomC{$\Gamma \vdash E : o$}
    \AxiomC{$\Gamma \vdash F : o$}
    \BinaryInfC{$\Gamma \vdash E \Rightarrow F : o$}
    \DisplayProof\hfill
    \AxiomC{$\Gamma \vdash E : o$}
    \AxiomC{$\Gamma \vdash F_1 : \tau$}
    \AxiomC{$\Gamma \vdash F_2 : \tau$}
    \TrinaryInfC{$\Gamma \vdash E \restriction F_1 \equiv F_2 : o$}
  \end{prooftree}
  \begin{prooftree}
    \AxiomC{$(\Gamma \vdash A_i : o)_{i \in I}$}
    \UnaryInfC{$\Gamma \vdash \{(l_i : A_i)_{i \in I}; \dots\} : o$}
    \DisplayProof\hfill
    \AxiomC{$(\Gamma \vdash A_i : o)_{i \in I}$}
    \UnaryInfC{$\Gamma \vdash \{(l_i : A_i)_{i \in I}\} : o$}
    \DisplayProof\hfill
    \AxiomC{$\Gamma \vdash E : \tau$}
    \AxiomC{$\Gamma \vdash F : o$}
    \BinaryInfC{$\Gamma \vdash E \in F : o$}
  \end{prooftree}
  \begin{prooftree}
    \AxiomC{$(\Gamma \vdash A_i : o)_{i \in I}$}
    \UnaryInfC{$\Gamma \vdash [(C_i : A_i)_{i \in I}] : o$}
    \DisplayProof\hfill
    \AxiomC{$\Gamma \vdash E : \kappa$}
    \AxiomC{$\Gamma, \chi : o \vdash F : o$}
    \BinaryInfC{$\Gamma \vdash \mu_E \chi.F : o$}
    \DisplayProof\hfill
    \AxiomC{$\Gamma \vdash E : \kappa$}
    \AxiomC{$\Gamma, \chi : o \vdash F : o$}
    \BinaryInfC{$\Gamma \vdash \nu_E \chi.F : o$}
  \end{prooftree}
  \caption{Sorting rules of the language.}
  \label{sorting}
\end{figure}
As the type system of \pml is higher-order, we first need to define the
syntax of expressions. It will gather the types, but also the terms and
several sort of first-order objects. This will include pre-values, pre-terms,
pre-stacks, syntactic ordinals and propositions.
\begin{definition}[sorts]
  Our language is built with expressions of several base sorts: $\iota$ for
  values, $\tau$ for terms, $\sigma$ for stacks, $\kappa$ for ordinals, and
  $o$ for propositions (or types). There is also a function sort to
  build parametric expressions. The set of all sorts $\mathcal{S}$ is
  generated by the following BNF grammar.
  \begin{align*}
    s,r \bnfeq &\iota \bnfor \tau \bnfor \sigma \bnfor \kappa
        \bnfor o \bnfor s \to t
  \end{align*}
\end{definition}
\begin{definition}[semantic domain]
  For every sort $s$ we define a set $\sem{s}$ called its
  semantic domain. It is defined inductively as follows.
  $$
    \sem{\iota } = \Lambda_\iota
    \quad\quad
    \sem{\tau  } = \Lambda_\tau
    \quad\quad
    \sem{\sigma} = \Pi
    \quad\quad
    \sem{\kappa} = \Omega + \omega
    \quad\quad
    \sem{s \to r} =
      \sem{r}^{\sem{s}}
  $$
  $$
    \sem{o     } =
      \{P \in \mathcal{P}(\Lambda_\iota / \equiv) \;|\; \square \in P\}
  $$
  Note that $\Omega$ denotes an ordinal that is large enough for semantics
  of our inductive and coinductive type to reach a stationary point. It must
  exist by a cardinality argument (see previous work \cite{subml}).
\end{definition}
\begin{definition}[higher-order expressions]
  The syntax of higher-order expressions is defined in Figure~\ref{expr},
  and valid expressions are those that can be assigned a sort using the
  rules of Figure~\ref{sorting}. An expression of sort $o$ will be called
  a type.
\end{definition}

\begin{definition}[semantics of closed parametric expressions]
  Let $E$ be a higher-order expression such that $\Gamma \vdash E : s$ is
  derivable using the rules of Figure~\ref{sorting}. Given a semantical
  map $\rho$ associating every $\chi \in FV(E)$ to an element of
  $\sem{\Gamma(\chi)}$, the expression $E\rho$ is called
  a closed parametric expression. Every such closed parametric expression is
  given a semantic interpretation $\sem{E\rho}$ according to
  Figures~\ref{semany}, \ref{semiota}, \ref{semtau}, \ref{semsigma},
  \ref{semkappa} or \ref{semomicron} depending on the sort $s$.
\end{definition}

\begin{lemma}[semantics of sorted expressions]
  If $\Gamma \vdash E : s$ is derivable using the rules of
  Figure~\ref{sorting} and if $\rho$ is a semantical map corresponding to
  $\Gamma$, then $\sem{E\rho} \in \sem{s}$.
\end{lemma}
\begin{proof}
  See the author's PhD thesis \cite[Chapter~4]{lepigrePhD}.
\end{proof}

\begin{figure}
  \begin{align*}
    \sem{\Phi} &= \Phi\\
    \sem{\chi} &= \chi\\
    \sem{(\chi:s \mapsto E)}
      &= (\chi:s \mapsto \sem{E})\\
    \sem{E\langle F \rangle}
      &= \sem{E} \langle \sem{F} \rangle\\
    \sem{\varepsilon_{\chi:s}(t \in A)}
      &= \begin{cases}
           \Phi \in \sem{s} \text{ such that}
             \sem{t} \in \sem{A[\chi := \Phi]
            }^{\bot\bot} \text{ if any}\\
           \text{ an arbitrary element of} \sem{s}
             \text { otherwise}
         \end{cases}\\
    \sem{\varepsilon_{\chi:s}(t \notin A)}
      &= \begin{cases}
           \Phi \in \sem{s} \text{ such that}
             \sem{t} \notin \sem{A[\chi := \Phi]
            }^{\bot\bot} \text{ if any}\\
           \text{ an arbitrary element of} \sem{s}
             \text { otherwise}
         \end{cases}
  \end{align*}
  \caption{Semantics for elements of arbitrary sort.}\label{semany}
\end{figure}

\begin{figure}
  \begin{align*}
    \sem{\lambda x.t}
      &= \lambda x.\sem{t}\\
    \sem{C_k[v]}
      &= C_k[\sem{v}]\\
    \sem{\{(l_i=v_i)_{i \in I}\}}
      &= \{(l_i = \sem{v_i})_{i \in I}\}\\
    \sem{\varepsilon_{x \in A}(t \notin B)}
      &= \begin{cases}
           v \in \sem{A} \setminus \{\square\}
             \text{ such that} \sem{t[x := v]} \notin
             \sem{B}^{\bot\bot} \text{ if any}\\
           \square \text { otherwise}
         \end{cases}
  \end{align*}
  \caption{Semantics for values (of sort $\iota$).}
  \label{semiota}
\end{figure}

\begin{figure}
  \begin{align*}
    \sem{t\;u}
      &= \sem{t}\; \sem{u}\\
    \sem{\mu\alpha.t}
      &= \mu \alpha.\sem{t}\\
    \sem{[\pi]t}
      &= [\sem{\pi}]\sem{t}\\
    \sem{v.l_k}
      &= \sem{v}.l_k\\
    \sem{[v|(C_i[x_i] \to t_i)_{i \in I}]}
      &= [\sem{v} | (C_i[x_i]
         \to \sem{t_i})_{i \in I}]\\
    \sem{\varphi a.v}
      &= \varphi a.\sem{v}
  \end{align*}
  \caption{Semantics for terms (of sort $\tau$).}
  \label{semtau}
\end{figure}

\begin{figure}
  \begin{align*}
    \sem{\varepsilon} &= \varepsilon\\
    \sem{[-\;v]\pi}
      &= [-\;\sem{v}]\sem{\pi}\\
    \sem{[t\;-]\pi}
      &= [\sem{t}\;-]\sem{\pi}\\
    \sem{\varepsilon_{\alpha \in \lnot A}(t \notin A)}
      &= \begin{cases}
           \pi \in \sem{A}^\bot \text{ such that}
             \sem{t[\alpha := \pi]} \notin
             \sem{B}^{\bot\bot} \text{ if any}\\
           [\square\;-]\varepsilon \text { otherwise}
         \end{cases}
  \end{align*}
  \caption{Semantics for stacks (of sort $\sigma$).}
  \label{semsigma}
\end{figure}

\begin{figure}
  \begin{align*}
    \sem{\infty} &= \Omega\\
    \sem{\tau+1} &= \sem{\tau} + 1\\
    \sem{\varepsilon_{\theta<\tau}(t \in A)}
      &= \begin{cases}
            o < \sem{\tau} \text{ such that}
              \sem{t} \in \sem{A[\theta := o]
             }^{\bot\bot}\\
            0 \text{ otherwise}
         \end{cases}\\
    \sem{\varepsilon_{\theta<\tau}(t \notin A)}
      &= \begin{cases}
            o < \sem{\tau} \text{ such that}
              \sem{t} \notin \sem{A[\theta := o]
             }^{\bot\bot}\\
            0 \text{ otherwise}
         \end{cases}
  \end{align*}
  \caption{Semantics for syntactic ordinals (of sort $\kappa$).}
  \label{semkappa}
\end{figure}

\begin{figure}
  \begin{align*}
    \sem{A \Rightarrow B} &=
      \{\lambda x.t \;|\; \forall v \in \sem{A} \setminus
      \{\square\},
      t[x := v] \in \sem{B}^{\bot\bot}\} \cup \{\square\}\\
    \sem{[(C_i : A_i)_{i \in I}]}
      &= \bigcup_{i \in I} \{C_i[v] \;|\; v \in \sem{A_i}
           \setminus \{\square\}\} \cup \{\square\}\\
    \sem{\{(l_i : A_i)_{i \in I}; \dots\}}
      &= \{\{(l_i = v_i)_{i \in I \cup K}\} \;|\; \forall i \in I, v_i \in \sem{A_i}
           \setminus \{\square\}\} \cup \{\square\}\\
    \sem{\{(l_i : A_i)_{i \in I}\}}
      &= \{\{(l_i = v_i)_{i \in I}\} \;|\; \forall i \in I, v_i \in
    \sem{A_i}
           \setminus \{\square\}\} \cup \{\square\}\\
    \sem{\forall \chi:s.A}
      &= \bigcap_{\Phi \in \sem{s}}
           \sem{A[\chi := \Phi]}\\
    \sem{\exists \chi:s.A}
      &= \bigcup_{\Phi \in \sem{s}}
           \sem{A[\chi := \Phi]}\\
    \sem{\mu_\tau X.A}
      &= \bigcup_{o < \sem{\tau}}
            \sem{A[X := \mu_o X.A]} \cup \{\square\}\\
    \sem{\nu_\tau X.A}
      &= \bigcap_{o < \sem{\tau}}
            \sem{A[X := \nu_o X.A]}\\
    \sem{t \in A}
      &= \{v \in \sem{A} \;|\; \sem{t}
      \equiv v\} \cup \{\square\}\\
    \sem{A \restriction t \equiv u}
      &= \begin{cases}
           \sem{A} \text{ if} \sem{t}
             \equiv \sem{u}\\
           \{\square\} \text{ otherwise}
         \end{cases}
  \end{align*}
  \caption{Semantics for proposition (of sort $o$).}\label{semomicron}
\end{figure}

\begin{definition}[judgements]
  The type system of \pml relies on four forms of judgements, where $\gamma$
  denotes a positivity context (a set of syntactic ordinals assumed to be
  non-zero), and $\Xi$ denotes an equational context (a set of program
  equivalences assumed to be correct).
  \begin{itemize}
    \item Typing judgments of the form $\gamma; \Xi \vdash t : A$, derived
          using the rules of Figure~\ref{typing}.
    \item Local subtyping judgements of the form $\gamma; \Xi \vdash t : A
          \subseteq B$, derived using the rules of Figure~\ref{subtyping}
          (and we use the notation $\gamma; \Xi \vdash A \subseteq B$ for
          $\gamma;\Xi\vdash\varepsilon_{x \in a}(x \notin B):A \subseteq B$).
    \item Equivalence decision judgements of the form $\Xi \vdash t \equiv u$
          (and we use the notation $\Xi \vdash v \not\equiv \square$ as a
          shorthand for $\Xi \vdash (\lambda x.\{\})\;v \equiv \{\}$). Such
          a judgement is not proved using derivation rules, but using an
          abstract decision procedure (that is necessarily incomplete).
    \item Syntactic ordinal ordering judgements of the form $\gamma \vdash
          \upsilon < \tau$ derived using similar rules as in previous work
          \cite{subml}.
  \end{itemize}
  Note that we implicitly consider that every syntactic element in our
  judgements is well-sorted, according to the rules of Figure~\ref{sorting}.
\end{definition}

\begin{figure}
  \begin{prooftree}
    \AxiomC{$\gamma; \Xi \vdash \varepsilon_{x \in A}(t \notin B) : A
      \subseteq C$}
    \AxiomC{$\Xi \vdash \varepsilon_{x \in A}(t \notin B) \not= \square$}
    \RightLabel{$\varepsilon$-axiom}
    \BinaryInfC{$\gamma; \Xi \vdash \varepsilon_{x \in A}(t \notin B) : C$}
  \end{prooftree}
  \begin{prooftree}
    \AxiomC{$\gamma; \Xi \vdash \lambda x.t : A \Rightarrow B \subseteq C$}
    \AxiomC{$\gamma; \Xi, \varepsilon_{x \in A}(t \notin B) \not=
      \square \vdash t[x := \varepsilon_{x \in A}(t \notin B)] : B$}
    \RightLabel{$\Rightarrow$-intro}
    \BinaryInfC{$\gamma; \Xi \vdash \lambda x.t : C$}
  \end{prooftree}
  \begin{prooftree}
    \AxiomC{$\gamma; \Xi \vdash t : A \Rightarrow B$}
    \AxiomC{$\gamma; \Xi \vdash u : A$}
    \RightLabel{$\Rightarrow$-elim}
    \BinaryInfC{$\gamma; \Xi \vdash t\;u : B$}
  \end{prooftree}
  \begin{prooftree}
    \AxiomC{$\gamma; \Xi \vdash t : (u \in A) \Rightarrow B$}
    \AxiomC{$\gamma; \Xi \vdash u : A$}
    \AxiomC{$\Xi \vdash u \equiv v$}
    \RightLabel{$\Rightarrow$-strong-elim}
    \TrinaryInfC{$\gamma; \Xi \vdash t\;u : B$}
  \end{prooftree}
  \begin{prooftree}
    \AxiomC{$\gamma; \Xi \vdash t[\alpha := \varepsilon_{\alpha \in
      \lnot A}(t \notin A)] : A$}
    \RightLabel{save}
    \UnaryInfC{$\gamma; \Xi \vdash \mu \alpha.t : A$}
    \DisplayProof\hfill
    \AxiomC{$\gamma; \Xi \vdash u : A$}
    \RightLabel{restore}
    \UnaryInfC{$\gamma; \Xi \vdash [\varepsilon_{\alpha\in\lnot A}(t \notin
      A)]u : B$}
  \end{prooftree}
  \begin{prooftree}
    \AxiomC{$\gamma; \Xi \vdash v[a := \varphi a.v] : A$}
    \RightLabel{unfold}
    \UnaryInfC{$\gamma; \Xi \vdash \varphi a.v : A$}
    \DisplayProof\hfill
    \AxiomC{$\gamma; \Xi \vdash v : \{l_k : A; \dots\}$}
    \RightLabel{$\times$-elim}
    \UnaryInfC{$\gamma; \Xi \vdash v.l_k : A$}
  \end{prooftree}
  \begin{prooftree}
    \AxiomC{$\gamma; \Xi \vdash \{(l_i = v_i)_{i \in I}\} :
      \{(l_i : A_i)_{i \in I}\} \subseteq C$}
    \AxiomC{$(\gamma; \Xi \vdash v_i : A_i)_{i \in I}$}
    \RightLabel{$\times$-intro}
    \BinaryInfC{$\gamma; \Xi \vdash \{(l_i = v_i)_{i \in I}\} : C$}
  \end{prooftree}
  \begin{prooftree}
    \AxiomC{$\gamma; \Xi \vdash v : A$}
    \AxiomC{$\gamma; \Xi \vdash C_k[v] : [C_k : A] \subseteq B$}
    \RightLabel{$+$-intro}
    \BinaryInfC{$\gamma; \Xi \vdash C_k[v] : B$}
  \end{prooftree}
  \begin{prooftree}
    \AxiomC{$\gamma; \Xi \vdash v : [(C_i : A_i)_{i \in I}]$}
    \AxiomC{$(\gamma; \Xi, w_i \not= \square
      \vdash t_i[x_i := w_i] : C)_{i \in I}$}
    \RightLabel{$+$-elim}
    \BinaryInfC{$\gamma; \Xi \vdash [v|(C_i[x_i] \to t_i)_{i \in I}] : C$}
  \end{prooftree}
  \begin{center}
    \vspace{-2mm}
    where $w_i = \varepsilon_{x_i \in A_i \restriction {C_i[x_i] \,\equiv\,
    v}}(t_i \notin C)$ for all $i \in I$
  \end{center}
  \begin{prooftree}
    \AxiomC{$\gamma; \Xi \vdash u : A$}
    \AxiomC{$\Xi \vdash t \equiv u$}
    \RightLabel{replace}
    \BinaryInfC{$\gamma; \Xi \vdash t : A$}
  \end{prooftree}
  \caption{Typing rules for terms (including values).}\label{typing}
\end{figure}

\begin{figure}
  \begin{prooftree}
    \AxiomC{$(\Xi \vdash \rho_1(\chi) \equiv \rho_2(\chi))_{\chi \in FV(A)}$}
    \RightLabel{$\subseteq$-refl}
    \UnaryInfC{$\gamma; \Xi \vdash t : A\rho_1 \subseteq A\rho_2$}
    \DisplayProof\hfill
    \AxiomC{$\gamma; \Xi \vdash A \subseteq B$}
    \RightLabel{gen}
    \UnaryInfC{$\gamma; \Xi \vdash t : A \subseteq B$}
  \end{prooftree}
  \begin{prooftree}
    \AxiomC{$\gamma; \Xi, w \not= \square \vdash w : A_2 \subseteq A_1$}
    \AxiomC{$\gamma; \Xi, w \not= \square \vdash t\;w : B_1 \subseteq B_2$}
    \AxiomC{$\Xi \vdash t \equiv v$}
    \RightLabel{$\Rightarrow$-sub}
    \TrinaryInfC{$\gamma; \Xi \vdash t : A_1 \Rightarrow B_1
      \subseteq A_2 \Rightarrow B_2$}
  \end{prooftree}
  \begin{center}
    \vspace{-2mm}
    where $w = \varepsilon_{x \in A_2}(t\;x \notin B_2)$
  \end{center}
  \begin{prooftree}
    \AxiomC{$I_1 \subseteq I_2$}
    \AxiomC{$(\gamma; \Xi \vdash (\lambda x.[x | C_i[x_i] \to x_i])\;t : A_i
      \subseteq B_i)_{i \in I_1}$}
    \AxiomC{$\Xi \vdash t \equiv v$}
    \RightLabel{$+$-sub}
    \TrinaryInfC{$\gamma; \Xi \vdash t : [(C_i : A_i)_{i \in I_1}]
      \subseteq [(C_i : B_i)_{i \in I_2}]$}
  \end{prooftree}
  \begin{prooftree}
    \AxiomC{$(\gamma; \Xi \vdash (\lambda x.x.l_i)\;t : A_i
      \subseteq B_i)_{i \in I}$}
    \AxiomC{$\Xi \vdash t \equiv v$}
    \RightLabel{$\times$-sub-strict}
    \BinaryInfC{$\gamma; \Xi \vdash t : \{(l_i : A_i)_{i \in I}\}
      \subseteq \{(l_i : B_i)_{i \in I}\}$}
  \end{prooftree}
  \begin{prooftree}
    \AxiomC{$I_2 \subseteq I_1$}
    \AxiomC{$(\gamma; \Xi \vdash (\lambda x.x.l_i)\;t:A_i
      \subseteq B_i)_{i \in I_2}$}
    \AxiomC{$\Xi \vdash t \equiv v$}
    \RightLabel{$\times$-sub-ext / $\times$-sub-mixed}
    \TrinaryInfC{$\gamma; \Xi \vdash t : \{(l_i : A_i)_{i \in I_1}; \dots\}
      \subseteq \{(l_i : B_i)_{i \in I_2} [; \dots]\}$}
  \end{prooftree}
  \begin{prooftree}
    \AxiomC{$\gamma; \Xi \vdash t : A[\chi := \psi] \subseteq B$}
    \RightLabel{$\forall$-left}
    \UnaryInfC{$\gamma; \Xi \vdash t : \forall \chi^s.A \subseteq B$}
    \DisplayProof\hfill
    \AxiomC{$\gamma; \Xi \vdash t : A \subseteq B[\chi := \psi]$}
    \RightLabel{$\exists$-right}
    \UnaryInfC{$\gamma; \Xi \vdash t : A \subseteq \exists \chi^s.B$}
  \end{prooftree}
  \begin{prooftree}
    \AxiomC{$\gamma; \Xi \vdash t : A \subseteq
      B[\chi := \varepsilon_{\chi:s}(t \notin B)]$}
    \AxiomC{$\Xi \vdash t \equiv v$}
    \RightLabel{$\forall$-right}
    \BinaryInfC{$\gamma; \Xi \vdash t : A \subseteq \forall \chi^s.B$}
  \end{prooftree}
  \begin{prooftree}
    \AxiomC{$\gamma; \Xi \vdash
      t : A[\chi := \varepsilon_{\chi:s}(t \notin A)] \subseteq B$}
    \AxiomC{$\Xi \vdash t \equiv v$}
    \RightLabel{$\exists$-left}
    \BinaryInfC{$\gamma; \Xi \vdash t : \exists \chi^s.A \subseteq B$}
  \end{prooftree}
  \begin{prooftree}
    \AxiomC{$\gamma; \Xi, u_1 \equiv u_2 \vdash t : A \subseteq B$}
    \AxiomC{$\Xi \vdash t \equiv v$}
    \RightLabel{$\restriction$-left}
    \BinaryInfC{$\gamma; \Xi \vdash
      t : A \restriction u_1 \equiv u_2 \subseteq B$}
    \DisplayProof\hfill
    \AxiomC{$\gamma; \Xi \vdash t : A \subseteq B$}
    \AxiomC{$\Xi \vdash u_1 \equiv u_2$}
    \RightLabel{$\restriction$-right}
    \BinaryInfC{$\gamma; \Xi \vdash t : A \subseteq
      B \restriction u_1 \equiv u_2$}
  \end{prooftree}
  \begin{prooftree}
    \AxiomC{$\gamma; \Xi, t \equiv u \vdash t : A \subseteq B$}
    \AxiomC{$\Xi \vdash t \equiv v$}
    \RightLabel{$\in$-left}
    \BinaryInfC{$\gamma; \Xi \vdash t : u \in A \subseteq B$}
    \DisplayProof\hfill
    \AxiomC{$\gamma; \Xi \vdash t : A \subseteq B$}
    \AxiomC{$\Xi \vdash t \equiv u$}
    \RightLabel{$\in$-right}
    \BinaryInfC{$\gamma; \Xi \vdash t : A \subseteq u \in B$}
  \end{prooftree}
  \begin{prooftree}
    \AxiomC{$\gamma; \Xi \vdash t : A \subseteq B[X := \mu_\infty X.B]$}
    \RightLabel{$\mu_\infty$-left}
    \UnaryInfC{$\gamma; \Xi \vdash t : A \subseteq \mu_\infty X.B$}
    \DisplayProof\hfill
    \AxiomC{$\gamma; \Xi \vdash t : A[X := \nu_\infty X.A] \subseteq B$}
    \RightLabel{$\nu_\infty$-right}
    \UnaryInfC{$\gamma; \Xi \vdash t : \nu_\infty X.A \subseteq B$}
  \end{prooftree}
  \begin{prooftree}
    \AxiomC{$\gamma; \Xi \vdash t : A \subseteq B[X := \mu_\upsilon X.B]$}
    \AxiomC{$\gamma \vdash \upsilon < \tau$}
    \RightLabel{$\mu$-right}
    \BinaryInfC{$\gamma; \Xi \vdash t : A \subseteq \mu_\tau X.B$}
  \end{prooftree}
  \begin{prooftree}
    \AxiomC{$\gamma; \Xi \vdash t : A[X := \nu_\upsilon X.A] \subseteq B$}
    \AxiomC{$\gamma \vdash \upsilon < \tau$}
    \RightLabel{$\nu$-left}
    \BinaryInfC{$\gamma; \Xi \vdash t : \nu_\tau X.A \subseteq B$}
  \end{prooftree}
  \begin{prooftree}
    \AxiomC{$\gamma, \tau; \Xi \vdash t : A[X := \mu_{\varepsilon_{\theta <
      \tau}(t \in A[X := \mu_\theta X.A])}X.A] \subseteq B$}
    \AxiomC{$\Xi \vdash t \equiv v$}
    \RightLabel{$\mu$-left}
    \BinaryInfC{$\gamma; \Xi \vdash t : \mu_\tau X.A \subseteq B$}
  \end{prooftree}
  \begin{prooftree}
    \AxiomC{$\gamma, \tau; \Xi \vdash t : A \subseteq B[X :=
      \nu_{\varepsilon_{\theta < \tau}(t \notin B[X := \nu_\theta X.B])}X.B]$}
    \AxiomC{$\Xi \vdash t \equiv v$}
    \RightLabel{$\nu$-right}
    \BinaryInfC{$\gamma; \Xi \vdash t : A \subseteq \nu_\tau X.B$}
  \end{prooftree}
  \caption{Local subtyping rules.}\label{subtyping}
\end{figure}

\begin{theorem}[adequacy]
  We have the following adequacy properties.
  \begin{itemize}
    \item If $\gamma; \Xi \vdash t : A$ is derivable and if the contexts
          $\gamma$ and $\Xi$ are realized, then $\sem{t}
          \in \sem{A}^{\bot\bot}$. Moreover, if $t$ is a
          value then $\sem{t} \not= \square$.
    \item If $\gamma; \Xi \vdash t : A \subseteq B$ is derivable, the contexts
          $\gamma$ and $\Xi$ are realized, and $\sem{t}
          \in \sem{A}^{\bot\bot}$, then $\sem{t
         } \in \sem{B}^{\bot\bot}$, In particular,
          if the judgment is of the form $\gamma; \Xi \vdash A \subseteq B$
          then $\sem{A} \subseteq \sem{B}$.
  \end{itemize}
\end{theorem}
\begin{proof}
  See the author's PhD thesis \cite[Chapter~4]{lepigrePhD}. An essential
  point of the semantics is that the bi-orthogonal completion operator
  $\Psi \mapsto \Psi^{\bot\bot}$ on the elements of $\sem{o}$
  is closed for values (i.e., any value in $\Psi^{\bot\bot}$ already appears
  in $\Psi$). The complex definitions of reduction and observational
  equivalence allow us to construct a model with this property (see previous
  work for more details \cite{lepigre2016,lepigrePhD}).
\end{proof}
\begin{corollary}[proof of equivalence]
  If $\vdash t : \{\} \restriction u_1 \equiv u_2$ is derivable for
  user-accessible terms $t$, $u_1$ and $u_2$, then the equivalence
  $u_1 \equiv u_2$ holds.
\end{corollary}


\chapter{Realizability semantics and adequacy}

\input{part2_theory/chap3_semantics/semantics.tex}


\part{The implementation}

\input{part3_implem/implem.tex}

\bibliographystyle{plain}
\bibliography{biblio}

\end{document}
